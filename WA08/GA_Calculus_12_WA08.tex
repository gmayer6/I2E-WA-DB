\documentclass{article}
\usepackage[utf8]{inputenc}
\usepackage{amsmath}
\usepackage{amssymb} % for \mathbb
\usepackage{graphicx} % for figures
\usepackage{color}
\usepackage[usenames,dvipsnames]{xcolor}
\usepackage{hyperref} % for hyperlinks
\usepackage{float} % for figures

\usepackage{fancyheadings}
\pagestyle{fancy}

%\rfoot[\thesection]{}
% ~ ~ ~ ~ ~ ~ ~ ~ ~ ~ ~ ~ ~ ~ ~ ~ ~ ~ ~ ~ ~ ~ ~ ~ ~ ~ ~ ~ ~ ~ ~ ~ ~ ~ ~ ~ ~ ~ ~ ~ ~ ~ ~ ~ ~ ~ ~ ~
% ENUMERATION

\usepackage{enumitem}   % so that question numbers can be formatted 
\setenumerate[1]{label=\thesubsection.\arabic*.} % enumerate environment: add section numbers to items
% ~ ~ ~ ~ ~ ~ ~ ~ ~ ~ ~ ~ ~ ~ ~ ~ ~ ~ ~ ~ ~ ~ ~ ~ ~ ~ ~ ~ ~ ~ ~ ~ ~ ~ ~ ~ ~ ~ ~ ~ ~ ~ ~ ~ ~ ~ ~ ~
% MARGINS
\usepackage{anysize}
\marginsize{2.5cm}{2.5cm}{1cm}{1cm}
% ~ ~ ~ ~ ~ ~ ~ ~ ~ ~ ~ ~ ~ ~ ~ ~ ~ ~ ~ ~ ~ ~ ~ ~ ~ ~ ~ ~ ~ ~ ~ ~ ~ ~ ~ ~ ~ ~ ~ ~ ~ ~ ~ ~ ~ ~ ~ ~
% PAGE NUMBERING
\pagenumbering{arabic}
% ~ ~ ~ ~ ~ ~ ~ ~ ~ ~ ~ ~ ~ ~ ~ ~ ~ ~ ~ ~ ~ ~ ~ ~ ~ ~ ~ ~ ~ ~ ~ ~ ~ ~ ~ ~ ~ ~ ~ ~ ~ ~ ~ ~ ~ ~ ~ ~
% Custom Commands
\newcommand{\Emph}[1]{\textbf{#1}} % Emphasize
\newcommand{\R}{\mathbb{R}} 
\newcommand{\BM}{\begin{bmatrix}} % Begin Matrix
\newcommand{\EM}{\end{bmatrix}} % End Matrix
\newcommand{\BEN}{\begin{enumerate}[leftmargin=1.1cm]}% Begin ENumerate
\newcommand{\EEN}{\end{enumerate}} % End ENumerate
\newcommand{\MB}{\mathbf} % Math Bold

\newcommand{\px}{\frac{\partial}{\partial x}} % Partial wrt x
\newcommand{\py}{\frac{\partial}{\partial y}} % Partial wrt y
\newcommand{\pxu}{\frac{\partial x}{\partial u}} % x wrt u
\newcommand{\pxv}{\frac{\partial x}{\partial v}} % x wrt v
\newcommand{\pyu}{\frac{\partial y}{\partial u}} % y wrt u
\newcommand{\pyv}{\frac{\partial y}{\partial v}} % y wrt v

\newcommand{\VCT}{\textit{Vector Calculus} by Michael Corral} % Vector Calculus Textbook
\newcommand{\CAT}{\textit{College Algebra} by Carl Stitz and Jeff Zeager} % College Algebra Textbook
\newcommand{\From}{The following questions are related to } % Questions ....
% ~ ~ ~ ~ ~ ~ ~ ~ ~ ~ ~ ~ ~ ~ ~ ~ ~ ~ ~ ~ ~ ~ ~ ~ ~ ~ ~ ~ ~ ~ ~ ~ ~ ~ ~ ~ ~ ~ ~ ~ ~ ~ ~ ~ ~ ~ ~ ~
% ONLY USED FOR EDITING
\newcommand{\rednote}[1]{{\color{red}\textit{#1}}} % Shortcut for red italics
\newcommand{\purple}[1]{{\color{Violet}\textit{#1}}} % Shortcut for red italics 
% ~ ~ ~ ~ ~ ~ ~ ~ ~ ~ ~ ~ ~ ~ ~ ~ ~ ~ ~ ~ ~ ~ ~ ~ ~ ~ ~ ~ ~ ~ ~ ~ ~ ~ ~ ~ ~ ~ ~ ~ ~ ~ ~ ~ ~ ~ ~ ~
% AUGMENTED MATRIX MACRO
% thanks to http://tex.stackexchange.com/questions/2233/whats-the-best-way-make-an-augmented-coefficient-matrix
\newenvironment{amatrix}[1]{%
  \left[\begin{array}{@{}*{#1}{c}|c@{}}
}{%
  \end{array}\right]
}



\title{}
\date{}
% ~~~~~~~~~~~~~~~~~~~~~~~~~~~~~~~~~~~~~~~~~~~~~~~~~~~~~~~~~~~~~~~~~~~~~~~~~~~~~~~~~
\begin{document}

\begin{center}
\textsc{\LARGE Written Assignment 8}\\[0.5cm]
\end{center}
\From sections 3.5 and 3.6 of \VCT.
% ~~~~~~~~~~~~~~~~~~~~~~~~~~~~~~~~~~~~~~~~~~~~~~~~~~~~~~~~~~~~~~~~~~~~~~~~~~~~~~~~~
%\section*{Wolfram Alpha Syntax}
%You may want to use Wolfram Alpha (wolframalpha.com) to check your answers. If you're not sure what syntax to use to compute double integrals with Wolfram Alpha, let's suppose that we want to determine the value of
%\begin{align*} 
%   \mathop{\int_{-2}^{-1} \!  \int_0^{x-1}}( x^{2C} +y)  dy  dx
%\end{align*}
%where $C$ is a constant. The syntax we could use to compute this particular integral is
%\begin{quote}
%  \begin{verbatim}
%    integrate x^{2C}+y dydx, x from -2 to -1 and y from 0 to (x-1)
%  \end{verbatim}
%\end{quote}
%%The syntax for single and triple integrals is similar, and you can probably figure out what would work. 
%Note also that Wolfram Alpha is very forgiving the syntax, so 
%\begin{quote}
%  \begin{verbatim}
%    int x^(2c)y cos(3y) dydx, x in -2, -1 and y in 0, 1
%  \end{verbatim}
%\end{quote}
% ~~~~~~~~~~~~~~~~~~~~~~~~~~~~~~~~~~~~~~~~~~~~~~~~~~~~~~~~~~~~~~~~~~~~~~~~~~~~~~~~~
\section*{Questions}
\BEN
% ~~~~~~~~~~~~~~~~~~~~~~~~~~~~~~~~~~~~~~~~~~~~~~~~~~~~~~~~~~~~~~~~~~~~~~~~~~~~~~~~~
\item % CHANGE OF VARIABLES TWICE
When working with double integrals in polar coordinates, a helpful simplification  ...
\begin{align*}
   \iint\limits_{R} g(x) h(y) dA 
  & = \int_a^b g(x) dx \int_c^d h(y) dy  
\end{align*}
Use this property and the transformation $x = 3u, y = 2v$ to evaluate the double integral
\begin{align*}
  \iint\limits_E x^2 \ dxdy,
\end{align*}
where $E$ is region bounded by the ellipse $4x^2 + 9y^2 = 36$. 
% ~~~~~~~~~~~~~~~~~~~~~~~~~~~~~~~~~~~~~~~~~~~~~~~~~~~~~~~~~~~~~~~~~~~~~~~~~~~~~~~~~
\item % SIMPLE CYLINDRICAL WITH CONE
Evaluate the integral
\begin{align*}
  \iiint\limits_S y^2dV,
\end{align*}
where $S$ is the solid that lies inside the cylinder $x^2+y^2 = 1$, above the plane $z=0$ and below the cone $z^2 = 9x^2 + 9y^2$. 
% ~~~~~~~~~~~~~~~~~~~~~~~~~~~~~~~~~~~~~~~~~~~~~~~~~~~~~~~~~~~~~~~~~~~~~~~~~~~~~~~~~
\item % FROM SALAS, ETC
Evaluate the integral
\begin{align*}
  \iiint\limits_V \frac{1}{\sqrt{x^2+y^2}}dV,
\end{align*}
using cylindrical coordinates, where $V$ is the region:
\begin{align*}
  0 \le &x \le 2 \\
  0 \le &y \le \sqrt{4 - x^2}\\
  0 \le &z \le \sqrt{4 - x^2}
\end{align*} 
% ~~~~~~~~~~~~~~~~~~~~~~~~~~~~~~~~~~~~~~~~~~~~~~~~~~~~~~~~~~~~~~~~~~~~~~~~~~~~~~~~~
\item % EVEN AND ODD
Determine the value of
\begin{align*}
  \mathop{\int_0^{\pi} \!\! \int_{-1}^1} x^4e^{x^2 + y^2}\sin(y) dydx.
\end{align*}
\textit{Hint: integration by parts is not necessary.}\\
\textit{push this question to a midterm or final exam?}
% ~~~~~~~~~~~~~~~~~~~~~~~~~~~~~~~~~~~~~~~~~~~~~~~~~~~~~~~~~~~~~~~~~~~~~~~~~~~~~~~~~
\item % FLOOD GATE? TRIM 14.4

\EEN % END OF QUESTIONS

%
%\noindent \textbf{References} \\
%The wind tunnel problem was based on a similar exercise in Linear Algebra and Its Applications, 4th Edition, by David C. Lay, Addison-Wesley, 2012.
%
% 

%%%%%%%%%%%%%%%%%%%%%%%%%%%%%%%%%%%%%%%%%%%%%%%%%%%%%%%%%%%%%%%%
%%%%%%%%%%%%%%%%%%%%%%%%%%%%%%%%%%%%%%%%%%%%%%%%%%%%%%%%%%%%%%%%
%%%%%%%%%%%%%%%%%%%%%%%%%%%%%%%%%%%%%%%%%%%%%%%%%%%%%%%%%%%%%%%%
%%%%%%%%%%%%%%%%%%%%%%%%%%%%%%%%%%%%%%%%%%%%%%%%%%%%%%%%%%%%%%%%
% SOLUTIONS
\newpage
\section*{Solutions}

\BEN
% ~~~~~~~~~~~~~~~~~~~~~~~~~~~~~~~~~~~~~~~~~~~~~~~~~~~~~~~~~~~~~~~~~~~~~~~~~~~~~~~~~
\item % CHANGE OF VARIABLES TWICE WITH POLAR
The Jacobian is
\begin{align*}   J &=  
  \begin{vmatrix}
   \pxu &  \pxv \\ \\
   \pyu & \pyv \\
  \end{vmatrix}
  =     \begin{vmatrix}
   3 & 0 \\ \\
   0 & 2 \\
  \end{vmatrix}
  = 6 .
 \end{align*}
We also need to find the limits of integration in the transformed integral. The region $R$ is bounded by the ellipse, $4x^2 + 9y^2 = 36$, which becomes the region bounded by the circle $u^2 + v^2 = 1$. Therefore
\begin{align*}
  \iint\limits_R x^2 \ dxdy &= \iint\limits_{u^2 + v^2 \le 1} (9u^2) 6dudv =  54 \iint\limits_{u^2 + v^2 \le 1} (u^2) dudv
\end{align*}
Switching to polar coordinates, 
\begin{align*}
  u = r\cos\theta, \quad v = r\sin\theta, \quad J = r
\end{align*}
our double integral becomes
\begin{align*}
  54 \iint\limits_{u^2 + v^2 \le 1} (u^2) dudv 
  & = 54 \mathop{\int_0^{2\pi} \!\! \int_0^1} r^3\cos^2\theta drd\theta  \\
  & = 54 \Bigg( \int_0^{2\pi} \cos^2\theta d\theta \Bigg)\Bigg( \int_0^1 r^3 dr \Bigg) \\
  & = 54 \Bigg( \int_0^{2\pi} \frac{1}{2} \big(1+\cos(2\theta) \big) d\theta \Bigg) \Bigg( \frac{1}{4} \Bigg) \\
  & =\frac{27}{4} \Bigg( \int_0^{2\pi}  \big(1+\cos(2\theta) \big) d\theta \Bigg) \\
  & =\frac{27}{4} \big(\theta+\frac{1}{2}\sin(2\theta) \big|_0^{2\pi}  \\
  & =\frac{27}{4} \big(2\pi\big)  \\
  & =\frac{27\pi}{2} 
\end{align*}
% ~~~~~~~~~~~~~~~~~~~~~~~~~~~~~~~~~~~~~~~~~~~~~~~~~~~~~~~~~~~~~~~~~~~~~~~~~~~~~~~~~
\item % SIMPLE CYLINDRICAL WITH CONE
In cylindrical coordinates, the region is described by 
\begin{align*}
  V = \{(r,\theta,z) \ | \ 0 \le r \le 1, 0 \le \theta \le 2\pi, 0 \le z \le 3r \}.
\end{align*}
Our integral becomes 
\begin{align*}
  \iiint\limits_V (r\sin\theta)^2 rdzdrd\theta 
  &=    \mathop{\int_0^{2\pi} \!\! \int_0^1  \!\! \int_0^{3r}} r^3\sin^2\theta dzdrd\theta  \\
  &=   3 \mathop{\int_0^{2\pi} \!\! \int_0^1} r^4\sin^2\theta drd\theta  \\
  &=   \frac{3}{5} \int_0^{2\pi} \sin^2\theta d\theta  \\
  &=   \frac{3}{10} \int_0^{2\pi} \big(1 - \cos(2\theta)\big) d\theta  \\
  &=   \frac{3}{10} \big(\theta - \frac{1}{2}\sin(2\theta)\big) \Big|_0^{2\pi} \\
  &=   \frac{3\pi}{5}
\end{align*}
% ~~~~~~~~~~~~~~~~~~~~~~~~~~~~~~~~~~~~~~~~~~~~~~~~~~~~~~~~~~~~~~~~~~~~~~~~~~~~~~~~~
\item % 
% ~~~~~~~~~~~~~~~~~~~~~~~~~~~~~~~~~~~~~~~~~~~~~~~~~~~~~~~~~~~~~~~~~~~~~~~~~~~~~~~~~
\item % EVEN AND ODD
The integral can be written as
\begin{align*}
  \mathop{\int_0^{\pi} \!\! \int_{-1}^1} e^{x^2 + y^2}\sin(y) dydx 
  &= \mathop{\int_0^{\pi} \!\! \int_{-1}^1} e^{x^2}e^{y^2}\sin(y) dydx  = \Bigg( \int_0^{\pi} e^{x^2} dx \Bigg)\Bigg( \int_{-1}^1 e^{y^2}\sin(y)dy \Bigg)  
\end{align*}
The second integral is the integral of an odd function over a symmetrical interval, and so is equal to zero. Therefore,
\begin{align*}
  \mathop{\int_0^{\pi} \!\! \int_{-1}^1} e^{x^2 + y^2}\sin(y) dydx =0.
\end{align*}
% ~~~~~~~~~~~~~~~~~~~~~~~~~~~~~~~~~~~~~~~~~~~~~~~~~~~~~~~~~~~~~~~~~~~~~~~~~~~~~~~~~
\EEN % END OF SOLUTIONS

\end{document}

