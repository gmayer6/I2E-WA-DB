\subsection{Double Integrals}
\FromC{\From Section 3.1 of \VCT.}

\BEN
% ~~~~~~~~~~~~~~~~~~~~~~~~~~~~~~~~~~~~~~~~~~~~~~~~~~~~~~~~~~~~~~~~~~~~~~~~~~~~~~~~~
\item % VOLUME OF SIMPLE SOLID 
\textbf{Volume of a Solid} \\
Consider the solid that lies under the plane $z = -x-2y+2$ and above the rectangle \\$\{(x,y) \ | \ -2\le x\le 0, 0\le y \le1 \}$.
\BEN
\item Sketch the solid in $\R^3$. \textit{Hint: start by plotting the points that are located on the given plane and above the corners of the rectangle. Then connect the points with solid lines.}
\item Find the volume of the solid.
\EEN
% ~~~~~~~~~~~~~~~~~~~~~~~~~~~~~~~~~~~~~~~~~~~~~~~~~~~~~~~~~~~~~~~~~~~~~~~~~~~~~~~~~
\item % FLUID MECHANICS
\textbf{Application to Fluid Mechanics} \\
In a two-dimensional, steady-state, incompressible fluid flow, the velocity \textbf{v} of the flow can be expressed as $\MB{v} = u(x,y)\MB{i} + v(x,y)\MB{j}$. The functions $u(x,y)$ and $v(x,y)$ must satisfy 
\begin{align*}
  \nabla\cdot\MB{v} = 0.
\end{align*}
\BEN
\item If $u(x,y) = x^2 + y^2$, find the most general form of $v(x,y)$. 
\item If $v(x,y) = \cos(x)$, find the most general form of $u(x,y)$.
\EEN
\textit{Hint: you are asked to find the most \Emph{general} form of functions u and v}.
% ~~~~~~~~~~~~~~~~~~~~~~~~~~~~~~~~~~~~~~~~~~~~~~~~~~~~~~~~~~~~~~~~~~~~~~~~~~~~~~~~~
\item % INTEGRAL WITH AN ABSOLUTE VALUE
\textbf{Double Integral with an Absolute Value}\\
Evaluate the double integral
\begin{align*}
  \mathop{\int_{0}^{1} \! \int_0^{2}  }y \big| x-1\big| dxdy.
\end{align*}
% ~~~~~~~~~~~~~~~~~~~~~~~~~~~~~~~~~~~~~~~~~~~~~~~~~~~~~~~~~~~~~~~~~~~~~~~~~~~~~~~~~
\item % MAXIMUM AND MINIMUM VALUES
\textbf{Double Integral Given Maximum and Minimum Values}\\
Suppose that we are given a function, $f(x,y)$, that is continuous on the rectangle 
\begin{align*}
S = [a,b]\times[c,d]
\end{align*}
in $\R^2$. Suppose also that the maximum value $f(x,y)$ on $S$ is equal to the constant $K$, and that the minimum value of $f(x,y)$ on $S$ is also equal to $K$. Evaluate
\begin{align*}
  \iint\limits_S f(x,y) dxdy.
\end{align*}
% ~~~~~~~~~~~~~~~~~~~~~~~~~~~~~~~~~~~~~~~~~~~~~~~~~~~~~~~~~~~~~~~~~~~~~~~~~~~~~~~~~
\EEN % END OF SUBSECTION

