\subsection{Line Integrals}

\BEN
%~~~~~~~~~~~~~~~~~~~~~~~~~~~~~~~~~~~~~~~~~~~~~~~~~~~~~~~~~~~~~~~~~~~~~~~~~~~~~

\item The surface area of the container is equal to the sum of area of the
bottom and the line integral with respect to arc length of the height function
along the circumference of the cylinder.  The circumference of the cylinder
can be parameterized as the curve $C$: $x(t) = 2\cos(t)$, $y(t) = 2\sin(t)$
for $0 \leq t \leq 2\pi$.

\begin{align*}
 \int_C \! f(x,y) \, ds
 &= \int_{0}^{2\pi} \! f(x(t), y(t)) \sqrt{x'(t)^2 + y'(t)^2} \, dt \\
 &= \int_{0}^{2\pi} \! (16\cos^2(t)\sin^2(t) + 2) \, dt \\
 &= (\left. 4t - \frac{1}{2}\sin{4t}) \right|_{0}^{2\pi} \\
 &= 8\pi
\end{align*}

The area of the bottom is $\pi(4)^2 = 16\pi$, so the surface area of the
container is $24\pi$.

\item We compute the work using a line integral, where the path of
integration is piecewise: $C = C_1 + C_2$, where $C_1$ is defined by
$\mathbf{r}_1(t) = t\mathbf{i} + t\mathbf{j}$ for $0 \leq t \leq 1$ and $C_2$
is defined by $\mathbf{r}_2 = (1 + t)\mathbf{i} + \mathbf{j}$ for $0 \leq t
\leq 1$.

\begin{align*}
 \int_C \! \mathbf{F}(\mathbf{r}) \cdot d\mathbf{r} \,
 &= \int_{C_1} \! \mathbf{F}(\mathbf{r_1}) \cdot d\mathbf{r_1} \, +
    \int_{C_2} \! \mathbf{F}(\mathbf{r_2}) \cdot d\mathbf{r_2} \, \\
 &= \int_0^1   \! \begin{bmatrix}
                   -kt \\
                   -kt
                  \end{bmatrix} \cdot
                  \begin{bmatrix}
                   1 \\
                   1
                  \end{bmatrix}
                  \, dt +
    \int_0^1   \! \begin{bmatrix}
                   -k(1 + t) \\
                   -k
                  \end{bmatrix} \cdot
                  \begin{bmatrix}
                   1 \\
                   0
                  \end{bmatrix}
                  \, dt \\
 &= \left. -\frac{1}{2}kt^2 \right|_0^1 +
    \left. (-kt - \frac{1}{2}kt^2) \right|_0^1 \\
 &= -2k
\end{align*}

\item We compute the work done by the wind by evaluating a line integral along
the curve $C$ defined by $\mathbf{r}(t) = t\mathbf{i} + (t^3 - t)\mathbf{j}$
for $-2 \leq t \leq 2$:

\begin{align*}
 \int_C \! \mathbf{F}(\mathbf{r}) \cdot d\mathbf{r} \,
 &= \int_{-2}^{2} \! \begin{bmatrix}
                      1 + (t)(t^3 - t) \\
                      2t^2
                     \end{bmatrix} \cdot
                     \begin{bmatrix}
                      1 \\
                      3t^2 - 1
                     \end{bmatrix} \, dt \\
 &= \int_{-2}^{2} \! [(1 + t^4 - t^2) + (6t^4 - 2t^2) \, dt \\
 &= \int_{-2}^{2} \! (7t^4 - 3t^2 + 1) \, dt \\
 &= (\left. \frac{7}{5}t^5 - t^3 + t) \right|_{-2}^{2} \\
 &= 0
\end{align*}

So there is no net work done by the wind on the airplane.

\item
\BEN
\item

A potential $F$ for $\mathbf{f}$ must satisfy the equations $\frac{\partial
F}{\partial x} = y - x^3$ and $\frac{\partial F}{\partial y} = x + 3y^2$.
Integrating the first equation with respect to $x$ yields $F(x,y) = xy -
\frac{1}{4}x^4 + g(y)$ for some unknown function $g$ of $y$.  Integrating the
second equation with respect to $y$ yields $F(x,y) = xy + y^3 + h(x)$ for some
unknown function $h$ of $x$.  We can satisfy both of these by setting $h(x) =
-\frac{1}{4}x^4$ and $g(y) = y^3$.  Therefore $F(x,y) = xy - \frac{1}{4}x^4 +
y^3$ is a potential for $f$.

Since both components $p(x,y) = y - x^3$ and $q(x,y) = x + 3y^2$ of
$\mathbf{f}$ are continuously differentiable inside $C$, the existence of a
potential $F$ for $\mathbf{f}$ implies that $\mathbf{f}$ is a conservative
vector field, and thus $\int_C \! \mathbf{f} \cdot d\mathbf{r} \, = 0$.

\item

We again look for a potential $F$ for $\mathbf{f}$ by integrating its first and
second component with respect to $x$ and $y$, respectively:

\begin{equation*}F(x,y) = \frac{x^2}{2y} + g(y) = \frac{5}{2}xy^2 + h(x). \end{equation*}

However, there are no functions $g$ and $h$ that we can choose to satisfy this
equation.  Therefore no potential exists for $\mathbf{f}$, so it is not a
conservative vector field.

We proceed to evaluate the desired line integral:

\begin{align*}
 \int_C \! \mathbf{f} \cdot d\mathbf{r} \,
 &= \int_{0}^{2pi} \! \mathbf{f}(\cos(t), \sin(t)) \cdot \begin{bmatrix} -\sin(t) \\ \cos(t) \end{bmatrix} dt \, \\
 &= \int_{0}^{2pi} \! \begin{bmatrix} \dfrac{\cos(t)}{\sin(t)} \\ 5\cos(t)\sin(t) \end{bmatrix} \cdot \begin{bmatrix} -\sin(t) \\ \cos(t) \end{bmatrix} dt \, \\
 &= \int_{0}^{2pi} \! (-\cos(t) + 5\cos^2(t)\sin(t)) dt \, \\
 &= (\left. -\sin(t) - \frac{5}{3} \cos^3(t)) \right|_{0}^{2\pi} \\
 &= (0 - \frac{5}{3}) - (0 - \frac{5}{3}) \\
 &= 0
\end{align*}

\item
Once more we look for a potential $F$ for $\mathbf{f}$ by integrating its
components:

\begin{equation*} F(x,y) = \dfrac{1}{x^2+y^2} + g(y) = \dfrac{1}{x^2+y^2} + h(x).\end{equation*}

We can satisfy this equation by setting $g(y) = h(x) = 0$, so that $\nabla F =
\mathbf{f}$, and $\mathbf{f}$ is a conservative vector field.  However, since
neither component of $\mathbf{f}$ is defined at $(0,0)$, neither is
continuously differentiable on the entire region bounded by $C$, so it is not
guaranteed that $\int_C \! \mathbf{f} \cdot d\mathbf{r} \, = 0$.  We evaluate
the integral by hand:

\begin{align*}
 \int_C \! \mathbf{f} \cdot d\mathbf{r} \,
 &= \int_{0}^{2pi} \! \mathbf{f}(\cos(t), \sin(t)) \cdot \begin{bmatrix} -\sin(t) \\ \cos(t) \end{bmatrix} dt \, \\
 &= \int_{0}^{2pi} \! \begin{bmatrix} \dfrac{2\cos(t)}{(\cos^2(t) + \sin^2(t))^2} \\ \dfrac{2\sin(t)}{(\cos^2(t) + \sin^2(t))^2} \end{bmatrix} \cdot \begin{bmatrix} -\sin(t) \\ \cos(t) \end{bmatrix} dt \, \\
 &= \int_{0}^{2pi} \! \dfrac{-2\cos(t)\sin(t)}{(\cos^2(t) + \sin^2(t))^2} + \dfrac{2\sin(t)\cos(t)}{(\cos^2(t) + \sin^2(t))^2} dt \, \\
 &= \int_{0}^{2pi} \! 0 dt \, \\
 &= 0
\end{align*}

\EEN

\item
\BEN
\item Using Green's Theorem requires computing
\begin{align*}
 \frac{\partial Q}{\partial x} &= 4y \\
 \frac{\partial P}{\partial y} &= 2xy^3
\end{align*}
so that, when $A$ is the region contained by the curve $C$,
\begin{align*}
 \int_C \! (\frac{1}{2}xy^4)dx + (4xy)dy \,
 &= \iint\limits_A (4y - 2xy^3) dA \\
 &= \int_{-1}^{1} \int_{-\sqrt{4-4x^2}}^{\sqrt{4-4x^2}} \! (4y - 2xy^3) dy dx \, \\
 &= \int_{-1}^{1} (\left. 2y^2 - \frac{1}{2}xy^4) \right|_{-\sqrt{4-4x^2}}^{\sqrt{4-4x^2}} dx \, \\
 &= \int_{-1}^{1} (2(4-4x^2) - \frac{1}{2}x(4-4x^2)^2) - (2(4-4x^2) - \frac{1}{2}x(4-4x^2)^2) dx \, \\
 &= \int_{-1}^{1} 0 dx \, \\
 &= 0
\end{align*}

\item Again we compute
\begin{align*}
 \frac{\partial Q}{\partial x} &= e^{xy} + xye^{xy} + 3 \\
 \frac{\partial P}{\partial y} &= e^{xy} + xye^{xy} - 7
\end{align*}
so that, when $A$ is the region contained by the curve $C$,
\begin{align*}
 \int_C \! (ye^{xy} - 7y)dx + (xe^{xy} + 3x)dy \,
 &= \iint\limits_A 10 dA \\
 &= 10 \iint\limits_A dA \\
 &= 10 \text{(area of ellipse of minor radius 1 and major radius 2)} \\
 &= 10 (\pi * 1 * 2) \\
 &= 20\pi.
\end{align*}
\EEN

\item

Green's theorem tells us that, for smooth vector fields $\mathbf{f} =
P(x,y)\mathbf{i} + Q(x,y)\mathbf{j}$,
\begin{equation*}
 \int_C \! P(x,y) dx + Q(x,y) dy =
 \iint\limits_A \frac{\partial Q}{\partial x} - \frac{\partial P}{\partial y} dA.
\end{equation*}

To compute the mass of the annulus from the density function, we want to
evaluate the integral \[\iint\limits_{A_1} \! \dfrac{4x^2 + 3y^2}{x^2 + y^2}
dA_1 \, - \iint\limits_{A_2} \! \dfrac{4x^2 + 3y^2}{x^2 + y^2} dA_2 \,,\] where
$A_1$ and $A_2$ are circles of radius 5 and 2, respectively, centered at the
origin.

If we define $P(x,y) = -4x\arctan(y/x)$ and $Q(x,y) = 3y\arctan(x/y)$, then by
Green's Theorem, the mass is equal to the line integral \[\int_{C_1} \!
-4x\arctan(y/x) dx + 3y\arctan(x/y) dy \, - \int_{C_2} \! -4x\arctan(y/x) dx +
3y\arctan(x/y) dy \,.\]


%~~~~~~~~~~~~~~~~~~~~~~~~~~~~~~~~~~~~~~~~~~~~~~~~~~~~~~~~~~~~~~~~~~~~~~~~~~~~~
\EEN
