\documentclass{article}
\usepackage[utf8]{inputenc}
\usepackage{amsmath}
\usepackage{amssymb} % for \mathbb
\usepackage{graphicx} % for figures
\usepackage{color}
\usepackage[usenames,dvipsnames]{xcolor}
\usepackage{hyperref} % for hyperlinks
\usepackage{float} % for figures

\usepackage{fancyheadings}
\pagestyle{fancy}

%\rfoot[\thesection]{}
% ~ ~ ~ ~ ~ ~ ~ ~ ~ ~ ~ ~ ~ ~ ~ ~ ~ ~ ~ ~ ~ ~ ~ ~ ~ ~ ~ ~ ~ ~ ~ ~ ~ ~ ~ ~ ~ ~ ~ ~ ~ ~ ~ ~ ~ ~ ~ ~
% ENUMERATION

\usepackage{enumitem}   % so that question numbers can be formatted 
\setenumerate[1]{label=\thesubsection.\arabic*.} % enumerate environment: add section numbers to items
% ~ ~ ~ ~ ~ ~ ~ ~ ~ ~ ~ ~ ~ ~ ~ ~ ~ ~ ~ ~ ~ ~ ~ ~ ~ ~ ~ ~ ~ ~ ~ ~ ~ ~ ~ ~ ~ ~ ~ ~ ~ ~ ~ ~ ~ ~ ~ ~
% MARGINS
\usepackage{anysize}
\marginsize{2.5cm}{2.5cm}{1cm}{1cm}
% ~ ~ ~ ~ ~ ~ ~ ~ ~ ~ ~ ~ ~ ~ ~ ~ ~ ~ ~ ~ ~ ~ ~ ~ ~ ~ ~ ~ ~ ~ ~ ~ ~ ~ ~ ~ ~ ~ ~ ~ ~ ~ ~ ~ ~ ~ ~ ~
% PAGE NUMBERING
\pagenumbering{arabic}
% ~ ~ ~ ~ ~ ~ ~ ~ ~ ~ ~ ~ ~ ~ ~ ~ ~ ~ ~ ~ ~ ~ ~ ~ ~ ~ ~ ~ ~ ~ ~ ~ ~ ~ ~ ~ ~ ~ ~ ~ ~ ~ ~ ~ ~ ~ ~ ~
% Custom Commands
\newcommand{\Emph}[1]{\textbf{#1}} % Emphasize
\newcommand{\R}{\mathbb{R}} 
\newcommand{\BM}{\begin{bmatrix}} % Begin Matrix
\newcommand{\EM}{\end{bmatrix}} % End Matrix
\newcommand{\BEN}{\begin{enumerate}[leftmargin=1.1cm]}% Begin ENumerate
\newcommand{\EEN}{\end{enumerate}} % End ENumerate
\newcommand{\MB}{\mathbf} % Math Bold

\newcommand{\px}{\frac{\partial}{\partial x}} % Partial wrt x
\newcommand{\py}{\frac{\partial}{\partial y}} % Partial wrt y
\newcommand{\pxu}{\frac{\partial x}{\partial u}} % x wrt u
\newcommand{\pxv}{\frac{\partial x}{\partial v}} % x wrt v
\newcommand{\pyu}{\frac{\partial y}{\partial u}} % y wrt u
\newcommand{\pyv}{\frac{\partial y}{\partial v}} % y wrt v

\newcommand{\VCT}{\textit{Vector Calculus} by Michael Corral} % Vector Calculus Textbook
\newcommand{\CAT}{\textit{College Algebra} by Carl Stitz and Jeff Zeager} % College Algebra Textbook
\newcommand{\From}{The following questions are related to } % Questions ....
% ~ ~ ~ ~ ~ ~ ~ ~ ~ ~ ~ ~ ~ ~ ~ ~ ~ ~ ~ ~ ~ ~ ~ ~ ~ ~ ~ ~ ~ ~ ~ ~ ~ ~ ~ ~ ~ ~ ~ ~ ~ ~ ~ ~ ~ ~ ~ ~
% ONLY USED FOR EDITING
\newcommand{\rednote}[1]{{\color{red}\textit{#1}}} % Shortcut for red italics
\newcommand{\purple}[1]{{\color{Violet}\textit{#1}}} % Shortcut for red italics 
% ~ ~ ~ ~ ~ ~ ~ ~ ~ ~ ~ ~ ~ ~ ~ ~ ~ ~ ~ ~ ~ ~ ~ ~ ~ ~ ~ ~ ~ ~ ~ ~ ~ ~ ~ ~ ~ ~ ~ ~ ~ ~ ~ ~ ~ ~ ~ ~
% AUGMENTED MATRIX MACRO
% thanks to http://tex.stackexchange.com/questions/2233/whats-the-best-way-make-an-augmented-coefficient-matrix
\newenvironment{amatrix}[1]{%
  \left[\begin{array}{@{}*{#1}{c}|c@{}}
}{%
  \end{array}\right]
}



\title{Assignment 3}
\date{}
% ~ ~ ~ ~ ~ ~ ~ ~ ~ ~ ~ ~ ~ ~ ~ ~ ~ ~ ~ ~ ~ ~ ~ ~ ~ ~ ~ ~ ~ ~ ~ ~ ~ ~ ~ ~ ~ ~ ~ ~ ~ ~ ~ ~ ~ ~ ~ ~
\begin{document}
\begin{center}
\textsc{\LARGE Written Assignment 3}\\[0.5cm]
\end{center}

%\section*{Instructions}

\section*{Questions}

\begin{enumerate}
% -  -  -  -  -  -  -  -  -  -  -  -  -  -  -  -  -  -  -  -  -  -  -  -  -  -  -  -  -  -  
\item
% QUESTION 1, EASY
% GPS: MMCA2a

Find the values of $x$ that satisfy the equations

\begin{enumerate}
\item \(
 \begin{vmatrix}
  3 &  2 &  0 \\
  1 &  x &  0 \\
  7 & -3 &  4
 \end{vmatrix} = 4.
\)
\item \(
 \begin{vmatrix}
  x & 1 \\
  4 & 4x
 \end{vmatrix} = \begin{vmatrix}
  -x & -2 \\
  2 & 2x + 8
 \end{vmatrix}
\)
\end{enumerate}

% -  -  -  -  -  -  -  -  -  -  -  -  -  -  -  -  -  -  -  -  -  -  -  -  -  -  -  -  -  -  
\item
% QUESTION 2, EASY
% GPS: MMCA2b

Solve the matrix equation $Ax = b$ for $x$ when $A$ and $b$ are the 
following:

\begin{enumerate}
\item \(
 A = \begin{bmatrix}
  1 & 2 & 0 & 1 \\
  3 & -1 & 1 & 2 \\
  0 & 1 & 0 & 2 \\
  4 & 0 & 3 & -1
 \end{bmatrix},
 b = \begin{bmatrix} 1 \\ 5 \\ 0 \\ 10 \end{bmatrix}
\)
\item \(
 A = \begin{bmatrix}
  -3 &  1 & 2 &  4 \\
   2 & -1 & 2 &  3 \\
   1 &  0 & 4 & -2
 \end{bmatrix},
 b = \begin{bmatrix} 1 \\ 1 \\ 1 \end{bmatrix}
\)
\end{enumerate}

% -  -  -  -  -  -  -  -  -  -  -  -  -  -  -  -  -  -  -  -  -  -  -  -  -  -  -  -  -  -  
\item 
% QUESTION 3, MEDIUM
%GPS: MMCA2b 

Consider the matrix \( A = \begin{bmatrix}
 1 & 2 & -1 \\
 2 & 5 & 0 \\
 -3 & -5 & 4
\end{bmatrix} \) and the vector \(b =
\begin{bmatrix} 1 \\ 2 \\ -1 \end{bmatrix} \).
\begin{enumerate}
\item Compute $det(A)$.
\item Compute $A^{-1}$.
\item Compute the Cramer's Rule matrices $A_1, A_2, A_3$ and their 
determinants with respect to the linear system $Ax = b$.
\item Solve the linear system $Ax = b$.
\end{enumerate}

% -  -  -  -  -  -  -  -  -  -  -  -  -  -  -  -  -  -  -  -  -  -  -  -  -  -  -  -  -  -  
\item 
% QUESTION 4, HARD
%GPS: MMCA2b

Consider the system of simultaneous linear equations
\begin{align*}
  x + 2y -  z &= 2 \\
 2x + ay - 2z &= b \\
 3x + 2y      &= 1
\end{align*}
where $x, y, z$ are unknown.

\begin{enumerate}
\item Find all values of $a$ and $b$ such that the above system has
\begin{enumerate}
\item \label{q:one_solution} exactly one solution;
\item no solutions;
\item \label{q:many_solutions} infinitely many solutions.
\end{enumerate}
\item For those values of $a$ and $b$ from~\ref{q:one_solution}, what is the
unique solution?
\item For those values of $a$ and $b$ from~\ref{q:many_solutions},
 parameterize the set of all solutions.
\end{enumerate}

% -  -  -  -  -  -  -  -  -  -  -  -  -  -  -  -  -  -  -  -  -  -  -  -  -  -  -  -  -  -  
\item 
% QUESTION 5, MEDIUM
%GPS: MMCA2c
Let $A, B, C$ be three square matrices with the same dimensions.
\begin{enumerate}
\item Prove that if $A$ and $B$ are similar and $B$ and $C$ are similar then 
$A$ and $C$ are similar.
\item Let $x \in R$.  Prove that $A$ and $B + xI$ are similar whenever
$A - xI$ and $B$ are similar.
\item Suppose the system $Ax = b$ has infinitely many solutions $x$ and that
$A$ and $B$ are similar.  How many solutions can $Bx = b$ have?
\end{enumerate}

% -  -  -  -  -  -  -  -  -  -  -  -  -  -  -  -  -  -  -  -  -  -  -  -  -  -  -  -  -  -  
\item 
% QUESTION 6, MEDIUM
%GPS: MMCA2b 

\begin{enumerate}
\item Prove that, for any vectors $u, v$ and any orthogonal matrix $X$, $u 
\cdot v = Xu \cdot Xv$
\item Let $u$ and $v$ be two distinct vectors in $R^{2}$.  Prove that, when 
$X$ is an orthogonal matrix, the area of the parallelogram determined by $u$ 
and $v$ is equal to the area of the parallelogram determined by $Xu$ and 
$Xv$.
\end{enumerate}

% -  -  -  -  -  -  -  -  -  -  -  -  -  -  -  -  -  -  -  -  -  -  -  -  -  -  -  -  -  -  
\item 
% QUESTION 6, MEDIUM
%GPS: MMCA2b 
\textbf{Application to Mechanical Engineering}\\
In many areas of engineering, experimental data is collected that must be analyzed and interpreted. Suppose for example we have a set of measurements that represent -------. 


We might 
An interpreting polynomial is one that passes through every point in the collected data. 
\begin{enumerate}
\item Prove that, for any vectors $u, v$ and any orthogonal matrix $X$, $u 
\cdot v = Xu \cdot Xv$
\item Let $u$ and $v$ be two distinct vectors in $R^{2}$.  Prove that, when 
$X$ is an orthogonal matrix, the area of the parallelogram determined by $u$ 
and $v$ is equal to the area of the parallelogram determined by $Xu$ and 
$Xv$.
\end{enumerate}


\end{enumerate}


% ~ ~ ~ ~ ~ ~ ~ ~ ~ ~ ~ ~ ~ ~ ~ ~ ~ ~ ~ ~ ~ ~ ~ ~ ~ ~ ~ ~ ~ ~ ~ ~ ~ ~ ~ ~ ~ ~ ~ ~ ~ ~ ~ ~ ~ ~ ~ ~
% ~ ~ ~ ~ ~ ~ ~ ~ ~ ~ ~ ~ ~ ~ ~ ~ ~ ~ ~ ~ ~ ~ ~ ~ ~ ~ ~ ~ ~ ~ ~ ~ ~ ~ ~ ~ ~ ~ ~ ~ ~ ~ ~ ~ ~ ~ ~ ~
% SOLUTIONS
\newpage
\section*{Solutions}

\subsection*{Question 1}

\subsubsection*{(a)}
We can simplify the left-hand side of the equation by computing the
determinant:
\begin{align*}
  \begin{vmatrix}
   3 &  2 &  0 \\
   1 &  x &  0 \\
   7 & -3 &  4 \\
  \end{vmatrix}
 &=
  3 \begin{vmatrix} x & 0 \\ -3 & 4 \end{vmatrix} -
  2 \begin{vmatrix} 1 & 0 \\ 7 & 4 \end{vmatrix} +
  0 \begin{vmatrix} 1 & x \\ 7 & -3 \end{vmatrix} \\
 &=
  3(4x - 0) - 2(4 - 0) + 0(-3 -7x) \\
 &=
  12x - 8
\end{align*}

Substituting this result yields the equation $12x - 8 = 4$, from which the 
solution $x = 1$ is easily recovered.

\subsubsection*{(b)}
Expand both determinants and rearrange the equation:
\begin{align*}
 (x)(4x) - (1)(4) &= (-x)(2x+8) - (-2)(2) \\
 4x^2 - 4 &= -2x^2 - 8x + 4 \\
 6x^2 + 8x - 8 &= 0 \\
 (6x - 4)(x + 2) &= 0
\end{align*}
Therefore the solutions are $x = 2/3$ and $x = -2$.

\subsection*{Question 2}
Use row-reduction operations on the augmented matrices.

\subsubsection*{(a)}

\begin{align}
 \begin{amatrix}{4}
  1 & 2 & 0 & 1 & 1 \\
  3 & -1 & 1 & 2 & 5 \\
  0 & 1 & 0 & 2 & 0 \\
  4 & 0 & 3 & -1 & 10
 \end{amatrix}&
 \begin{array}{l}
  R_2 \leftarrow R_2 - 3R_1 \\ R_4 \leftarrow R_4 - 4R_1 
 \end{array}
 \nonumber \\
 \begin{amatrix}{4}
  1 & 2 & 0 & 1 & 1 \\
  0 & -7 & 1 & -1 & 2 \\
  0 & 1 & 0 & 2 & 0 \\
  0 & -8 & 3 & -5 & 6
 \end{amatrix}&
 \begin{array}{l}
  R_2 \leftarrow R_2 + 7R_3 \\ R_4 \leftarrow R_4 + 8R_3
 \end{array}
 \nonumber \\
 \begin{amatrix}{4}
  1 & 2 & 0 & 1 & 1 \\
  0 & 0 & 1 & 13 & 2 \\
  0 & 1 & 0 & 2 & 0 \\
  0 & 0 & 3 & 13 & 6
 \end{amatrix}&
 \begin{array}{l}
  R_4 \leftarrow R_4 - 3R_2
 \end{array}
 \nonumber \\
 \begin{amatrix}{4}
  1 & 2 & 0 & 1 & 1 \\
  0 & 0 & 1 & 13 & 2 \\
  0 & 1 & 0 & 2 & 0 \\
  0 & 0 & 0 & -26 & 0
 \end{amatrix}
 \nonumber
\end{align}

This is equivalent to the system of equations

\begin{align}
 x + 2y + w &= 1 \nonumber \\
 z + 13w &= 2 \nonumber \\
 y + 2w &= 0 \label{q2_a:a} \\
 -26w &= 0 \label{q2_a:b}
\end{align}

Equations (\ref{q2_a:a}) and (\ref{q2_a:b}) show that $w = 0$ and
$y = 0$, and substitution yields $z = 2$ and $x = 1$.

\subsubsection*{(b)}

\begin{align}
 \begin{amatrix}{4}
  -3 &  1 & 2 &  4 & 1 \\
   2 & -1 & 2 &  3 & 1 \\
   1 &  0 & 4 & -2 & 1
 \end{amatrix}&
 \begin{array}{l}
  R_1 \leftarrow R_1 + 3R_3 \\ R_2 \leftarrow R_2 - 2R_3
 \end{array}
 \nonumber \\
 \begin{amatrix}{4}
  0 &  1 & 14 & -2 &  4 \\
  0 & -1 & -6 &  7 & -1 \\
  1 &  0 &  4 & -2 &  1
 \end{amatrix}&
 \begin{array}{l}
  R_2 \leftarrow R_2 + R_1
 \end{array}
 \nonumber \\
 \begin{amatrix}{4}
  0 & 1 & 14 & -2 & 4 \\
  0 & 0 &  8 &  5 & 3 \\
  1 & 0 &  4 & -2 & 1
 \end{amatrix}&
 \nonumber
\end{align}

This is equivalent to the system of equations

\begin{align}
 y + 14z -2w &= 4 \label{q2_b:a} \\
 8z + 5w &= 3 \label{q2_b:b} \\
 x + 4z -2w &= 1
\end{align}

Since there are fewer equations than unknowns we will parameterize the 
solution set by setting $w = t$, where t is any real number.  From equation 
(\ref{q2_b:b}) we get $z = \frac{3}{8} - \frac{5}{8}t$.  Substituting $z$ in 
equations (\ref{q2_b:a}) and (\ref{q2:b:c}) shows that $x = -\frac{1}{2} + 
\frac{9}{2}t$ and $y = -\frac{5}{4} + \frac{43}{4}t$.

\subsection*{Question 3}

\subsubsection*{(a)}

Expand the 3x3 determinant into 2x2 determinants:
\begin{align*}
 \begin{vmatrix}
  1 &  2 & -1 \\
  2 &  5 &  0 \\
 -3 & -5 &  4
 \end{vmatrix} &=
 1 \begin{vmatrix} 5 & 0 \\ -5 & 4 \end{vmatrix}
 -2 \begin{vmatrix} 2 & 0 \\ -3 & 4 \end{vmatrix}
 -1 \begin{vmatrix} 2 & 5 \\ -3 & -5 \end{vmatrix} \\
 &= (20) -2 (8) -(-10 + 15) \\
 &= -1
\end{align*}

\subsubsection*{(b)}

Form the augmented matrix and apply row-reduction operations:

\begin{align}
 \begin{bmatrix}
  1 & 2 & -1 & 1 & 0 & 0 \\
  2 & 5 & 0 & 0 & 1 & 0 \\
  -3 & -5 & 4 & 0 & 0 & 1
 \end{bmatrix}&
 \begin{array}{l}
  R_2 \leftarrow R_2 - 2R_1 \\
  R_3 \leftarrow R_3 + 3R_1
 \end{array}
 \nonumber \\
 \begin{bmatrix}
  1 & 2 & -1 & 1 & 0 & 0 \\
  0 & 1 & 2 & -2 & 1 & 0 \\
  0 & 1 & 1 & 3 & 0 & 1
 \end{bmatrix}&
 \begin{array}{l}
  R_3 \leftarrow R_3 - R_2
 \end{array}
 \nonumber \\
 \begin{bmatrix}
  1 & 2 & -1 & 1 & 0 & 0 \\
  0 & 1 & 2 & -2 & 1 & 0 \\
  0 & 0 & -1 & 5 & -1 & 1
 \end{bmatrix}&
 \begin{array}{l}
  R_2 \leftarrow R_2 + 2R_3 \\
  R_3 \leftarrow -1 * R_3
 \end{array}
 \nonumber \\
 \begin{bmatrix}
  1 & 2 & -1 & 1 & 0 & 0 \\
  0 & 1 & 0 & 8 & -1 & 2 \\
  0 & 0 & 1 & -5 & 1 & -1
 \end{bmatrix}&
 \begin{array}{l}
  R_1 \leftarrow R_1 - 2R_2 + R_3
 \end{array}
 \nonumber \\
 \begin{bmatrix}
  1 & 0 & 0 & -20 & 3 & -5 \\
  0 & 1 & 0 & 8 & -1 & 2 \\
  0 & 0 & 1 & -5 & 1 & -1
 \end{bmatrix}
 \nonumber
\end{align}

So the inverse matrix is
\(\begin{bmatrix} -20 & 3 & -5 \\ 8 & -1 & 2 \\ -5 & 1 & -1 \end{bmatrix}\).

\subsubsection*{(c)}

The matrices $A_i$ are found by replacing the $i^{th}$ column of $A$ with
the vector $b$.

\begin{align*}
 A_1 &=
  \begin{bmatrix}
   1 & 2 & -1 \\
   2 & 5 & 0 \\
   -1 & -5 & 4
  \end{bmatrix}, det(A_1) = 20 - 16 + 5 = 9 \\
 A_2 &=
  \begin{bmatrix}
   1 & 1 & -1 \\
   2 & 2 & 0 \\
   -3 & -1 & 4
  \end{bmatrix}, \det(A_2) = 8 - 8 - 4 = -4 \\
 A_3 &=
  \begin{bmatrix}
   1 & 2 & 1 \\
   2 & 5 & 2 \\
   -3 & -5 & -1
  \end{bmatrix}, \det(A_3) = 5 - 8 + 5 = 2
\end{align*}

\subsubsection*{(d)}

Using the previous results, we can solve the system using Cramer's Rule:

\[
 x =
 \begin{bmatrix} 9/-1 \\ -4/-1 \\ 2/-1 \end{bmatrix} =
 \begin{bmatrix} -9 \\ 4 \\ -2 \end{bmatrix}
\]

\subsection*{Question 4}

We first convert the system of equations into an equivalent augmented-matrix
form:
\[
 \begin{amatrix}{3}
  1 & 2 & -1 & 2 \\
  2 & a & -2 & b \\
  3 & 2 & 0 & 1
 \end{amatrix}
\]
Then apply a sequence of row-reduction operations:
\begin{align}
 \begin{amatrix}{3}
  1 &   2 & -1 &   2 \\
  0 & a-4 &  0 & b-4 \\
  0 &  -4 &  3 &  -5
 \end{amatrix} & \begin{array}{l} R_2 - 2R_1 \\ R_3 - 3R_1 \end{array} 
 \nonumber \\
 \begin{amatrix}{3}
  1 &  2 & -1 &   2 \\
  0 &  a & -3 & b+1 \\
  0 & -4 &  3 &  -5
 \end{amatrix} & \begin{array}{l} R_2 - R_3 \end{array}
 \nonumber \\
 \begin{amatrix}{3}
  1 & 2 & -1 &   2 \\
  0 & a & -3 & b+1 \\
  0 & 4 & -3 &   5
 \end{amatrix} & \begin{array}{l} -1 \cdot R_3 \end{array}
 \nonumber \\
 \begin{amatrix}{3}
  1 & 2 & -1 &   2 \\
  0 & 4 & -3 &   5 \\
  0 & a & -3 & b+1
 \end{amatrix} & \begin{array}{l} R_2 \leftrightarrow R_3 \end{array} 
 \label{eq:reduced}
\end{align}

Suppose that $a = 4$.  Then (\ref{eq:reduced}) becomes
\begin{align}
 \begin{amatrix}{3}
  1 & 2 & -1 &   2 \\
  0 & 4 & -3 &   5 \\
  0 & 4 & -3 & b+1
 \end{amatrix} & \begin{array}{l} R_3 - R_2 \end{array}
 \nonumber \\
 \begin{amatrix}{3}
  1 & 2 & -1 &   2 \\
  0 & 4 & -3 &   5 \\
  0 & 0 &  0 & b-4
 \end{amatrix}
 \label{eq:substitute_a}
\end{align}

If $b \neq 4$, then the system has no solutions, since the last row is
equivalent to the equation $0 = b-4$ where $b-4 \neq 0$.  Conversely, if
$b = 4$, then (\ref{eq:substitute_a}) is equivalent to the system of
equations
\begin{align*}
 x + 2y -z &= 2 \\
 4y - 3z &= 5
\end{align*}
There are fewer equations than unknowns, so we have to parameterize the
solution set; let $z = t$.  Solving these two equations yields the solution
\begin{equation}\begin{aligned}
 x &= -\frac{1}{2} - \frac{1}{2}t \\
 y &= \frac{5}{4} + \frac{3}{4}t \\
 z &= t
\end{aligned}\label{q4:inf}\end{equation}

Now suppose that $a = 0$.  Then the matrix (\ref{eq:reduced}) is equivalent
to the system

\begin{align*}
 x + 2y -z &= 2 \\
 4y -3z &= 5 \\
 -3z &= b+1
\end{align*}

Using back-substitution we find that

\begin{align*}
 x &= -\frac{1}{3} - \frac{1}{6}b \\
 y &= 1 - \frac{1}{4}b \\
 z &= \frac{-1}{3} - \frac{1}{3}b
\end{align*}

Now suppose that $a \neq 4$ and $a \neq 0$.  Then we can continue row-
reducing from (\ref{eq:reduced}):

\begin{align}
 \begin{amatrix}{3}
  1 & 2 & -1 &   2 \\
  0 & 4 & -3 &   5 \\
  0 & a & -3 & b+1
 \end{amatrix} &
 \begin{array}{l}
  R_3 \leftarrow R_3 - \dfrac{a}{4}R_2
 \end{array} 
 \nonumber \\
 \begin{amatrix}{3}
  1 & 2 & -1 & 2 \\
  0 & 4 & -3 & 5 \\
  0 & 0 & -3 + \frac{3}{4}a & 1 + b - \frac{5}{4}a
 \end{amatrix}
 \nonumber
\end{align}

This is equivalent to the system

\begin{align*}
 x + 2y - z &= 2 \\
 4y -3z &= 5 \\
 (-3 + \frac{3}{4}a)z &= 1 + b - \frac{5}{4}a
\end{align*}

Let $c = \dfrac{1 + b - \frac{5}{4}a}{-3 + \frac{3}{4}a}$.  Using back-substitution we find the solution

\begin{equation}\begin{aligned}
 x &= -\frac{1}{2} - \frac{1}{2}c \\
 y &= \frac{5}{4} + \frac{3}{4}c \\
 z &= c
\end{aligned}\label{q4:one}\end{equation}

Thus we have shown that:
\begin{enumerate}
\item When $a = 4$ and $b = 4$ there are infinitely many solutions, given
by the parameterization (\ref{q4:inf}).
\item When $a = 4$ and $b \neq 4$ there are no solutions.
\item When $a \neq 4$ and $a \neq 0$ there is one solution, given by 
(\ref{q4:one}).
\end{enumerate}

\subsection*{Question 5}

\subsubsection*{(a)}
Since $A$ and $B$ are similar and $B$ and $C$ are similar there exist
matrices $S$ and $T$ such that $A = S^{-1}BS$ and $B = T^{-1}CT$; by
substitution we see that $A = S^{-1}(T^{-1}CT)S = (TS)^{-1}C(TS)$.  If
we set $U = TS$, then $A = U^{-1}CU$, so that A and C are similar.

\subsubsection*{(b)}
Since $A - xI$ and $B$ are similar there exists a square invertible matrix
$C$ such that $C^{-1}(A - xI)C = B$.  By expanding the left-hand side of the
equation we get $C^{-1}AC - C^{-1}xIC = B$.  However, since $x$ is a scalar,
$C^{-1}xIC = xC^{-1}IC = xC^{-1}C = xI$, so that $C^{-1}AC - xI = B$.  By
rearranging we see that $B + xI = C^{-1}AC$, so that $A$ and $B + xI$ are
similar.

\subsubsection*{(c)}
Since $A$ and $B$ are similar, they have the same determinant.  The system
$Ax = b$ has infinitely many solutions, which, since $A$ is square, implies 
that $A$ is not invertible, so that $det(A) = det(B) = 0$.  Therefore the 
system $Bx = b$ is either unsolvable or has infinitely many solutions.

\subsection*{Question 6}

\subsubsection*{(a)}
Since $u \cdot v = u^{T}v$ for any vectors $u, v$ and $X^{T}X = I$, $Xu
\cdot Xv = (Xu)^{T}Xv = u^{T}X^{T}Xv = u^{T}v = u \cdot v$.

\subsubsection*{(b)}
The area $area_{u,v}$ of the parallelogram determined by $u$ and $v$ is given by the magnitude of the cross product $u \times v$:
\[area_{u,v} = ||u||||v|| sin(\theta_{u,v})\]
where $\theta_{u,v}$ is the angle between the vectors $u, v$.  Similarly,
\[area_{Xu,Xv}\ = ||Xu||||Xv|| sin(\theta_{Xu,Xv})\]
Applying part (a) we see that $||Xu|| = \sqrt{Xu \cdot Xu} = \sqrt{u \cdot u} = ||u||$ and $||Xv|| = ||v||$.  Additionally,
\begin{align*}
 sin(\theta_{Xu,Xv}) &= \sqrt{1 - cos(\theta_{Xu,Xv})^{2}} \\
 & = \sqrt{1 - (\dfrac{Xu \cdot Xv}{||Xu||||Xv||})^{2}} \\
 &= \sqrt{1 - (\dfrac{u \cdot v}{||u||||v||})^{2}} \\
 &= \sqrt{1 - cos(\theta_{u,v})^{2}} \\
 &= sin(\theta_{u,v}).
\end{align*}

Thus $area_{Xu,Xv} = area_{u,v}$.

\end{document}
