\documentclass{article}
\usepackage[utf8]{inputenc}
\usepackage{amsmath}
\usepackage{amssymb} % for \mathbb
\usepackage{graphicx} % for figures
\usepackage{color}
\usepackage[usenames,dvipsnames]{xcolor}
\usepackage{hyperref} % for hyperlinks
\usepackage{float} % for figures

\usepackage{fancyheadings}
\pagestyle{fancy}

%\rfoot[\thesection]{}
% ~ ~ ~ ~ ~ ~ ~ ~ ~ ~ ~ ~ ~ ~ ~ ~ ~ ~ ~ ~ ~ ~ ~ ~ ~ ~ ~ ~ ~ ~ ~ ~ ~ ~ ~ ~ ~ ~ ~ ~ ~ ~ ~ ~ ~ ~ ~ ~
% ENUMERATION

\usepackage{enumitem}   % so that question numbers can be formatted 
\setenumerate[1]{label=\thesubsection.\arabic*.} % enumerate environment: add section numbers to items
% ~ ~ ~ ~ ~ ~ ~ ~ ~ ~ ~ ~ ~ ~ ~ ~ ~ ~ ~ ~ ~ ~ ~ ~ ~ ~ ~ ~ ~ ~ ~ ~ ~ ~ ~ ~ ~ ~ ~ ~ ~ ~ ~ ~ ~ ~ ~ ~
% MARGINS
\usepackage{anysize}
\marginsize{2.5cm}{2.5cm}{1cm}{1cm}
% ~ ~ ~ ~ ~ ~ ~ ~ ~ ~ ~ ~ ~ ~ ~ ~ ~ ~ ~ ~ ~ ~ ~ ~ ~ ~ ~ ~ ~ ~ ~ ~ ~ ~ ~ ~ ~ ~ ~ ~ ~ ~ ~ ~ ~ ~ ~ ~
% PAGE NUMBERING
\pagenumbering{arabic}
% ~ ~ ~ ~ ~ ~ ~ ~ ~ ~ ~ ~ ~ ~ ~ ~ ~ ~ ~ ~ ~ ~ ~ ~ ~ ~ ~ ~ ~ ~ ~ ~ ~ ~ ~ ~ ~ ~ ~ ~ ~ ~ ~ ~ ~ ~ ~ ~
% Custom Commands
\newcommand{\Emph}[1]{\textbf{#1}} % Emphasize
\newcommand{\R}{\mathbb{R}} 
\newcommand{\BM}{\begin{bmatrix}} % Begin Matrix
\newcommand{\EM}{\end{bmatrix}} % End Matrix
\newcommand{\BEN}{\begin{enumerate}[leftmargin=1.1cm]}% Begin ENumerate
\newcommand{\EEN}{\end{enumerate}} % End ENumerate
\newcommand{\MB}{\mathbf} % Math Bold

\newcommand{\px}{\frac{\partial}{\partial x}} % Partial wrt x
\newcommand{\py}{\frac{\partial}{\partial y}} % Partial wrt y
\newcommand{\pxu}{\frac{\partial x}{\partial u}} % x wrt u
\newcommand{\pxv}{\frac{\partial x}{\partial v}} % x wrt v
\newcommand{\pyu}{\frac{\partial y}{\partial u}} % y wrt u
\newcommand{\pyv}{\frac{\partial y}{\partial v}} % y wrt v

\newcommand{\VCT}{\textit{Vector Calculus} by Michael Corral} % Vector Calculus Textbook
\newcommand{\CAT}{\textit{College Algebra} by Carl Stitz and Jeff Zeager} % College Algebra Textbook
\newcommand{\From}{The following questions are related to } % Questions ....
% ~ ~ ~ ~ ~ ~ ~ ~ ~ ~ ~ ~ ~ ~ ~ ~ ~ ~ ~ ~ ~ ~ ~ ~ ~ ~ ~ ~ ~ ~ ~ ~ ~ ~ ~ ~ ~ ~ ~ ~ ~ ~ ~ ~ ~ ~ ~ ~
% ONLY USED FOR EDITING
\newcommand{\rednote}[1]{{\color{red}\textit{#1}}} % Shortcut for red italics
\newcommand{\purple}[1]{{\color{Violet}\textit{#1}}} % Shortcut for red italics 
% ~ ~ ~ ~ ~ ~ ~ ~ ~ ~ ~ ~ ~ ~ ~ ~ ~ ~ ~ ~ ~ ~ ~ ~ ~ ~ ~ ~ ~ ~ ~ ~ ~ ~ ~ ~ ~ ~ ~ ~ ~ ~ ~ ~ ~ ~ ~ ~
% AUGMENTED MATRIX MACRO
% thanks to http://tex.stackexchange.com/questions/2233/whats-the-best-way-make-an-augmented-coefficient-matrix
\newenvironment{amatrix}[1]{%
  \left[\begin{array}{@{}*{#1}{c}|c@{}}
}{%
  \end{array}\right]
}



\usepackage{multienum}
\usepackage{wrapfig}

\date{}
% ~ ~ ~ ~ ~ ~ ~ ~ ~ ~ ~ ~ ~ ~ ~ ~ ~ ~ ~ ~ ~ ~ ~ ~ ~ ~ ~ ~ ~ ~ ~ ~ ~ ~ ~ ~ ~ ~ ~ ~ ~ ~ ~ ~ ~ ~ ~ ~
\begin{document}
\begin{center}
\textsc{\LARGE Written Assignment 3}\\[0.5cm]
\end{center}
\From section 2.1 of \VCT.
% ~~~~~~~~~~~~~~~~~~~~~~~~~~~~~~~~~~~~~~~~~~~~~~~~~~~~~~~~~~~~~~~~~~~~~~~~~~~~~~~~~
\section*{Wolfram Alpha Syntax}
You may want to use Wolfram Alpha (wolframalpha.com) to check your answers. If you're not sure what syntax to use to compute square roots and evaluate multivariable limits with Wolfram Alpha, let's suppose that we want to determine the value of
\begin{align*} 
  f(x,y,z) = \frac{2 \sqrt{x}}{y^2 + \sqrt{z}}
\end{align*}
at the point (4,5,9). The syntax we could use to evaluate $f$ at the given point is
\begin{quote}
  \begin{verbatim}
     substitute x=4 and y=5 and z=9 into (2 * sqrt(x) / ( y^2 + sqrt(z) )
  \end{verbatim}
\end{quote}
Or we can simply use the syntax
\begin{quote}
  \begin{verbatim}
     (2 * sqrt(4) / ( 5^2 + sqrt(9) )
  \end{verbatim}
\end{quote}
\section*{Questions}

\begin{enumerate}
% ~~~~~~~~~~~~~~~~~~~~~~~~~~~~~~~~~~~~~~~~~~~~~~~~~~~~~~~~~
\item 
% QUESTION
% GPS: Not in GPS
For the following functions, 
\begin{itemize}
\item determine the domain of the function,
\item sketch the domain of the function in the $xy$ plane, and
\item determine the range of the function
\end{itemize}
\begin{enumerate}
  \item $f(x,y) = \frac{1}{\sqrt{(x-1)(y-1)}}$
  \item $g(x,y) = \frac{\sqrt{x+1}}{yx^2+xy^2}$
  \item $h(x,y) = \frac{xyz}{x^2+y^2-1}$
  \item $z(x,y) = \ln(x+2y)$
\end{enumerate}
% ~~~~~~~~~~~~~~~~~~~~~~~~~~~~~~~~~~~~~~~~~~~~~~~~~~~~~~~~~
\item 
% QUESTION
%GPS: MMCD1
Evaluate the following limits or show that they do not exist.  
\begin{enumerate}
\item $\displaystyle \lim_{(r,s)\rightarrow(0,2\pi) } \frac{3r^2+rs^3-3r^4\sin (s/4)}{r^2}$
\item $\displaystyle \lim_{(x,y)\rightarrow(0,0) } \frac{x-y}{x+y}$
\item $\displaystyle \lim_{(x,y,z)\rightarrow(1,1,1)} \big|3x - 2y - z \big|$
\item $\displaystyle \lim_{(x,y,z)\rightarrow(0,0,0) } \frac{x^2-y^2-z^2}{x^2+y^2+z^2}$
\item $\displaystyle \lim_{(x,y)\rightarrow(1,0) } \frac{\sqrt{2x+y}-\sqrt{2x-y}}{2y}$
\end{enumerate}
% ~~~~~~~~~~~~~~~~~~~~~~~~~~~~~~~~~~~~~~~~~~~~~~~~~~~~~~~~~
\item 
% QUESTION
%GPS: MMCD1
Consider the function
\begin{align*}
g(x,y,z) = \left\{
\begin{array}{rl}
1 & \text{if } (x,y,z) = (0,0,1) \\
x + y + 2z & \text{if } (x,y,z) \ne (0,0,1) 
\end{array} \right. .
\end{align*}
Find all points or regions where $g(x,y,z)$ has a discontinuity.
% ~~~~~~~~~~~~~~~~~~~~~~~~~~~~~~~~~~~~~~~~~~~~~~~~~~~~~~~~~
\item 
% QUESTION
%GPS: MMCD1
Evaluate the following limit, if it exists.
\begin{align*} 
\lim_{(x,y)\rightarrow(1,0) } \frac{x(x-1)^3 + y^2}{4(x-1)^2+9y^3}
\end{align*}
 \textit{Hint: approach the limit point along straight lines}.
% ~~~~~~~~~~~~~~~~~~~~~~~~~~~~~~~~~~~~~~~~~~~~~~~~~~~~~~~~~
\item 
% QUESTION
%GPS: MMCA3b
Sketch, by hand, the level curves of the following functions for the indicated values of $c$, if possible. You may of course want to check your answer using a calculator or with graphing software. 
\begin{enumerate}
\item $f(x,y) =\frac{\ln y}{x^2} $, $c=-2, - 1, 0 , 1 , 2$
\item $f(x,y) = \frac{x^2}{x^2+y^2}$, $c=0,\frac{1}{4},\frac{1}{2},1,2$
\end{enumerate}
% ~~~~~~~~~~~~~~~~~~~~~~~~~~~~~~~~~~~~~~~~~~~~~~~~~~~~~~~~~
\item 
% QUESTION
%GPS: MMCA3b
\begin{enumerate}
\item Find a function $f(x,y)$ whose family of level curves are straight lines that pass through the point $(2,1)$.
\item Find a function $f(x,y)$ whose family of level curves are circles with radius $\sqrt{e^c}$. 
\end{enumerate}

% ~~~~~~~~~~~~~~~~~~~~~~~~~~~~~~~~~~~~~~~~~~~~~~~~~~~~~~~~~
\item Let $m$ be any real number. 
\begin{enumerate}
\item Provide an example of a function, $f(x,y)$, with the following two properties:
\begin{itemize}
\item $f(x,y) \rightarrow m$ as $(x,y) \rightarrow (1,0)$ along any parabola, $y=m(x-1)^2$
\item $f(x,y)$ is not constant
\end{itemize}
\item Explain why the limit you provided in part (a) does not exist. 
\end{enumerate}
% ~~~~~~~~~~~~~~~~~~~~~~~~~~~~~~~~~~~~~~~~~~~~~~~~~~~~~~~~~
% TEMPERATURE DISTRIBUTION
\item \textbf{Application to Temperature Distributions}\\
Suppose that a metal object occupies a space in three dimensions, and that the temperature, $T(x,y)$ of the object at the point $(x,y)$ is inversely proportional to the distance between the point and the origin. 
\BEN
\item Write down an expression for $T$ as a function of $x$ and $y$.
\item Describe the level curves in words. Sketching them is not necessary, but the level curves are known as \Emph{isothermals}, and represent curves upon which the temperature is constant. 
\item Suppose the temperature of the object at the point (2,1) is $10^{\circ}$C. Find the temperature of the object at (1,3).
\EEN
% ~~~~~~~~~~~~~~~~~~~~~~~~~~~~~~~~~~~~~~~~~~~~~~~~~~~~~~~~~
% ELECTRICAL FIELD DISTRIBUTION
\item \textbf{Application to Electrical Potential Distributions}\\
The scalar function
\begin{align*}
V(x,y) = \frac{c}{\sqrt{r^2 - x^2 -y^2}}\ ,
\end{align*}
where $c$ and $r$ are positive constants, represents the electrical potential (in volts) at a point $(x,y)$ in the $xy-$plane. Describe, in words, the level curves $V(x,y)=K$ for constant $K\in\R$, and determine the values of $K$ for which the curves exist. Note that the level curves are known as \Emph{equipotential curves}, because all points on a given curve have the same potential. \textit{can we do something a little more interesting with this question? its somewhat similar to the previous one}

% ~~~~~~~~~~~~~~~~~~~~~~~~~~~~~~~~~~~~~~~~~~~~~~~~~~~~~~~~~
\item 
% BONUS QUESTION, LIMITS
\textbf{Bonus Problem}\\
\textit{This question goes beyond the requirements for this course. Before starting this problem, make sure that your instructor will give you marks for solving it.} \\
Using the definition of limit (the $\epsilon, \delta$ definition) for a function of two variables, prove that the limit
\begin{align*}
\lim_{(x,y)\rightarrow(0,0) } \frac{xy^2}{x^2+y^2}
\end{align*}
exists. \textit{Hint: start by evaluating the limit along the $x$-axis, or along other straight lines to determine what the limit could be equal to}.
% ~~~~~~~~~~~~~~~~~~~~~~~~~~~~~~~~~~~~~~~~~~~~~~~~~~~~~~~~~
\end{enumerate}
\noindent \textbf{References} \\
The temperature and electrical potential questions are based on a similar exercises that can be found in \textit{Calculus, One and Several Variables}, 10th Edition, by Salas, Hille, Etgen, Wiley, 2007.
%%%%%%%%%%%%%%%%%%%%%%%%%%%%%%%%%%%%%%
%%%%%%%%%%%%%%%%%%%%%%%%%%%%%%%%%%%%%%
%%%%%%%%%%%%%%%%%%%%%%%%%%%%%%%%%%%%%%

% SOLUTIONS
\newpage
\section*{Solutions}
\begin{enumerate}
% ~~~~~~~~~~~~~~~~~~~~~~~~~~~~~~~~~~~~~~~~~~~~~~~~~~~~~~~~~
%DOMAIN AND RANGE PROBLEMS
\item 
\begin{enumerate}
% -  -  -  -  -  -  -  -  -  -  -  -  -  -  -  -  -  -  -  -  -  -  -  -  -  -  -  -  -  -  
% PART A
\item The function is not defined when the denominator is zero, or when the argument of the square root is negative. Either $x$ and $y$ are greater than 1, or less than -1. The domain, $D$, is therefore
\begin{align*}D = \{(x,y)|x > 1, y > 1; \text{ and } x < -1, y < -1\}. \end{align*}
The function can't be zero, and cannot be negative, because the square root will always yield a positive number. The range is $f > 0$.  
\begin{figure}[!htbp]
  \begin{center}
    \includegraphics[width=0.4\textwidth]{WA021A.jpg}
  \end{center}
\end{figure}
% -  -  -  -  -  -  -  -  -  -  -  -  -  -  -  -  -  -  -  -  -  -  -  -  -  -  -  -  -  -  
% PART B
\item The function is not defined when the denominator is zero, so we require that $yx^2+xy^2\ne0$, or $xy(x+y)\ne0$. This implies that the function is not defined along the lines $x=0$, $y=0$, and $y=-x$. Moreover, the numerator has a square root. The argument must be non-negative, and so we also have the restriction that $y\ge-1$. The domain can therefore be expressed as $D=\{(x,y)| y\ge -1, x\ne0, y\ne0, y\ne-x\}$. The numerator can take on any non-negative value, and the denominator can take on any value, so the function $f$ can take on any value, so the range is $\mathbb{R}$.
\begin{figure}[!htbp]
  \begin{center}
    \includegraphics[width=0.4\textwidth]{WA021B.jpg}
  \end{center}
\end{figure}
% -  -  -  -  -  -  -  -  -  -  -  -  -  -  -  -  -  -  -  -  -  -  -  -  -  -  -  -  -  -  
% PART C
\item The function is not defined when the denominator is zero, so we require that $x^2+y^2\ne1$. The domain can therefore be expressed as $D=\{(x,y)| x^2+y^2 \ne 1 \}$. The numerator can take on any value, so the function $f$ can take on any value, so the range is $\mathbb{R}$.
\begin{figure}[!htbp]
  \begin{center}
    \includegraphics[width=0.4\textwidth]{WA021C.jpg}
  \end{center}
\end{figure}
% -  -  -  -  -  -  -  -  -  -  -  -  -  -  -  -  -  -  -  -  -  -  -  -  -  -  -  -  -  -  
% PART D
\newpage
\item For the domain, we require that 
\begin{align*}
x+2y &> 0  \text{, or } y > - x/2.
\end{align*}
The domain can therefore be expressed as $D=\{(x,y)|y > - x/2\}$. The function $f$ can take on any value, so the range is $\mathbb{R}$.
\begin{figure}[!htbp]
  \begin{center}
    \includegraphics[width=0.4\textwidth]{WA021D.jpg}
  \end{center}
\end{figure}
% -  -  -  -  -  -  -  -  -  -  -  -  -  -  -  -  -  -  -  -  -  -  -  -  -  -  -  -  -  -  
\end{enumerate}

% ~~~~~~~~~~~~~~~~~~~~~~~~~~~~~~~~~~~~~~~~~~~~~~~~~~~~~~~~~
\item
% LIMIT SOLUTIONS
% -  -  -  -  -  -  -  -  -  -  -  -  -  -  -  -  -  -  -  -  -  -  -  -  -  -  -  -  -  -  
\begin{enumerate}
\item We can simply evaluate the limit to obtain
\begin{align*}
\lim_{(r,s)\rightarrow(0,2\pi) } \frac{3r^2+rs^3-3\sin (s/4)}{r^2}
&= \lim_{(r,s)\rightarrow(0,2\pi) }3+\frac{s^3}{r}-3r^2\sin (s/4)
\end{align*}
Because of the $s^3/r$ term, this limit tends to infinity, and therefore does not exist.
% -  -  -  -  -  -  -  -  -  -  -  -  -  -  -  -  -  -  -  -  -  -  -  -  -  -  -  -  -  -  
\item Let $f(x,y) = (x-y)/(x+y)$. Along the $x$-axis, $f=f(x,0)$, so 
\begin{align*}
f(x,0) = \frac{x}{x} = 1, \quad x\ne 0.
\end{align*}
A similar calculation shows that along the $y$-axis, $f$ = -1, if $y\ne0$. If we approach the point $(0,0)$ along the $x$-axis and the $y$-axis, we find that 
\begin{align*}
\text{along the $x$-axis, } & f = f(x,0),\text{ and } f \rightarrow +1 \\
\text{along the $y$-axis, } & f = f(0,y),\text{ and } f \rightarrow -1 
\end{align*}
These limits are not equal, and so the limit does not exist. 
% -  -  -  -  -  -  -  -  -  -  -  -  -  -  -  -  -  -  -  -  -  -  -  -  -  -  -  -  -  -  
\item We can simply evaluate the limit to obtain 
\begin{align*}
  \lim_{(x,y,z)\rightarrow(1,1,1)} \big|3x - 2y - z \big| = 3 -2 -1 = 0
\end{align*}
% -  -  -  -  -  -  -  -  -  -  -  -  -  -  -  -  -  -  -  -  -  -  -  -  -  -  -  -  -  -  
\item  Let $f(x,y,z) = (x^2-y^2-z^2)/(x^2+y^2+z^2)$. Then, along the $x$-axis, $f=(x,0,0)$. As long as $x$ is not zero, $f$ is equal to 1. However, along the $y$-axis, $f=f(0,y,0)$. Provided that $y$ is not zero, $f$ is equal to -1. Therefore,
\begin{align*}
\text{along the $x$-axis, } & f \rightarrow +1 \\
\text{along the $y$-axis, } & f \rightarrow -1 
\end{align*}
These limits are not equal, and so the limit does not exist. 
% -  -  -  -  -  -  -  -  -  -  -  -  -  -  -  -  -  -  -  -  -  -  -  -  -  -  -  -  -  -  
\item  We can evaluate this limit by rationalizing the numerator.
\begin{align*}
  \lim_{(x,y)\rightarrow(1,0) } \frac{\sqrt{2x+y}-\sqrt{2x-y}}{2y} &= 
  \lim_{(x,y)\rightarrow(1,0) } \frac{\sqrt{2x+y}-\sqrt{2x-y}}{2y} \Bigg(\frac{\sqrt{2x+y}+\sqrt{2x-y}}{\sqrt{2x+y}+\sqrt{2x-y}} \Bigg) \\
  &=\lim_{(x,y)\rightarrow(1,0) } \frac{(2x+y)-(2x-y)}{2y \Big(\sqrt{2x+y}+\sqrt{2x-y} \Big)}  \\
  &=\lim_{(x,y)\rightarrow(1,0) } \frac{2y}{2y \Big(\sqrt{2x+y}+\sqrt{2x-y} \Big)}  \\
  &=\lim_{(x,y)\rightarrow(1,0) } \frac{1}{\sqrt{2x+y}+\sqrt{2x-y} }  \\
  &= \frac{1}{2\sqrt{2} }\\
  &= \frac{\sqrt{2}}{2}
\end{align*}
% -  -  -  -  -  -  -  -  -  -  -  -  -  -  -  -  -  -  -  -  -  -  -  -  -  -  -  -  -  -  
\end{enumerate}
% ~~~~~~~~~~~~~~~~~~~~~~~~~~~~~~~~~~~~~~~~~~~~~~~~~~~~~~~~~
\item
% CONTINUITY 
The given function is defined everywhere. Now, for it to be continuous at any point $(x_0,y_0,z_0)$, we require that 
\begin{align*} 
   \lim_{(x,y,z)\rightarrow(x_0,y_0,z_0) } g(x,y,z) = g(x_0,y_0,z_0)
 \end{align*}
If we evaluate the limit as $g$ approaches $(0,0,1)$, we have
\begin{align*}
\lim_{(x,y,z)\rightarrow(0,0,1) } f(x,y,z) 
&= \lim_{(x,y,z)\rightarrow(0,0,1) } x+y+2z \\
&= 0+0+2(1) \\
&= 2
\end{align*}
However, at $(0,0,1)$, $g = 2$. So the given function is not continuous at the point $(0,0,1)$. Elsewhere, the function is a polynomial in three variables, and so will be continuous everywhere except at the point $(0,0,1)$. 
% ~~~~~~~~~~~~~~~~~~~~~~~~~~~~~~~~~~~~~~~~~~~~~~~~~~~~~~~~~
\item The lines $y=m(x-1)$ all pass through the limit point, $(1,0)$, where $m$ is any real number. Approaching the limit point along these lines, our limit becomes
\begin{align*} 
   \lim_{(x,y)\rightarrow(1,0) } \frac{x(x-1)^3 + y^2}{4(x-1)^2+9y^3} 
   &= \lim_{(x,y)\rightarrow(1,0) } \frac{x(x-1)^3 + m^2(x-1)^2}{4(x-1)^2+9m^3(x-1)^3} \\
   &= \lim_{(x,y)\rightarrow(1,0) } \frac{x(x-1) + m^2}{4+9(x-1)} \\
   &= \frac{m^2}{4}
 \end{align*}
 The result depends on $m$, which is an arbitrary value. Therefore, the limit does not exist. 
% ~~~~~~~~~~~~~~~~~~~~~~~~~~~~~~~~~~~~~~~~~~~~~~~~~~~~~~~~~

\item
\begin{enumerate}
\item To plot the level curves, we set $z = c$, and solve for $y$: 
\begin{align*} 
  c &= \frac{\ln y}{x^2} \\
  cx^2&= \ln y \\
  y &= e^{cx^2}
\end{align*}
\begin{figure}[!htbp]
  \begin{center}
    \includegraphics[width=0.4\textwidth]{LevelCurvesA.pdf}
  \end{center}
\end{figure}
% -  -  -  -  -  -  -  -  -  -  -  -  -  -  -  -  -  -  -  -  -  -  -  -  -  -  -  -  -  -  
\item To plot the level curves, we set $z = c$, and solve for $y$: 
\begin{align*} 
  c &= \frac{x^2}{x^2+y^2} \\
  cy^2 &= (1-c)x^2
\end{align*}
If $c=0$, then we obtain the level curve $x = 0$. If $c \ne0$, then
\begin{align*} 
  y &= \pm \sqrt{\frac{1-c}{c}}x, \quad c \ne0
\end{align*}
For $c=2$, $y$ is undefined, so the level curves do not exist. The level curves for the other values of $c$ are shown in the graph below and are as follows:
\begin{align*} 
  \text{if } c&=0, \quad x = 0 \text{ (the vertical line that lies on the } y \text{-axis})\\
  \text{if } c&=1/4, y = \pm \sqrt{3}x\\
  \text{if } c&=1/2, y = \pm x\\  
  \text{if } c&=1, \quad y = 0    
\end{align*}

\begin{figure}[!htbp]
  \begin{center}
    \includegraphics[width=0.4\textwidth]{LevelCurvesB.jpg}
  \end{center}
\end{figure}

\end{enumerate}
% ~~~~~~~~~~~~~~~~~~~~~~~~~~~~~~~~~~~~~~~~~~~~~~~~~~~~~~~~~
\item
\begin{enumerate}
\item The set of straight lines that pass through $(2,1)$ are given by $y = c(x-2) + 1$. Rearranging yields the equation 
\begin{align*}
c = \frac{y-1}{x-2}
\end{align*}
The desired function is 
\begin{align*}
f(x,y) = \frac{y-1}{x-2}
\end{align*}
\item The set of circles with radius $e^c$ are given by $x^2 + y^2 = e^c$. Applying the natural logarithm to both sides of the equation yields 
\begin{align*}
\ln(x^2+y^2) = c
\end{align*}
The desired function is 
\begin{align*}
f(x,y) =\ln(x^2+y^2)
\end{align*}\end{enumerate}
% ~~~~~~~~~~~~~~~~~~~~~~~~~~~~~~~~~~~~~~~~~~~~~~~~~~~~~~~~~
% PROVIDE EXAMPLE, LIMIT
\item 
\begin{enumerate}
\item
Suppose we let 
\begin{align*}
f(x,y) = \frac{y}{(x-1)^2}
\end{align*}
Then, along any parabola $y = m(x-1)^2$ the limit becomes
\begin{align*}
\lim_{(x,y)\rightarrow(1,0) } \frac{y}{(x-1)^2} = \lim_{(x,y)\rightarrow(1,0) } \frac{m(x-1)^2}{(x-1)^2} = m
\end{align*}
The function we chose therefore meets the specified criteria.
\item
The limit cannot exist because the limit depends on $m$, which is an arbitrary real number.
\end{enumerate}
% ~~~~~~~~~~~~~~~~~~~~~~~~~~~~~~~~~~~~~~~~~~~~~~~~~~~~~~~~~
% TEMPERATURE DISTRIBUTION
\item 
\BEN
\item The temperature can be written as 
\begin{align*}T(x,y) = \frac{k}{\sqrt{x^2 + y^2}},\end{align*} 
where $k$ is an unknown constant of proportionality.
\item The level curves are solution sets of the equation $T(x,y)=c$, for $c\in\R$. Therefore,
\begin{align*}
  T(x,y) = c &= \frac{k}{\sqrt{x^2 + y^2}} \\
  c^2 &= \frac{k^2}{x^2 + y^2} \\
  x^2 + y^2 &= \frac{k^2}{c^2}
\end{align*} 
The level curves are concentric circles with radius $k/c$. 
\item At the point (2,1), the temperature is known, which allows us to solve for $k$. 
\begin{align*}
  T(2,1) = 10 &= \frac{k}{\sqrt{2^2 + 1^2}}\\
  k& = 10\sqrt{5}
\end{align*} 
At (1,3), the temperature is
\begin{align*}
  T(1,3) &= \frac{10\sqrt{5}}{\sqrt{1^2 + 3^2}}\\
  & = 10\sqrt{\frac{5}{4}}
\end{align*} 
\EEN
% ~~~~~~~~~~~~~~~~~~~~~~~~~~~~~~~~~~~~~~~~~~~~~~~~~~~~~~~~~
% ELECTRIC POTENTIAL DISTRIBUTION
\item The level curves are solution sets of the equation $V(x,y) = K$, for constant $K\in\R$.
\begin{align*}
V(x,y) = K &= \frac{c}{\sqrt{r^2 - x^2 -y^2}} \\
r^2-x^2-y^2 &= \frac{c^2}{K^2} \\
x^2+y^2 &= r^2 -  \frac{c^2}{K^2}
\end{align*}
The left-hand side must be positive, so we must have that 
\begin{align*}
r^2 &>  \frac{c^2}{K^2}\\
K^2 &> c^2/r^2\\
 |K| &<  |c|/|r|
\end{align*}
But $c, K$, and $r$ are all positive constants, so this simplifies to $K<\frac{c}{r}$. The level curves are concentric circles, centered at the origin, with radius $r^2 -  \frac{c^2}{K^2}$, for $K<\frac{c}{r}$.
% ~~~~~~~~~~~~~~~~~~~~~~~~~~~~~~~~~~~~~~~~~~~~~~~~~~~~~~~~~
% LIMIT PROOF
\item 
If we approach the limit point along the $x$-axis, then $y=0$ and we obtain
\begin{align*}
  \lim_{(x,y)\rightarrow(0,0) } \frac{x(0)^2}{x^2+(0)^2} = 0
\end{align*}
We obtain the same result if we approach the origin along the $y$-axis, or along any line $y=mx$. It would seem that the limiting value could exist and could be equal to 0. To show that this is the case, we must show that for any $\epsilon > 0$, there exists a $\delta > 0$ such that   
\begin{align*}
 \Big|  \frac{xy^2}{x^2+y^2} - 0 \Big| < \epsilon \text{  whenever  } 0 < \sqrt{(x-0)^2 + (y-0)^2} < \delta ,
\end{align*}
or simply
\begin{align*}
 \Big|  \frac{xy^2}{x^2+y^2} \Big| < \epsilon \text{  whenever  } 0 < \sqrt{x^2 + y^2} < \delta .
\end{align*}
Now, $\big| x^2 + y^2\big| = x^2 + y^2$, and $|y^2| = y^2$, so
\begin{align*}
 \Big|  \frac{xy^2}{x^2+y^2} \Big| &=  \frac{ \big|xy^2 \big|}{ \big|x^2+y^2 \big|}  =   \frac{ \big|x\big|y^2 }{ x^2+y^2 }
\end{align*}
Also, $|x| \le \sqrt{x^2+y^2}$, and $y^2 \le x^2+y^2$, so
\begin{align*}
 \Big|  \frac{xy^2}{x^2+y^2} \Big| 
 &=  \frac{ \big|x\big|y^2 }{ x^2+y^2 }\\
 &\le \frac{ \ \sqrt{x^2+y^2} ( x^2+y^2 ) }{ x^2+y^2 }\\
 &= \sqrt{x^2+y^2}
\end{align*}
So if $\delta = \epsilon$, then 
\begin{align*}
 \Big|  \frac{xy^2}{x^2+y^2} \Big| < \epsilon \text{  whenever  } 0 < \sqrt{x^2 + y^2} < \delta .
\end{align*}
% ~~~~~~~~~~~~~~~~~~~~~~~~~~~~~~~~~~~~~~~~~~~~~~~~~~~~~~~~~

\end{enumerate}

\end{document}
