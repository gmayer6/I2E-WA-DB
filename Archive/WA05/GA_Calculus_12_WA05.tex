\documentclass{article}
% ~ ~ ~ ~ ~ ~ ~ ~ ~ ~ ~ ~ ~ ~ ~ ~ ~ ~ ~ ~ ~ ~ ~ ~ ~ ~ ~ ~ ~ ~ ~ ~ ~ ~ ~ ~ ~ ~ ~ ~ ~ ~ ~ ~ ~ ~ ~ ~

\usepackage[utf8]{inputenc}
\usepackage{amsmath}
\usepackage{amssymb} % for \mathbb
\usepackage{graphicx} % for figures
\usepackage{color}
\usepackage[usenames,dvipsnames]{xcolor}
\usepackage{hyperref} % for hyperlinks
\usepackage{float} % for figures
% ~ ~ ~ ~ ~ ~ ~ ~ ~ ~ ~ ~ ~ ~ ~ ~ ~ ~ ~ ~ ~ ~ ~ ~ ~ ~ ~ ~ ~ ~ ~ ~ ~ ~ ~ ~ ~ ~ ~ ~ ~ ~ ~ ~ ~ ~ ~ ~
% CUSTOM COLOUR
\definecolor{DGrey}{rgb}{0.1,0.1,0.1} % Dark Grey
% ~ ~ ~ ~ ~ ~ ~ ~ ~ ~ ~ ~ ~ ~ ~ ~ ~ ~ ~ ~ ~ ~ ~ ~ ~ ~ ~ ~ ~ ~ ~ ~ ~ ~ ~ ~ ~ ~ ~ ~ ~ ~ ~ ~ ~ ~ ~ ~
\title{ Introduction to Engineering Written Assignment Questions}
\date{January 2013}
\author{Greg Mayer and Daniel Connelly}
% ~ ~ ~ ~ ~ ~ ~ ~ ~ ~ ~ ~ ~ ~ ~ ~ ~ ~ ~ ~ ~ ~ ~ ~ ~ ~ ~ ~ ~ ~ ~ ~ ~ ~ ~ ~ ~ ~ ~ ~ ~ ~ ~ ~ ~ ~ ~ ~
% HEADER/FOOTER
\usepackage{fancyheadings}
\pagestyle{myheadings} % set headings to be user defined
\fancyhead{} % To create custom header, clear default layout
\renewcommand{\subsectionmark}[1]{\markright{{\color{DGrey}\thesubsection} \ {\color{DGrey}#1}}}
\fancyhead[LE,LO]{\subsectionmark} % To create custom header, clear default layout

% ~ ~ ~ ~ ~ ~ ~ ~ ~ ~ ~ ~ ~ ~ ~ ~ ~ ~ ~ ~ ~ ~ ~ ~ ~ ~ ~ ~ ~ ~ ~ ~ ~ ~ ~ ~ ~ ~ ~ ~ ~ ~ ~ ~ ~ ~ ~ ~
% ENUMERATION

\usepackage{enumitem}   % so that question numbers can be formatted 
\setenumerate[1]{label=\thesubsection.\arabic*.} % enumerate environment: add section numbers to items
% ~ ~ ~ ~ ~ ~ ~ ~ ~ ~ ~ ~ ~ ~ ~ ~ ~ ~ ~ ~ ~ ~ ~ ~ ~ ~ ~ ~ ~ ~ ~ ~ ~ ~ ~ ~ ~ ~ ~ ~ ~ ~ ~ ~ ~ ~ ~ ~
% MARGINS
\usepackage{anysize}
\marginsize{2.5cm}{2.5cm}{1cm}{1cm}
% ~ ~ ~ ~ ~ ~ ~ ~ ~ ~ ~ ~ ~ ~ ~ ~ ~ ~ ~ ~ ~ ~ ~ ~ ~ ~ ~ ~ ~ ~ ~ ~ ~ ~ ~ ~ ~ ~ ~ ~ ~ ~ ~ ~ ~ ~ ~ ~
% PAGE NUMBERING
\pagenumbering{arabic}
% ~ ~ ~ ~ ~ ~ ~ ~ ~ ~ ~ ~ ~ ~ ~ ~ ~ ~ ~ ~ ~ ~ ~ ~ ~ ~ ~ ~ ~ ~ ~ ~ ~ ~ ~ ~ ~ ~ ~ ~ ~ ~ ~ ~ ~ ~ ~ ~
% Custom Commands
\newcommand{\Emph}[1]{\textbf{#1}} % Emphasize
\newcommand{\R}{\mathbb{R}} 
\newcommand{\BM}{\begin{bmatrix}} % Begin Matrix
\newcommand{\EM}{\end{bmatrix}} % End Matrix
\newcommand{\BEN}{\begin{enumerate}[leftmargin=1.1cm]}% Begin ENumerate
\newcommand{\EEN}{\end{enumerate}} % End ENumerate
\newcommand{\MB}{\mathbf} % Math Bold

\newcommand{\px}{\frac{\partial}{\partial x}} % Partial wrt x
\newcommand{\py}{\frac{\partial}{\partial y}} % Partial wrt y

\newcommand{\pfx}{\frac{\partial f}{\partial x}} % Partial of f wrt x
\newcommand{\pfy}{\frac{\partial f}{\partial y}} % Partial of f wrt y
\newcommand{\pfxy}{\frac{\partial^2 f}{\partial y \partial x}} % Partial of f wrt y
\newcommand{\pfyx}{\frac{\partial^2 f}{\partial x \partial y}} % Partial of f wrt y

\newcommand{\ux}{\frac{\partial u}{\partial x }} % Partial of u wrt x
\newcommand{\uk}{\frac{\partial u}{\partial k }} % Partial of u wrt k
\newcommand{\ut}{\frac{\partial u}{\partial t}} % Partial of u wrt t
\newcommand{\utt}{\frac{\partial^2u}{\partial t^2}} % Partial of u wrt t
\newcommand{\us}{\frac{\partial u}{\partial s}} % Partial of u wrt t
\newcommand{\uss}{\frac{\partial^2 u}{\partial s^2}} % Partial of u wrt t
\newcommand{\kx}{\frac{\partial k}{\partial x }} % Partial of k wrt x
\newcommand{\kt}{\frac{\partial k}{\partial t }} % Partial of k wrt t

\newcommand{\pxu}{\frac{\partial x}{\partial u}} % x wrt u
\newcommand{\pxv}{\frac{\partial x}{\partial v}} % x wrt v
\newcommand{\pxw}{\frac{\partial x}{\partial w}} % x wrt v
\newcommand{\pxt}{\frac{\partial x}{\partial t}} % x wrt t
\newcommand{\pyu}{\frac{\partial y}{\partial u}} % y wrt u
\newcommand{\pyv}{\frac{\partial y}{\partial v}} % y wrt v
\newcommand{\pyw}{\frac{\partial y}{\partial w}} % y wrt v
\newcommand{\pyt}{\frac{\partial y}{\partial t}} % y wrt t
\newcommand{\pzu}{\frac{\partial z}{\partial u}} % z wrt u
\newcommand{\pzv}{\frac{\partial z}{\partial v}} % z wrt v
\newcommand{\pzw}{\frac{\partial z}{\partial w}} % z wrt v
\newcommand{\pzt}{\frac{\partial z}{\partial t}} % z wrt t


\newcommand{\VCT}{\textit{Vector Calculus} by Michael Corral} % Vector Calculus Textbook
\newcommand{\CAT}{\textit{College Algebra} by Carl Stitz and Jeff Zeager} % College Algebra Textbook
\newcommand{\From}{The following questions are related to } % Questions ....
% ~ ~ ~ ~ ~ ~ ~ ~ ~ ~ ~ ~ ~ ~ ~ ~ ~ ~ ~ ~ ~ ~ ~ ~ ~ ~ ~ ~ ~ ~ ~ ~ ~ ~ ~ ~ ~ ~ ~ ~ ~ ~ ~ ~ ~ ~ ~ ~
% ONLY USED FOR EDITING
\newcommand{\rednote}[1]{{\color{red}\textit{\textbf{#1}}}} % Shortcut for formatting notes for developers
\newcommand{\FromC}[1]{{\color{DGrey}\textit{#1}}} % Shortcut for coloring the "from" text
% ~ ~ ~ ~ ~ ~ ~ ~ ~ ~ ~ ~ ~ ~ ~ ~ ~ ~ ~ ~ ~ ~ ~ ~ ~ ~ ~ ~ ~ ~ ~ ~ ~ ~ ~ ~ ~ ~ ~ ~ ~ ~ ~ ~ ~ ~ ~ ~
% AUGMENTED MATRIX MACRO
% thanks to http://tex.stackexchange.com/questions/2233/whats-the-best-way-make-an-augmented-coefficient-matrix
\newenvironment{amatrix}[1]{%
  \left[\begin{array}{@{}*{#1}{c}|c@{}}
}{%
  \end{array}\right]
}
% ~ ~ ~ ~ ~ ~ ~ ~ ~ ~ ~ ~ ~ ~ ~ ~ ~ ~ ~ ~ ~ ~ ~ ~ ~ ~ ~ ~ ~ ~ ~ ~ ~ ~ ~ ~ ~ ~ ~ ~ ~ ~ ~ ~ ~ ~ ~ ~
% PAGE LAYOUT
\addtolength{\topmargin}{10pt}
\addtolength{\headsep}{10pt}
\addtolength{\textheight}{-20pt}


\usepackage{multienum}
\usepackage{wrapfig}

\date{}
% ~ ~ ~ ~ ~ ~ ~ ~ ~ ~ ~ ~ ~ ~ ~ ~ ~ ~ ~ ~ ~ ~ ~ ~ ~ ~ ~ ~ ~ ~ ~ ~ ~ ~ ~ ~ ~ ~ ~ ~ ~ ~ ~ ~ ~ ~ ~ ~
\begin{document}
\begin{center}
\textsc{\LARGE Written Assignment 5}\\[0.5cm]
\end{center}
\From sections 2.5, 2.7 and parts of 4.6 of \VCT.

\section*{Questions}

\begin{enumerate}

% ~~~~~~~~~~~~~~~~~~~~~~~~~~~~~~~~~~~~~~~~~~~~~~~~~~~~~~~~~
\item

Use the second derivative test to classify the local extrema of the
following functions:
\begin{enumerate}
 \item $f(x,y) = x^2 + 2x - xy + y^2$
 \item $h(x,y) = \sin(xy)$
\end{enumerate}

% ~~~~~~~~~~~~~~~~~~~~~~~~~~~~~~~~~~~~~~~~~~~~~~~~~~~~~~~~~
\item

Find the extreme values of
\begin{enumerate}
 \item the function
  $f(x,y) = e^{-(x^2+y^2)}$
  along the curve $x = y^2$
 \item the function
  $g(x,y) = e^{x-y^2}$ along the boundary of the ellipse
  $\frac{x^2}{4}+y+y^2=1$
\end{enumerate}

% ~~~~~~~~~~~~~~~~~~~~~~~~~~~~~~~~~~~~~~~~~~~~~~~~~~~~~~~~~
\item

Consider the function $f(x,y,z) = \begin{bmatrix}
                                   x^2 + y^2 \\
                                   y^2 + z^2 \\
                                   z^2 + x^2
                                  \end{bmatrix}$.
Find its divergence $\nabla \cdot f$ and curl $\nabla \times f$.

% ~~~~~~~~~~~~~~~~~~~~~~~~~~~~~~~~~~~~~~~~~~~~~~~~~~~~~~~~~
\item

Find the point on the sphere $x^2 + y^2 + z^2 = 1$ that is closest
to the plane $x + 2y + 2z = 5$. (Hint: using some geometric intuition
is is possible to complete this problem without using the method of
Lagrange multipliers.)

% ~~~~~~~~~~~~~~~~~~~~~~~~~~~~~~~~~~~~~~~~~~~~~~~~~~~~~~~~~
\item
\Emph{Application: Cost Optimization}

Suppose you are given a budget of \$500 to build a large glass
triangular prism.  Each rectangular side must have equal dimensions 
and the two triangular sides must also have equal dimensions.  You
can purchase glass for the rectangular sides at a cost of $\$8/ft^2$ 
and for the triangular sides at a cost of $\$10/ft^2$.

What are the dimensions of the prism of largest volume that you can build?

% ~~~~~~~~~~~~~~~~~~~~~~~~~~~~~~~~~~~~~~~~~~~~~~~~~~~~~~~~~
\item
\Emph{Application: Heat Optimization}

Suppose that the temperature in a space is given by the function
$T(x,y,z) = 200xyz^2$.  Find the hottest point on the unit sphere.

(This question taken from section 4.5 of {\it Vector Calculus} by Susan Colley)

\end{enumerate}

%%%%%%%%%%%%%%%%%%%%%%%%%%%%%%%%%%%%%%
%%%%%%%%%%%%%%%%%%%%%%%%%%%%%%%%%%%%%%
%%%%%%%%%%%%%%%%%%%%%%%%%%%%%%%%%%%%%%

% SOLUTIONS
\newpage
\section*{Solutions}
\begin{enumerate}

% ~~~~~~~~~~~~~~~~~~~~~~~~~~~~~~~~~~~~~~~~~~~~~~~~~~~~~~~~~
\item

Local extreme values of a function occur at points where the gradient of
the function is the zero vector.

 \begin{enumerate}
  \item First we find the critical points:
   \begin{align*}
    \nabla f(x,y) &= \begin{bmatrix} 0 \\ 0 \end{bmatrix} \\
    \begin{bmatrix}
     \dfrac{\partial}{\partial x}(x^2 + 2x - xy + y^2) \\
     \dfrac{\partial}{\partial y}(x^2 + 2x - xy + y^2)
    \end{bmatrix} &= \begin{bmatrix} 0 \\ 0 \end{bmatrix} \\
    \begin{bmatrix}
     2x + 2 \\
     -x + 2y
    \end{bmatrix} &= \begin{bmatrix} 0 \\ 0 \end{bmatrix} \\
    x &= -1 \\
    y &= -\frac{1}{2}
   \end{align*}
   To categorize this point $(-1, -\frac{1}{2})$ as a local minimum or maximum
   we apply the second partial derivative test:
   \begin{align*}
    D(x,y)
    &= \begin{vmatrix}
     \dfrac{\partial^2 f(x,y)}{\partial x^2} &
      \dfrac{\partial^2 f(x,y)}{\partial y \partial x} \\
     \dfrac{\partial^2 f(x,y)}{\partial x \partial y} &
      \dfrac{\partial^2 f(x,y)}{\partial y^2}
    \end{vmatrix} \\
    &= \begin{vmatrix}
     \dfrac{\partial^2}{\partial x^2}(x^2 + 2x - xy + y^2) &
      \dfrac{\partial^2}{\partial y \partial x}(x^2 + 2x - xy + y^2) \\
     \dfrac{\partial^2}{\partial x \partial y}(x^2 + 2x - xy + y^2) &
      \dfrac{\partial^2}{\partial y^2}(x^2 + 2x - xy + y^2)
    \end{vmatrix} \\
    &= \begin{vmatrix} 2 & 0 \\ -1 & 2 \end{vmatrix} \\
    &= 4 \\
    &> 0
   \end{align*}
  for all values of $x,y$.
  Additionally, $\dfrac{\partial^2 f(x,y)}{\partial x^2} = 2 > 0$, so that
  the point $(-1, -\frac{1}{2})$ is a local minimum.

  \item Again, we look for the critical points by setting the gradient to zero.
   \begin{align*}
    \nabla h(x,y) &= \begin{bmatrix} 0 \\ 0 \end{bmatrix} \\
    \begin{bmatrix} y\cos(xy) \\ x\cos(xy) \end{bmatrix}
    &= \begin{bmatrix} 0 \\ 0 \end{bmatrix}
   \end{align*}
  The solutions (and therefore the critical points) are the point $(0, 0)$
  and the family of curves $xy = \pi/2 + k\pi$, where $k \in \mathbb{Z}$.

  To categorize the points, again compute the matrix
  \begin{align*}
   D(x,y)
    &= \begin{vmatrix}
     \dfrac{\partial^2 h(x,y)}{\partial x^2} &
      \dfrac{\partial^2 h(x,y)}{\partial y \partial x} \\
     \dfrac{\partial^2 h(x,y)}{\partial x \partial y} &
      \dfrac{\partial^2 h(x,y)}{\partial y^2}
    \end{vmatrix} \\
    &= \begin{vmatrix}
     \dfrac{\partial^2}{\partial x^2}(\sin(xy)) &
      \dfrac{\partial^2}{\partial y \partial x}(\sin(xy)) \\
     \dfrac{\partial^2}{\partial x \partial y}(\sin(xy)) &
      \dfrac{\partial^2}{\partial y^2}(\sin(xy))
    \end{vmatrix} \\
    &= \begin{vmatrix}
     -y^2\sin(xy) & \cos(xy) - xy\sin(xy) \\
     \cos(xy) - xy\sin(xy) & -x^2\sin(xy)
    \end{vmatrix} \\
    D(0,0) &= \begin{vmatrix} 0 & 1 \\ 1 & 0 \end{vmatrix} \\
           &= -1 \\
   \end{align*}

   So $(0, 0)$ is a saddle point.

   \begin{align*}
    D(x, \frac{\pi/2 + k\pi}{x})
    &= \begin{vmatrix}
     -\dfrac{\pi/2 + k\pi}{x}^2\sin(\pi/2 + k\pi)
     & \cos(\pi/2 + k\pi) - (\pi/2 + k\pi)\sin(\pi/2 + k\pi) \\
     \cos(\pi/2 + k\pi) - (\pi/2 + k\pi)\sin(\pi/2 + k\pi)
     & -x^2\sin(\pi/2 + k\pi)
    \end{vmatrix} \\
    &= \begin{vmatrix}
     -(\dfrac{\pi/2 + k\pi}{x})^2 (-1)^k & - (\pi/2 + k\pi)(-1)^k \\
     - (\pi/2 + k\pi)(-1)^k & -x^2 (-1)^k
    \end{vmatrix} \\
    &= (\pi/2 + k\pi)^2 - (\pi/2 + k\pi)^2 \\
    &= 0
  \end{align*}

  So the test is inconclusive.

 \end{enumerate}

% ~~~~~~~~~~~~~~~~~~~~~~~~~~~~~~~~~~~~~~~~~~~~~~~~~~~~~~~~~
\item

We use the method of Lagrange Multipliers.  To find the extreme values
of the function $f(x,y)$ subject to the constraint $g(x,y)=c$, we define
a new function $F(x,y,t) = f(x,y) + t(g(x,y) - c)$ and solve the equation
$\nabla F(x,y,t) = 0$.  The solutions $(x,y)$ are the candidate solutions,
which we test by plugging into the original equation $f$.

\begin{enumerate}
 \item
  Define $F(x,y,t) = e^{-(x^2+y^2)} - t(x - y^2)$.  Then
  \begin{align}
   \nabla F(x,y,t) &= \begin{bmatrix} 0 \\ 0 \\ 0 \end{bmatrix} \nonumber \\
   \begin{bmatrix}
    -2xe^{-(x^2+y^2)} - t \\
    -2ye^{-(x^2+y^2)} + 2yt \\
    -(x - y^2)
   \end{bmatrix} &= \begin{bmatrix} 0 \\ 0 \\ 0 \end{bmatrix}. \label{lagr:2}
  \end{align}
  $(0,0)$ is one solution.  When $x \neq 0$, by the third component of
  (\ref{lagr:2}) we see that $y \neq 0$, so that we get $t = e^{-(x^2+y^2)}$
  from the second component, which in turn by the first component shows
  that $x = -1/2$.  However, there is no $y$ that satisfies the third
  component of (\ref{lagr:2}) when $x = -1/2$.  Thus $(0, 0)$ is the only
  extreme value.
  
 \item
  Define $F(x,y,t) = e^{x-y^2} - t(\frac{x^2}{4}+y+y^2 - 1)$.  Then
  \begin{align}
   \nabla F(x,y,t) &= \begin{bmatrix} 0 \\ 0 \\ 0 \end{bmatrix} \nonumber \\
   \begin{bmatrix}
    e^{x-y^2} - tx/2 \\
    -2ye^{x-y^2} - t + 2yt \\
    -(\frac{x^2}{4}+y+y^2 - 1)
   \end{bmatrix} &= \begin{bmatrix} 0 \\ 0 \\ 0 \end{bmatrix}. \label{lagr:1}
  \end{align}
  From the first two components we see that
  \begin{align*}
   e^{x-y^2} &= tx/2 \\
   e^{x - y^2} &= \frac{t-2yt}{-2y}.
  \end{align*}
  This yields the equation $y = \dfrac{1}{2-x}$, so that the extreme values
  of $f$ fall on the intersection of the ellipse $\frac{x^2}{4}+y+y^2 = 1$
  and the curve $y = \dfrac{1}{2-x}$.
  Plugging $y$ into the third component of (\ref{lagr:1}) gives the equation
  $\frac{x^2}{4} + \frac{1}{2-x} + (\frac{1}{2-x})^2 = 1$.

\end{enumerate}

% ~~~~~~~~~~~~~~~~~~~~~~~~~~~~~~~~~~~~~~~~~~~~~~~~~~~~~~~~~
\item

The divergence is given by the equation \[
 \nabla \cdot f = \frac{\partial f_1}{\partial x} + \frac{\partial f_2}{\partial y} + \frac{\partial f_3}{\partial z}
\] and the curl by the equation \[
 \nabla \times f = \begin{bmatrix}
  \dfrac{\partial f_3}{\partial y} - \dfrac{\partial f_2}{\partial z} \\
  \dfrac{\partial f_1}{\partial z} - \dfrac{\partial f_3}{\partial x} \\
  \dfrac{\partial f_2}{\partial x} - \dfrac{\partial f_1}{\partial y}
 \end{bmatrix}.
\]  Applying these equations yields the solutions:

\begin{align*}
 \nabla \cdot f &= \frac{\partial f_1}{\partial x} + \frac{\partial f_2}{\partial y} + \frac{\partial f_3}{\partial z} \\
 &= \frac{\partial}{\partial x}(x^2+y^2) + \frac{\partial}{\partial y}(y^2+z^2) + \frac{\partial}{\partial z}(z^2+x^2) \\
 &= 2x + 2y + 2z \\
 \nabla \times f &= \begin{bmatrix}
  \dfrac{\partial f_3}{\partial y} - \dfrac{\partial f_2}{\partial z} \\
  \dfrac{\partial f_1}{\partial z} - \dfrac{\partial f_3}{\partial x} \\
  \dfrac{\partial f_2}{\partial x} - \dfrac{\partial f_1}{\partial y}
 \end{bmatrix} \\
 &= \begin{bmatrix} - 2z \\ - 2x \\ - 2y \end{bmatrix}.
\end{align*}


% ~~~~~~~~~~~~~~~~~~~~~~~~~~~~~~~~~~~~~~~~~~~~~~~~~~~~~~~~~
\item

It is possible to solve this problem using Lagrangian optimization;
however, the closest point on the sphere to the plane will occur at
the point on the sphere where its normal vector is parallel to the plane's
normal vector.

% Should include more explanation for this?  Maybe not a proof,
% but perhaps a picture...

Knowing this, we first find the sphere's normal vector by computing
the gradient of the function $F(x,y,z) = x^2 + y^2 + z^2 - 1$, which by
a straightforward calculation is the vector
$\nabla F = \begin{bmatrix} 2x \\ 2y \\ 2z \end{bmatrix}$.  Now, by
rearranging the equation $x + 2y + 2z = 5$ into the form
$\begin{bmatrix} x - 3 \\ y -2 \\ z + 1 \end{bmatrix} \cdot
 \begin{bmatrix} 1 \\ 2 \\ 2 \end{bmatrix} = 0$, we see that
$\begin{bmatrix} 1 \\ 2 \\ 2 \end{bmatrix}$ is a vector normal to the plane.

We want to find a point on the sphere where these two normal vectors are
parallel; equivalently, we want to find a point on the sphere such that
$\begin{bmatrix} 2x \\ 2y \\ 2z \end{bmatrix}
 = k\begin{bmatrix} 1 \\ 2 \\ 2 \end{bmatrix}$ for some $k$.  We can plug
$x = k/2$, $y = k$, $z = k$ into the equation $x^2 + y^2 + z^2 = 1$, which
yields the solution $k = 2/3$.  Thus $(1/3, 2/3, 2/3)$ is a point on the
sphere where the sphere's normal vector is parallel to the plane's, and
therefore is the point closest to the plane.

% ~~~~~~~~~~~~~~~~~~~~~~~~~~~~~~~~~~~~~~~~~~~~~~~~~~~~~~~~~
\item

Let $l$ be the length
of the rectangular sides of the prism and $w$ be their width, so that
the sides of the equilateral triangle ends are also of length $w$.
The surface area of such a prism is given by the function $A(l,w) =
\frac{\sqrt{3}}{2}l^2 + 3lw$, and its volume by the function $V(l,w) =
\frac{\sqrt{3}}{4}l^2w$.  Since the sides cost $\$10/ft^2$ and the ends
cost $\$8/ft^2$, we wish to maximize the function $V(l,w)$ subject to
the restriction $C(l,w) = 5\sqrt{3}l^2 + 24lw = 500$.

We apply the method of Lagrange Multipliers.  Define the function
$F(l,w) = V(l,w) + t(C(l,w) - 500) =
\frac{\sqrt{3}}{4}l^2w + t(5\sqrt{3}l^2 + 24lw - 500)$;
the constrained extreme values of $V$ satisfy the equation
$\nabla F(l,w) = \mathbf{0}$.

\begin{align}
 \nabla F(l,w) &= \begin{bmatrix} 0 \\ 0 \\ 0 \end{bmatrix} \nonumber \\
 \begin{bmatrix}
  \dfrac{\partial F}{\partial l} \\
  \dfrac{\partial F}{\partial w} \\
  \dfrac{\partial F}{\partial t}
 \end{bmatrix} &= \begin{bmatrix} 0 \\ 0 \\ 0 \end{bmatrix} \nonumber \\
 \begin{bmatrix}
  \frac{\sqrt{3}}{2}lw + 10\sqrt{3}lt + 24wt \\
  \frac{\sqrt{3}}{4}l^2 + 24lt \\
  5\sqrt{3}l^2 + 24lw - 500
 \end{bmatrix} &= \begin{bmatrix} 0 \\ 0 \\ 0 \end{bmatrix} \label{soln:opt_box}
\end{align}

From the second row of (\ref{soln:opt_box}) we get $t = \frac{-\sqrt{3}}{96}l$.
Substitution into the first row yields
\begin{align}
 \frac{\sqrt{3}}{2}lw - \frac{10}{32}l^2 - \frac{\sqrt{3}}{4}lw &= 0 \nonumber \\
 \frac{\sqrt{3}}{4}lw - \frac{10}{32}l^2 &= 0 \nonumber \\
 8\sqrt{3}lw - 10l^2 &= 0 \nonumber \\
 8\sqrt{3}lw &= 10l^2 \nonumber \\
 w &= \frac{5}{4\sqrt{3}}l \label{soln:opt_box2}
\end{align}

Substituting (\ref{soln:opt_box2}) into the third row of (\ref{soln:opt_box})
yields
\begin{align*}
 5\sqrt{3}l^2 + 24l(\frac{5}{4\sqrt{3}}l) &= 500 \\
 5\sqrt{3}l^2 + \frac{30}{\sqrt{3}}l^2 &= 500 \\
 15l^2 + 30l^2 &= 500\sqrt{3} \\
 45l^2 &= 500\sqrt{3} \\
 l^2 &= \frac{100}{3^{3/2}} \\
 l &= \frac{10}{3^{3/4}}
\end{align*}

which, when combined with (\ref{soln:opt_box2}), gives
$w = \dfrac{50}{4\cdot3^{5/4}}$.

% ~~~~~~~~~~~~~~~~~~~~~~~~~~~~~~~~~~~~~~~~~~~~~~~~~~~~~~~~~
\item

We want to optimize the function $T(x,y,z) = 200xyz^2$ subject to the
constraint $x^2+y^2+z^2 = 1$.  Define the function
$F(x,y,z) = 200xyz^2 + t(x^2+y^2+z^2-1)$; the extreme points satisfy the
equation $\nabla F(x,y,z) = 0$.

\begin{align}
 \nabla F(x,y,z) &= \begin{bmatrix} 0 \\ 0 \\ 0 \\ 0 \end{bmatrix} \nonumber \\
 \begin{bmatrix}
  \dfrac{\partial F}{\partial x} \\
  \dfrac{\partial F}{\partial y} \\
  \dfrac{\partial F}{\partial z} \\
  \dfrac{\partial F}{\partial t}
 \end{bmatrix} &= \begin{bmatrix} 0 \\ 0 \\ 0 \\ 0 \end{bmatrix} \nonumber \\
 \begin{bmatrix}
  200yz^2 + 2xt \\
  200xz^2 + 2yt \\
  400xyz + 2zt \\
  x^2 + y^2 + z^2 - 1
 \end{bmatrix} &= \begin{bmatrix} 0 \\ 0 \\ 0 \end{bmatrix} \label{soln:opt_heat}
\end{align}

The third equation of (\ref{soln:opt_heat}) yields $t = -200xy$.  Substituting
this value of $t$ into the first and second equations gives $z^2 = 2x^2 = 2y^2$.
Combining this with the fourth equation yields $4x^2 = 1$, or
$x = \pm \frac{1}{2} \implies y = \pm \frac{1}{2}, z = \pm \frac{1}{\sqrt{2}}$.

\end{enumerate}

\end{document}
