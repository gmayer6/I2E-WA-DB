
\subsection{Partial Derivatives}

\BEN

% ~~~~~~~~~~~~~~~~~~~~~~~~~~~~~~~~~~~~~~~~~~~~~~~~~~~~~~~~~
\item  \Emph{Calculating First and Second Partial Derivatives}
\begin{enumerate}

 \item
  \begin{align*}
   \frac{\partial f}{\partial x}
   &= \frac{\partial}{\partial x}(xy \sin(\frac{x}{y})) \\
   &= y\sin(\frac{x}{y}) + xy\cos(\frac{x}{y})\frac{1}{y} \\
   &= y\sin(\frac{x}{y}) + x\cos(\frac{x}{y})
  \end{align*}
 \begin{align*}
   \frac{\partial^2 f}{\partial x^2}
   &= \frac{\partial}{\partial x}\frac{\partial f}{\partial x} \\
   &= \frac{\partial}{\partial x}(y\sin(\frac{x}{y}) + x\cos(\frac{x}{y})) \\
   &= y\cos(\frac{x}{y})\frac{1}{y} + \cos(\frac{x}{y}) - x\sin(\frac{x}{y})\frac{1}{y} \\
   &= 2\cos(\frac{x}{y}) - \frac{x}{y}\sin(\frac{x}{y})
 \end{align*}
 \begin{align*}
  \frac{\partial^2 f}{\partial y \partial x}
   &= \frac{\partial}{\partial y}\frac{\partial f}{\partial x} \\
   &= \frac{\partial}{\partial y}(y\sin(\frac{x}{y}) + x\cos(\frac{x}{y})) \\
   &= \sin(\frac{x}{y}) + y\cos(\frac{x}{y})(\frac{-x}{y^2}) - x\sin(\frac{x}{y})(\frac{-x}{y^2}) \\
   &= \sin(\frac{x}{y}) - \frac{x}{y}\cos(\frac{x}{y}) + \frac{x^2}{y^2}\sin(\frac{x}{y}) \\
   &= \frac{x^2+y^2}{y^2}\sin(\frac{x}{y}) - \frac{x}{y}\cos(\frac{x}{y}) 
 \end{align*}
 \begin{align*}
   \frac{\partial f}{\partial y}
   &= \frac{\partial}{\partial y}(xy \sin(\frac{x}{y})) \\
   &= x\sin(\frac{x}{y}) + xy\cos(\frac{x}{y})(\frac{-x}{y^2}) \\
   &= x\sin(\frac{x}{y}) - \frac{x^2}{y}\cos(\frac{x}{y})
 \end{align*}
 \begin{align*}
   \frac{\partial^2 f}{\partial x \partial y}
   &= \frac{\partial}{\partial x}\frac{\partial f}{\partial y} \\
   &= \frac{\partial}{\partial x}(x\sin(\frac{x}{y}) - \frac{x^2}{y}\cos(\frac{x}{y})) \\
   &= \sin(\frac{x}{y}) + x\cos(\frac{x}{y})(\frac{1}{y}) - \frac{2x}{y}\cos(\frac{x}{y}) + \frac{x^2}{y}\sin(\frac{x}{y})(\frac{1}{y}) \\
   &= \frac{x^2+y^2}{y^2}\sin(\frac{x}{y}) - \frac{x}{y}\cos(\frac{x}{y}) \\
   \nabla f
   &= \begin{bmatrix}
       \frac{\partial f}{\partial x} \\
       \frac{\partial f}{\partial y}
      \end{bmatrix} \\
   &= \begin{bmatrix}
       y\sin(\frac{x}{y}) + x\cos(\frac{x}{y}) \\
       x\sin(\frac{x}{y}) - \frac{x^2}{y}\cos(\frac{x}{y})
      \end{bmatrix}
  \end{align*}

 \item
  \begin{align*}
   \frac{\partial f}{\partial x}
   &= \frac{\partial}{\partial x}(\frac{x+y}{e^{xy}}) \\
   &= \frac{e^{xy} - (x+y)ye^{xy}}{e^{2xy}} \\
   &= \frac{1 - xy - y^2}{e^{xy}}
 \end{align*}
 \begin{align*}
   \frac{\partial^2 f}{\partial x^2}
   &= \frac{\partial}{\partial x}\frac{\partial f}{\partial x} \\
   &= \frac{\partial}{\partial x}(\frac{1 - xy - y^2}{e^{xy}}) \\
   &= \frac{e^{xy}(-y) - (1 - xy - y^2)ye^{xy}}{e^{2xy}} \\
   &= \frac{-2y + xy^2 + y^3}{e^{xy}}
 \end{align*}
 \begin{align*}
   \frac{\partial^2 f}{\partial y \partial x}
   &= \frac{\partial}{\partial y}\frac{\partial f}{\partial x} \\
   &= \frac{\partial}{\partial y}(\frac{1 - xy - y^2}{e^{xy}}) \\
   &= \frac{e^{xy}(-x - 2y) - (1 - xy - y^2)xe^{xy}}{e^{2xy}} \\
   &= \frac{-2x - 2y + x^2y + xy^2}{e^{xy}}
 \end{align*}
 \begin{align*}
   \frac{\partial f}{\partial y}
   &= \frac{\partial}{\partial y}(\frac{x+y}{e^{xy}}) \\
   &= \frac{e^{xy} - (x+y)xe^{xy}}{e^{2xy}} \\
   &= \frac{1 - x^2 - xy}{e^{xy}}
 \end{align*}
 \begin{align*}
   \frac{\partial^2 f}{\partial y^2}
   &= \frac{\partial}{\partial y}\frac{\partial f}{\partial y} \\
   &= \frac{\partial}{\partial y}\Big(\frac{1 - x^2 - xy}{e^{xy}}\Big) \\
   &= \frac{e^{xy}(-x) - (1 - x^2 - xy)xe^{xy}}{e^{2xy}} \\
   &= \frac{-2x + x^3 + x^2y}{e^{xy}}
 \end{align*}
 \begin{align*}
   \frac{\partial^2 f}{\partial x \partial y}
   &= \frac{\partial}{\partial x}\frac{\partial f}{\partial y} \\
   &= \frac{\partial}{\partial x}\Big(\frac{1 - x^2 - xy}{e^{xy}}\Big) \\
   &= \frac{e^{xy}(-2x - y) - (1 - x^2 - xy)ye^{xy}}{e^{2xy}} \\
   &= \frac{-2x - 2y + x^2y + xy^2}{e^{xy}}
 \end{align*}
 \begin{align*}
   \nabla f
   &= \begin{bmatrix}
       \frac{\partial f}{\partial x} \\
       \frac{\partial f}{\partial y}
      \end{bmatrix} \\
   &= \begin{bmatrix}
       \dfrac{1 - xy - y^2}{e^{xy}} \\
       \dfrac{1 - x^2 - xy}{e^{xy}}
      \end{bmatrix}
  \end{align*}

\end{enumerate}
% ~~~~~~~~~~~~~~~~~~~~~~~~~~~~~~~~~~~~~~~~~~~~~~~~~~~~~~~~~
\item
\Emph{Finding Tangent Planes}\\
 Recall that the equation of the plane tangent to the surface defined by
 $g(x,y,z) = 0$ at the point $(x_0, y_0, z_0)$ is $\nabla g(x_0,y_0,z_0)
 \cdot \begin{bmatrix} x - x_0 \\ y - y_0 \\ z - z_0 \end{bmatrix} = 0$.

 \begin{enumerate}
  \item Define $g(x,y,z) = f(x,y) - z = \frac{x^2}{9} - y^2 - z$, so that the
   surface  $z = f(x,y)$ is equivalently defined by the equation
   $g(x,y,z) = 0$.
   We can then find the equation of the tangent plane at the point
   $(2, \frac{1}{3}, \frac{1}{3})$ by computing $\nabla g$ and applying the
   above formula:

   \begin{align*}
    \nabla g(x,y,z) &=
     \begin{bmatrix} \frac{2x}{9} \\ -2y \\ -1 \end{bmatrix} \\
    \nabla g(2,\frac{1}{3},\frac{1}{3}) &=
     \begin{bmatrix} \frac{4}{9} \\ -\frac{2}{3} \\ -1 \end{bmatrix} \\
   \end{align*}

   This yields the equation
   \begin{align*}
    \frac{4}{9}(x - 2) - \frac{2}{3}\Big(y - \frac{1}{3}\Big) - \Big(z - \frac{1}{3}\Big) &= 0 \\
    \frac{4}{9}x - \frac{2}{3}y - z &= \frac{1}{3} \\
    4x - 6y - 9z &= 3.
   \end{align*}

  \item Define $g(x,y,z) = x^2 + x + y - z^2 - 7$.  Then compute
   \begin{align*}
    \nabla g(x,y,z) &= \begin{bmatrix} 2x + 1 \\ 1 \\ -2z \end{bmatrix} \\
    \nabla g(2,5,2) &= \begin{bmatrix} 5 \\ 1 \\ -4 \end{bmatrix}.
   \end{align*}

   Thus the equation of the tangent plane is
   \begin{align*}
    5(x - 2) + (y - 5) - 4(z - 2) &= 0 \\
    5x + y - 4z &= 7.
   \end{align*}
 \end{enumerate}

% ~~~~~~~~~~~~~~~~~~~~~~~~~~~~~~~~~~~~~~~~~~~~~~~~~~~~~~~~~
\item \Emph{Implicit Differentiation}\\ 
We must first compute the partial derivatives $\frac{\partial z}{\partial x}$ and $\frac{\partial z}{\partial y}$.

 \begin{align*}
  \frac{\partial}{\partial x}(xy) &= \frac{\partial}{\partial x}(z^{x+y}) \\
  y &= \frac{\partial}{\partial x}\Big(e^{\ln(z)(x+y)}\Big) \\
  y &= e^{\ln(z)(x+y)}\frac{\partial}{\partial x}\Big(\ln(z)(x+y)\Big) \\
  y &= e^{\ln(z)(x+y)}\Big(\frac{1}{z}\frac{\partial z}{\partial x}(x+y) + \ln(z)\Big) \\
  y &= z^{x+y}\Big(\frac{x+y}{z}\frac{\partial z}{\partial x} + \ln(z)\Big) \\
  y &= \frac{\partial z}{\partial x}(x+y)z^{x+y-1} + z^{x+y}\ln(z) \\
  \frac{\partial z}{\partial x} &= \frac{y - z^{x+y}\ln(z)}{(x+y)z^{x+y-1}}
 \end{align*}
The derivation for the partial derivative with respect to $y$ is similar.
 \begin{align*}
  \frac{\partial}{\partial y}(xy) &= \frac{\partial}{\partial y}(z^{x+y}) \\
  x &= \frac{\partial}{\partial y}(e^{\ln(z)(x+y)}) \\
  x &= e^{\ln(z)(x+y)}\frac{\partial}{\partial y}\big(\ln(z)(x+y)\big) \\
  x &= e^{\ln(z)(x+y)}\Big(\frac{1}{z}\frac{\partial z}{\partial y}(x+y) + \ln(z)\Big) \\
  x &= z^{x+y}\Big(\frac{x+y}{z}\frac{\partial z}{\partial y} + \ln(z)\Big) \\
  x &= \frac{\partial z}{\partial y}(x+y)z^{x+y-1} + z^{x+y}\ln(z) \\
  \frac{\partial z}{\partial y} &= \frac{x - z^{x+y}\ln(z)}{(x+y)z^{x+y-1}}
 \end{align*}

So $\nabla h(x,y) = \begin{bmatrix}
                \dfrac{y - z^{x+y}\ln(z)}{(x+y)z^{x+y-1}} \\\\
                \dfrac{x - z^{x+y}\ln(z)}{(x+y)z^{x+y-1}}
               \end{bmatrix}$.

% ~~~~~~~~~~~~~~~~~~~~~~~~~~~~~~~~~~~~~~~~~~~~~~~~~~~~~~~~~
\item
\Emph{Calculating the Rate of Change of a Function of Two Variables}
 \begin{enumerate}
  \item $f$ will increase fastest in the direction of the vector
   $\nabla f(x_0,y_0)$ from the point $(x_0, y_0)$.  Therefore we compute and
   evaluate the gradient $\nabla f$ at the point $(1, 1)$:

   \begin{align*}
    \nabla f = \begin{bmatrix}
                 \dfrac{\partial f}{\partial x} \\\\
                 \dfrac{\partial f}{\partial y}
                \end{bmatrix} 
             &= \begin{bmatrix}
                 \dfrac{\partial}{\partial x}(e^{-(x^2+x+1+y^2-2y)}) \\\\
                 \dfrac{\partial}{\partial y}(e^{-(x^2+x+1+y^2-2y)})
                \end{bmatrix} \tag{1} \label{fastest_incr:grad}  \\
             &= \begin{bmatrix}
                 (-2x - 1)e^{-(x^2+x+1+y^2-2y)} \\
                 (2y - 2)e^{-(x^2+x+1+y^2-2y)}
                \end{bmatrix}
   \end{align*}
   Therefore, the gradient at the point (1,1) is
   \begin{align*}        
    \nabla f(1, 1) &= \begin{bmatrix} -3e^{-2} \\ 0 \end{bmatrix}.
   \end{align*}
 \item The rate of change of $f$ in the direction of the vector $\mathbf{v}$
  from the point $(x, y)$ is given by $\nabla f(x,y) \cdot \mathbf{v}$.  Using
  Equation (\ref{fastest_incr:grad}) we find that $\nabla f(2, 1) =
  \begin{bmatrix} -5e^{-6} \\ 0 \end{bmatrix}$.  The direction from $(2, 1)$
  to $(5, 4)$ is given by the vector $\begin{bmatrix} 3 \\ 3 \end{bmatrix}$;
  thus the rate of change of $f$ from $(2, 1)$ towards $(5, 4)$ is
  $\begin{bmatrix} -5e^{-6} \\ 0 \end{bmatrix} \cdot
   \begin{bmatrix} 3 \\ 3 \end{bmatrix} = -15e^{-6}$.

 \end{enumerate}

% ~~~~~~~~~~~~~~~~~~~~~~~~~~~~~~~~~~~~~~~~~~~~~~~~~~~~~~~~~
\item  \Emph{Finding a Jacobian}\\ 
First we find the partial derivatives of each component with respect to each
variable:

 \begin{align*}
  \frac{\partial f_1}{\partial x} &= \cos(x + y) \\
  \frac{\partial f_1}{\partial y} &= \cos(x + y) \\
  \frac{\partial f_2}{\partial x} &= \frac{1}{x} \\
  \frac{\partial f_2}{\partial y} &= \frac{1}{y}
 \end{align*}

The Jacobian matrix, then, is
$\begin{bmatrix}
 \cos(x + y) & \cos(x + y) \\ \frac{1}{x} & \frac{1}{y}
\end{bmatrix}$.
% ~~~~~~~~~~~~~~~~~~~~~~~~~~~~~~~~~~~~~~~~~~~~~~~~~~~~~~~~~
\item \Emph{The Two-Dimensional Heat Equation}

We only need to verify that Equation (\ref{eqn:heat}) holds for the given function $u$:

\begin{align*}
 \frac{\partial u}{\partial t} &= \frac{K_0}{c\rho} \frac{\partial^2 u}{\partial x^2} \\
 -\frac{K_0}{c\rho}e^{-tK_0/(c\rho)}\sin(x) &=
 \frac{K_0}{c\rho}\frac{\partial}{\partial x}\Big(e^{-tK_0/(c\rho)}\cos(x)\Big) \\
 -\frac{K_0}{c\rho}e^{-tK_0/(c\rho)}\sin(x) &=
 -\frac{K_0}{c\rho}e^{-tK_0/(c\rho)}\sin(x)
\end{align*}

which is true.  So $u(x,t) = e^{-tK_0/(c\rho)}\sin(x)$ is indeed a solution
to the Heat Equation (\ref{eqn:heat}).
% ~~~~~~~~~~~~~~~~~~~~~~~~~~~~~~~~~~~~~~~~~~~~~~~~~~~~~~~~~
\item  \Emph{Differentiability}\\ 
Consider the function $f(x,y) = \big|x\big| + 5y$.
\begin{itemize}
\item $f(0,1)$ is defined. Indeed, $f(0,1)$ is equal to 5.
\item $f(x,y)$ is continuous at point $(0,1)$, because 
\begin{align*}
\lim_{(x,y) \rightarrow (0,1)} f(x,y) = f(0,1) = 5
\end{align*}
\item
The partial derivative with respect to $x$ is defined as
\begin{align*}
   \frac{\partial f}{\partial y}(a,b)= \frac{f(0+h,y)-f(0,y)}{h}.
\end{align*}
For this limit to exist, we need the limits as $h$ approaches 0 from the negative and positive directions to exist and to be equal to each other. But
\begin{align*}
\lim_{h \rightarrow 0^+} \frac{f(0+h,y)-f(0,y)}{h} = \lim_{h \rightarrow 0^+} \frac{\big(\big|0+h\big| +5y\big)-\big(\big|0\big| +5y\big)}{h} = \frac{\big|h\big|}{h} = \frac{+ h }{h} = +1,
\end{align*}
and the limit as $h$ approaches zero from the other direction is
\begin{align*}
\lim_{h \rightarrow 0^-} \frac{f(0+h,y)-f(0,y)}{h} = \lim_{h \rightarrow 0^-} \frac{\big(\big|0+h\big| +5y\big)-\big(\big|0\big| +5y\big)}{h} = \frac{\big|h\big|}{h} = \frac{- h }{h} = -1.
\end{align*}
The two limits are not equal, and so the partial derivative with respect to $x$ does not exist on the $y$-axis. Hence, the partial derivative does not exist at the point $(0,1)$. 
\end{itemize}

% ~~~~~~~~~~~~~~~~~~~~~~~~~~~~~~~~~~~~~~~~~~~~~~~~~~~~~~~~~
\item  \Emph{A Second Order Derivative}\\ 
The statement is false. Consider the function $f(x,y)=xy$. Then
\begin{align*}
   \frac{\partial f}{\partial y}\frac{\partial f}{\partial x} = (x)(y) = xy \ne \frac{\partial^2 f}{\partial x \partial y} = 1.
\end{align*}
% ~~~~~~~~~~~~~~~~~~~~~~~~~~~~~~~~~~~~~~~~~~~~~~~~~~~~~~~~~
\item  \Emph{The Rate of Change in the Direction $f_y\MB{i}-f_x\MB{j}$}
\BEN
\item Let $\MB{v}$ be the vector
\begin{align*}
 \MB{v} =  \frac{\partial f}{\partial y}(a,b)\MB{i}-\frac{\partial f}{\partial x}(a,b)\MB{j}.
\end{align*}
The rate of change of $f$ in the direction of the given vector $\mathbf{v}$ at the point $(a,b)$ is given by $\nabla f(a,b) \cdot \mathbf{v}$, which is
\begin{align*}
   \nabla f(a,b) \cdot \Bigg( \frac{\partial f}{\partial y}(a,b)\MB{i}-\frac{\partial f}{\partial x}(a,b)\MB{j}\Bigg)
   &=  \Bigg( \pfx (a,b)\MB{i} - \pfy(a,b)\MB{j}\Bigg) \cdot \Bigg( \pfy (a,b)\MB{i}- \pfx(a,b)\MB{j}\Bigg) \\
   &=  \Bigg( \pfx(a,b) \pfy(a,b)-\pfx(a,b) \pfy(a,b)\Bigg)\\
   &= 0
\end{align*}
\item The vector $\MB{v}$ is perpendicular to the gradient of $f(x,y)$ at $(a,b)$, and is parallel to the level curve at $(a,b)$. 
\EEN
% ~~~~~~~~~~~~~~~~~~~~~~~~~~~~~~~~~~~~~~~~~~~~~~~~~~~~~~~~~
\item  \Emph{Application of the Chain Rule to Model Traffic Flow}
\BEN
\item Using the chain rule, $u=f(k)$ implies that 
\begin{align*}
   \ux =f'(k)\kx ,
\end{align*}
and
\begin{align*}
   \ut = f'(k)\kt .
\end{align*}
Using these relations, the partial differential equation (Equation (\ref{Eq:bresbz})) becomes
\begin{align*}
  u\frac{\partial u}{\partial x}+\frac{\partial u}{\partial t}&=-c^2k^n\frac{\partial k}{\partial x} \\
  uf'(k)\kx+f'(k)\kt &=-c^2k^n\frac{\partial k}{\partial x} \\
  f'(k)\Big(u\kx+\kt\Big) + c^2k^n\frac{\partial k}{\partial x} &=0\\
\end{align*}
Since $u=f(k)$, $\frac{du}{dk}=f'(k)$. Thus,
\begin{align*}
  \frac{du}{dk}\Big(u\kx+\kt\Big) + c^2k^n\frac{\partial k}{\partial x} =0,
\end{align*}
as required.
\item The continuity equation can be rearranged to yield the relation
\begin{align*}
  0&=\kt+\frac{\partial}{\partial x}(ku) \\
  &= +k\ux+u\kx \\
  \kt &= - k\ux-u\kx .
\end{align*}
Substitution into Equation (\ref{xcvbxcvbxcvb}) yields
\begin{align*}
  \frac{du}{dk} \Bigg(u\kx+\kt\Bigg)&=-c^2k^n\frac{\partial k}{\partial x} \\
  \frac{du}{dk} \Bigg(u\kx+\bigg( - k\ux-u\kx \bigg) \Bigg)&=-c^2k^n\frac{\partial k}{\partial x} \\  
  - \frac{du}{dk} \Bigg( k\ux  \Bigg)&=-c^2k^n\frac{\partial k}{\partial x} \\  
  k \frac{du}{dk} \ux &= c^2k^n\frac{\partial k}{\partial x} \\  
  \frac{du}{dk} \ux &= c^2k^{n-1}\frac{\partial k}{\partial x} .
\end{align*}
But 
\begin{align*}
   \ux =f'(k)\kx ,
\end{align*}
so we have
\begin{align*}
  \frac{du}{dk} \ux &= c^2k^{n-1}\kx \\  
  \frac{du}{dk}\bigg( f'(k)\kx \bigg) &= c^2k^{n-1}\frac{\partial k}{\partial x} \\  
\end{align*}
Recall that $u=f(k)$, $\frac{du}{dk}=f'(k)$, which gives us 
\begin{align*}
  \uk \uk \kx &= c^2k^{n-1}\kx 
\end{align*}
Provided that $\kx\ne0$, we have
\begin{align*}
  \uk \uk = \bigg(\uk\bigg)^2  &= c^2k^{n-1} 
\end{align*}
We can take the square root of both sides to obtain
\begin{align*}
  \uk &= \pm ck^{(n-1)/2} 
\end{align*}
\EEN

% ~~~~~~~~~~~~~~~~~~~~~~~~~~~~~~~~~~~~~~~~~~~~~~~~~~~~~~~~~


\EEN % END OF SOLUTIONS

\newpage

