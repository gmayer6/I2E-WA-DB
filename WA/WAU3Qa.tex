\subsection{Partial Derivatives}
\FromC{\From sections 2.2 through 2.4 of \VCT.}

\BEN

% ~~~~~~~~~~~~~~~~~~~~~~~~~~~~~~~~~~~~~~~~~~~~~~~~~~~~~~~~~
\item
\Emph{Calculating First and Second Partial Derivatives}\\ 
Compute all first and second partial derivatives and the gradient
for each of the following functions.
\begin{enumerate}
 \item $f(x,y) = xy\sin(\frac{x}{y})$
 \item $g(x,y) = \dfrac{x+y}{e^{xy}}$
\end{enumerate}

% ~~~~~~~~~~~~~~~~~~~~~~~~~~~~~~~~~~~~~~~~~~~~~~~~~~~~~~~~~
\item
\Emph{Finding Tangent Planes}\\
Find the equations of the planes tangent to the following surfaces
at the specified points.
\begin{enumerate}
 \item $f(x,y) = \frac{x^2}{9} - y^2$ at the point $(2, \frac{1}{3})$.
 \item $x^2 + x + y - z^2 = 7$ at the point $(2, 5, 2)$.
\end{enumerate}
\rednote{Nathaniel: I think the answer for part (a) is $4x-6y-9z=4$, please check}

% ~~~~~~~~~~~~~~~~~~~~~~~~~~~~~~~~~~~~~~~~~~~~~~~~~~~~~~~~~
\item
\Emph{Implicit Differentiation}\\ 
Compute the gradient of the function $z = h(x, y)$, defined implicitly by the
equation $xy = z^{x+y}$.

% ~~~~~~~~~~~~~~~~~~~~~~~~~~~~~~~~~~~~~~~~~~~~~~~~~~~~~~~~~
\item
\Emph{Calculating the Rate of Change of a Function of Two Variables}\\ 

Let $f(x,y) = e^{-(x^2+x+1+y^2-2y)}$.
\begin{enumerate}
 \item In which direction does $f$ increase fastest from the point
  (1,1)?
 \item Compute the rate of change of $f$ in the direction of the
  point $(5,4)$ from the point $(2,1)$.
\end{enumerate}
% ~~~~~~~~~~~~~~~~~~~~~~~~~~~~~~~~~~~~~~~~~~~~~~~~~~~~~~~~~
\item \Emph{Finding a Jacobian}\\ 
Find the Jacobian matrix for the function
$f(x,y) = \begin{bmatrix} \sin(x+y) \\ \ln(xy) \end{bmatrix}$.
% ~~~~~~~~~~~~~~~~~~~~~~~~~~~~~~~~~~~~~~~~~~~~~~~~~~~~~~~~~
\item \Emph{The Two-Dimensional Heat Equation}

A \emph{partial differential equation} is an equation that involves
the partial derivatives of a function.  For example,
$(\frac{\partial f}{\partial x})^2 - \frac{\partial^2 f}{\partial x
\partial y} = 0$ is a partial differential equation that relates
the first- and second-order partial derivatives of an unknown function
$f$.

A function $f$ is said to \emph{satisfy} a partial differential
equation when the equation holds upon substitution of the function's
partial derivatives.  For example, the function $f(x, y) = y$
satisfies the above partial differential equation since
$(\frac{\partial}{\partial x}(y))^2 = 0$ and $\frac{\partial^2}{\partial x \partial y}(y) = \frac{\partial}{\partial x}(1) = 0$.

The one-dimensional heat equation is a partial differential equation
which relates physical properties of a material to the function that
specifies the exact temperature at any point along a fixed-length rod
of that material.  Under a few assumptions, it can be stated in the
following form:
\begin{equation}
\frac{\partial u}{\partial t} = \frac{K_0}{c\rho} \frac{\partial^2 u}{\partial x^2} \label{eqn:heat}
\end{equation}
where $u = u(x,t)$, $x$ is the position along the rod, $t$ is the point in time,
$K_0$ is the thermal conductivity of the material, $c$ is its specific heat,
and $\rho$ is its mass density.

Show that $u(x,t) = e^{-tK_0/(c\rho)}\sin(x)$ is a solution to (\ref{eqn:heat}).
% ~~~~~~~~~~~~~~~~~~~~~~~~~~~~~~~~~~~~~~~~~~~~~~~~~~~~~~~~~
\item  \Emph{Differentiability}\\ 
Construct a function $f(x,y)$ with the following three properties.
\begin{itemize}
\item $f(x,y)$ is defined at the point $(0,1)$.
\item $f(x,y)$ is continuous at the point $(0,1)$.
\item  $\frac{\partial f}{\partial x}$ does not exist at the point $(0,1)$.
\end{itemize}
Show that your function satisfies the above properties.
% ~~~~~~~~~~~~~~~~~~~~~~~~~~~~~~~~~~~~~~~~~~~~~~~~~~~~~~~~~
\item  \Emph{A Second Order Derivative}\\ 
Consider the following statement.
\begin{quote}
If the partial derivatives 
\begin{align*}
   \pfx, \quad \pfy, \quad \pfxy, \quad \pfyx
\end{align*}
exist everywhere on the domain of $f(x,y)$, then 
\begin{align*}
   \frac{\partial f}{\partial y}\frac{\partial f}{\partial x} = \frac{\partial^2 f}{\partial x \partial y} = \frac{\partial^2 f}{\partial y \partial x}.
\end{align*}
\end{quote}
Prove that the above statement is true, or provide a counterexample to show that the statement is false. 
% ~~~~~~~~~~~~~~~~~~~~~~~~~~~~~~~~~~~~~~~~~~~~~~~~~~~~~~~~~
\item  \Emph{The Rate of Change in the Direction $f_y\MB{i}-f_x\MB{j}$}\\ 
Suppose $f(x,y)$ is a function whose gradient is nonzero at the point $(a,b)$. 
\BEN
\item Find the rate of change of $f(x,y)$ at the point $(a,b)$ in the direction of $\MB{v}$, where
\begin{align*}
  \MB{v} = \frac{\partial f}{\partial y}(a,b)\MB{i}-\frac{\partial f}{\partial x}(a,b)\MB{j}.
\end{align*}
\item Provide a geometric interpretation of your answer to part (a). In your interpretation, describe the relationship between your answer to part (a), and the level curves of $f(x,y)$. 
\EEN
% ~~~~~~~~~~~~~~~~~~~~~~~~~~~~~~~~~~~~~~~~~~~~~~~~~~~~~~~~~
\item  \Emph{Application of the Chain Rule to Model Traffic Flow\footnote{This question is based on a problem from Section 13.6 of Calculus for Engineers, by Donald Trim, 3$^{rd}$ Edition.}}\\ 
Traffic flow on a highway can be modeled with the partial differential equation
\begin{align}\label{Eq:bresbz}
  u\frac{\partial u}{\partial x}+\frac{\partial u}{\partial t}=-c^2k^n\frac{\partial k}{\partial x},
\end{align}
where $u$ is the average traffic speed at time $t$ and position $x$ on a straight highway,  $c$ and $n$ are positive constants, and $k$ is the traffic concentration (the number of vehicles per unit area) at position $x$ and time $t$. 
\BEN
\item If we  let $k=g(x,t)$, and model the average speed as a function of the concentration so that $u=f(k)$, use the chain rule to show that
\begin{align}\label{xcvbxcvbxcvb}
  \frac{du}{dk} \Bigg(u\frac{\partial k}{\partial x}+\frac{\partial k}{\partial t}\Bigg)=-c^2k^n\frac{\partial k}{\partial x}.
\end{align}
\item Further assumptions on the continuity of the traffic flow would lead us to the equation
\begin{align}\label{Eq:sefnsefbn}
  \frac{\partial k}{\partial t}+\frac{\partial}{\partial x}(ku)=0.
\end{align}
Use Equations (\ref{Eq:bresbz}) and (\ref{Eq:sefnsefbn}) to derive the ordinary differential equation
\begin{align}\label{myrtntebrebrebrez}
  \frac{du}{dk}= \pm ck^{(n-1)/2}.
\end{align}
\EEN
% ~~~~~~~~~~~~~~~~~~~~~~~~~~~~~~~~~~~~~~~~~~~~~~~~~~~~~~~~~~~~~~~~~~~~~~~~~~~~~~~~~
\item % WAVE EQUATION  
\Emph{The Wave Equation and the Chain Rule}\\ 

If $f=f(t)$ and $g=g(t)$ are functions of one variable whose second derivatives exist, and \\ $u=f(s+ct)+g(s-ct)$, show that 
\begin{align*}
  \utt=c^2\uss,
\end{align*}
where $c$ is a constant. 
% ~~~~~~~~~~~~~~~~~~~~~~~~~~~~~~~~~~~~~~~~~~~~~~~~~~~~~~~~~~~~~~~~~~~~~~~~~~~~~~~~~

\EEN
