\setcounter{subsection}{-1}
\subsection{Partial Derivatives}
\FromC{\From sections 2.2 through 2.4 of \VCT.}

\BEN

% ~~~~~~~~~~~~~~~~~~~~~~~~~~~~~~~~~~~~~~~~~~~~~~~~~~~~~~~~~
\item

Compute all first and second partial derivatives and the gradient
for each of the following functions.
\begin{enumerate}
 \item $f(x,y) = xy\sin(\frac{x}{y})$
 \item $g(x,y) = \dfrac{x+y}{e^{xy}}$
\end{enumerate}

% ~~~~~~~~~~~~~~~~~~~~~~~~~~~~~~~~~~~~~~~~~~~~~~~~~~~~~~~~~
\item

Compute the gradient of the function $z = h(x, y)$ defined implicitly by the
equation $xy = z^{x+y}$.

% ~~~~~~~~~~~~~~~~~~~~~~~~~~~~~~~~~~~~~~~~~~~~~~~~~~~~~~~~~
\item

Find the equations of the planes tangent to the following surfaces
at the specified points.
\begin{enumerate}
 \item $f(x,y) = \frac{x^2}{9} - y^2$ at the point $(2, \frac{1}{3})$.
 \item $x^2 + x + y - z^2 = 7$ at the point $(2, 5, 2)$.
\end{enumerate}

% ~~~~~~~~~~~~~~~~~~~~~~~~~~~~~~~~~~~~~~~~~~~~~~~~~~~~~~~~~
\item

Let $f(x,y) = e^{-(x^2+x+1+y^2-2y)}$.
\begin{enumerate}
 \item In which direction does $f$ increase fastest from the point
  (1,1)?
 \item Compute the rate of change of $f$ in the direction of the
  point $(5,4)$ from the point $(2,1)$.
\end{enumerate}

% ~~~~~~~~~~~~~~~~~~~~~~~~~~~~~~~~~~~~~~~~~~~~~~~~~~~~~~~~~
\item

Find the Jacobian matrix for the function
$f(x,y) = \begin{bmatrix} \sin(x+y) \\ \ln(xy) \end{bmatrix}$.

% ~~~~~~~~~~~~~~~~~~~~~~~~~~~~~~~~~~~~~~~~~~~~~~~~~~~~~~~~~
\item
\Emph{Application: The Heat Equation}

A \emph{partial differential equation} is an equation that involves
the partial derivatives of a function.  For example,
$(\frac{\partial f}{\partial x})^2 - \frac{\partial^2 f}{\partial x
\partial y} = 0$ is a partial differential equation that relates
the first- and second-order partial derivatives of an unknown function
$f$.

A function $f$ is said to \emph{satisfy} a partial differential
equation when the equation holds upon substitution of the function's
partial derivatives.  For example, the function $f(x, y) = y$
satisfies the above partial differential equation since
$(\frac{\partial}{\partial x}(y))^2 = 0$ and $\frac{\partial^2}{\partial x \partial y}(y) = \frac{\partial}{\partial x}(1) = 0$.

The one-dimensional heat equation is a partial differential equation
which relates physical properties of a material to the function that
specifies the exact temperature at any point along a fixed-length rod
of that material.  Under a few assumptions, it can be stated in the
following form:
\begin{equation}
\frac{\partial u}{\partial t} = \frac{K_0}{c\rho} \frac{\partial^2 u}{\partial x^2} \label{eqn:heat}
\end{equation}
where $u = u(x,t)$, $x$ is the position along the rod, $t$ is the point in time,
$K_0$ is the thermal conductivity of the material, $c$ is its specific heat,
and $\rho$ is its mass density.

Show that $u(x,t) = e^{-tK_0/(c\rho)}\sin(x)$ is a solution to (\ref{eqn:heat}).

\EEN

% ----------------------------------------------------------------------------
\newpage
\subsection{Optimization}
\FromC{\From sections 2.5, 2.7 and parts of 4.6 of \VCT.}

\BEN

% ~~~~~~~~~~~~~~~~~~~~~~~~~~~~~~~~~~~~~~~~~~~~~~~~~~~~~~~~~
\item

Use the second derivative test to classify the local extrema of the
following functions:
\begin{enumerate}
 \item $f(x,y) = x^2 + 2x - xy + y^2$
 \item $h(x,y) = \sin(xy)$
\end{enumerate}

% ~~~~~~~~~~~~~~~~~~~~~~~~~~~~~~~~~~~~~~~~~~~~~~~~~~~~~~~~~
\item

Find the extreme values of
\begin{enumerate}
 \item the function
  $f(x,y) = e^{-(x^2+y^2)}$
  along the curve $x = y^2$
 \item the function
  $g(x,y) = e^{x-y^2}$ along the boundary of the ellipse
  $\frac{x^2}{4}+y+y^2=1$
\end{enumerate}

% ~~~~~~~~~~~~~~~~~~~~~~~~~~~~~~~~~~~~~~~~~~~~~~~~~~~~~~~~~
\item

Consider the function $f(x,y,z) = \begin{bmatrix}
                                   x^2 + y^2 \\
                                   y^2 + z^2 \\
                                   z^2 + x^2
                                  \end{bmatrix}$.
Find its divergence $\nabla \cdot f$ and curl $\nabla \times f$.

% ~~~~~~~~~~~~~~~~~~~~~~~~~~~~~~~~~~~~~~~~~~~~~~~~~~~~~~~~~
\item

Find the point on the sphere $x^2 + y^2 + z^2 = 1$ that is closest
to the plane $x + 2y + 2z = 5$. (Hint: using some geometric intuition
is is possible to complete this problem without using the method of
Lagrange multipliers.)

% ~~~~~~~~~~~~~~~~~~~~~~~~~~~~~~~~~~~~~~~~~~~~~~~~~~~~~~~~~
\item
\Emph{Application: Cost Optimization}

Suppose you are given a budget of \$500 to build a large glass
triangular prism.  Each rectangular side must have equal dimensions
and the two triangular sides must also have equal dimensions.  You
can purchase glass for the rectangular sides at a cost of $\$8/ft^2$
and for the triangular sides at a cost of $\$10/ft^2$.

What are the dimensions of the prism of largest volume that you can build?

% ~~~~~~~~~~~~~~~~~~~~~~~~~~~~~~~~~~~~~~~~~~~~~~~~~~~~~~~~~
\item
\Emph{Application: Heat Optimization}

Suppose that the temperature in a space is given by the function
$T(x,y,z) = 200xyz^2$.  Find the hottest point on the unit sphere.

(This question taken from section 4.5 of {\it Vector Calculus} by Susan Colley)

\EEN
