
% ----------------------------------------------------------------------------
\newpage
\subsection{Optimization}
\FromC{\From sections 2.5, 2.7 of \VCT.}

\BEN

% ~~~~~~~~~~~~~~~~~~~~~~~~~~~~~~~~~~~~~~~~~~~~~~~~~~~~~~~~~
\item % PUDDLE
The topography of a surface is described by the equation $z=x^2-2xy+y^2$. When it rains on this surface, a puddle will form. Where does the puddle form? 
% ~~~~~~~~~~~~~~~~~~~~~~~~~~~~~~~~~~~~~~~~~~~~~~~~~~~~~~~~~
\item

Use the second derivative test to classify the local extrema of the
following functions:
\begin{enumerate}
 \item $f(x,y) = x^2 + 2x - xy + y^2$
 \item $h(x,y) = \sin(xy)$
\end{enumerate}

% ~~~~~~~~~~~~~~~~~~~~~~~~~~~~~~~~~~~~~~~~~~~~~~~~~~~~~~~~~
\item

Find the extreme values of
\begin{enumerate}
 \item the function
  $f(x,y) = e^{-(x^2+y^2)}$
  along the curve $x = y^2$
 \item the function
  $g(x,y) = e^{x-y^2}$ along the boundary of the ellipse
  $\frac{x^2}{4}+y+y^2=1$
\end{enumerate}

% ~~~~~~~~~~~~~~~~~~~~~~~~~~~~~~~~~~~~~~~~~~~~~~~~~~~~~~~~~
\item

Find the point on the sphere $x^2 + y^2 + z^2 = 1$ that is closest
to the plane $x + 2y + 2z = 5$. (Hint: using some geometric intuition
is is possible to complete this problem without using the method of
Lagrange multipliers.)

% ~~~~~~~~~~~~~~~~~~~~~~~~~~~~~~~~~~~~~~~~~~~~~~~~~~~~~~~~~
\item
\Emph{Cost Optimization}

Suppose you are given a budget of \$500 to build a large glass
triangular prism.  Each rectangular side must have equal dimensions
and the two triangular sides must also have equal dimensions.  You
can purchase glass for the rectangular sides at a cost of $\$8/ft^2$
and for the triangular sides at a cost of $\$10/ft^2$.

What are the dimensions of the prism of largest volume that you can build?

% ~~~~~~~~~~~~~~~~~~~~~~~~~~~~~~~~~~~~~~~~~~~~~~~~~~~~~~~~~
\item
\Emph{Heat Optimization\footnote{This question is similar to a question from section 4.5 of {\it Vector Calculus} by Susan Colley}}

Suppose that the temperature in a space is given by the function
$T(x,y,z) = 200xyz^2$.  Find the hottest point on the unit sphere.

% ~~~~~~~~~~~~~~~~~~~~~~~~~~~~~~~~~~~~~~~~~~~~~~~~~~~~~~~~~
\item % CRITICAL POINTS
\Emph{Critical Points of a Function of Two Variables}\\
If a continuous function of one variable, $f(x)$, has only two distinct local minima, then it \textit{must} have at least one local maxima. Let's explore this concept for the two-dimensional case. 
\BEN
\item Suppose that a continuous and differentiable function $f(x,y)$ has exactly three critical points at $(-a,-a),(0,0)$, and at $(a,a)$, where $a>0$. Suppose also that $f(x,y)$ has local minima at the points $(-a,-a)$ and $(a,a)$. Is the point $(0,0)$ a local maximum? Explain why or why not.
\item Consider the function $f(x,y)=x^4+y^4-4xy-1$. 
\BEN
\item Show that the only critical points of $f(x,y)$ are located at $(-1,-1),(0,0),(1,1)$. 
\item Classify the critical points of $f(x,y)$, and revisit your answer in part (a) to make sure that it is correct. 
\EEN
\EEN
% ~~~~~~~~~~~~~~~~~~~~~~~~~~~~~~~~~~~~~~~~~~~~~~~~~~~~~~~~~
\item
\rednote{This question will be moved to a later assignment}\\
Consider the function $f(x,y,z) = \begin{bmatrix}
                                   x^2 + y^2 \\
                                   y^2 + z^2 \\
                                   z^2 + x^2
                                  \end{bmatrix}$.
Find its divergence $\nabla \cdot f$ and curl $\nabla \times f$.

% ~~~~~~~~~~~~~~~~~~~~~~~~~~~~~~~~~~~~~~~~~~~~~~~~~~~~~~~~~

\EEN
