\subsection{Change of Variables in Multiple Integrals}

\BEN
% ~~~~~~~~~~~~~~~~~~~~~~~~~~~~~~~~~~~~~~~~~~~~~~~~~~~~~~~~~~~~~~~~~~~~~~~~~~~~~~~~~
\item % GENERAL LINEAR TRANSFORMATION
\Emph{Linear Transformations} \\
\BEN
\item Substituting $v = v_0$ into the linear transformation yields the two equations
\begin{align*}
  x &= c_1u + c_2v_0 \\ 
  y &= d_1u + d_2v_0
\end{align*}
To find the equation of the line in the $xy$-plane, we need to eliminate $u$. There are many ways to do this, but let's multiply the first equation by $d_1$ and the second by $c_1$.
\begin{align*}
  d_1x &= c_1d_1u + c_2d_1v_0 \\ 
  c_1y &= c_1d_1u + d_2c_1v_0
\end{align*}
Subtracting these equations yields
\begin{align*}
  d_1x -  c_1y &=  (c_2d_1 -  d_2c_1)v_0 
\end{align*}
A simple rearrangement gives us
\begin{align*}
  c_1y &= d_1x - (c_2d_1 -  d_2c_1)v_0 .
\end{align*}
Provided that $c_1$ is not zero, we could write this in the form
\begin{align*}
  y &= \frac{d_1}{c_1}x - \frac{c_2d_1 -  d_2c_1}{c_1}v_0 .
\end{align*}
% ~  ~  ~  ~  ~  ~  ~  ~  ~  ~  ~  ~  ~  ~  ~  ~  ~  ~  ~  ~  ~  ~  ~  ~  ~  ~  ~  ~
\item 
Substituting $x=x_0$ into $x =c_1u + c_2v$ gives us 
\begin{align*}
  x_0 &= c_1u + c_2v,
\end{align*}
Provided that $c_2$ is not zero, 
This is the is mapped into the $uv$-plane REWORD
\EEN

% ~~~~~~~~~~~~~~~~~~~~~~~~~~~~~~~~~~~~~~~~~~~~~~~~~~~~~~~~~~~~~~~~~~~~~~~~~~~~~~~~~
\item % STRAIGHTFORWARD CHANGE OF VARIABLES
\Emph{Double Integral Over a Parallelogram} \\
The integral can be written as
\begin{align*}
  \iint\limits_R \big(x^2 - y^2\big) dxdy =  \iint\limits_R (x - y)(x+y) dxdy.
\end{align*}
Recall that $R$ is the region bounded by
\begin{align*}
  x+y = 0, \quad x+y = 1, \quad x-y=0, \quad x-y=1.
\end{align*}
The appearance of the terms $(x+y)$ and $(x - y)$ in the integrand and in the lines that bound $R$ suggests the transformation 
\begin{align}
  u & = x+y  \label{zcvzcxvvad}\\
  v &= x - y \label{ntrsraaewvbernst}.
\end{align}
In order to compute the Jacobian, we need explicit expressions for $u$ and $v$. If we add  equations \ref{zcvzcxvvad} and \ref{ntrsraaewvbernst} we find that 
\begin{align*}   x = \frac{u+v}{2}   \end{align*}
And if we subtract equations \ref{zcvzcxvvad} and \ref{ntrsraaewvbernst} we find that
\begin{align*}   y = \frac{u-v}{2}   \end{align*}
The Jacobian becomes
\begin{align*}   J &=  
  \begin{vmatrix}
   \pxu &  \pxv \\ \\
   \pyu & \pyv \\
  \end{vmatrix}
  =     \begin{vmatrix}
 \frac{1}{2} &  \frac{1}{2} \\ \\
   \frac{1}{2} & -\frac{1}{2} \\
  \end{vmatrix}
  = - \frac{1}{4}- \frac{1}{4} = - \frac{1}{2}.
 \end{align*}
We also need to find the limits of  integration in the transformed integral. Using equations \ref{zcvzcxvvad} and \ref{ntrsraaewvbernst} the four lines bounding $R$ in the $xy$-plane become 
\begin{align*}
  u = 0, \quad u = 1, \quad v=0, \quad v=1.
\end{align*}
The double integral therefore becomes
\begin{align*}
  \iint\limits_R \big(x^2 - y^2\big) dxdy 
  &=  \iint\limits_R (x - y)(x+y) dxdy \\
  &=    \mathop{\int_0^1 \! \int_0^1} uv \Bigg|-\frac{1}{2} \Bigg| dudv \\
  &=     \frac{1}{2}\mathop{\int_0^1 \! \int_0^1} (uv) dudv \\
  &=     \frac{1}{2}\int_0^1 \frac{v}{2} dv \\
  &=   \frac{1}{8}.
\end{align*}

% ~~~~~~~~~~~~~~~~~~~~~~~~~~~~~~~~~~~~~~~~~~~~~~~~~~~~~~~~~~~~~~~~~~~~~~~~~~~~~~~~~
\item % 
\Emph{Triple Integral Bounded by an Ellipse} \\
The Jacobian is
\begin{align*}   J &=  
  \begin{vmatrix}
   \pxu &  \pxv \\ \\
   \pyu & \pyv \\
  \end{vmatrix}
  =     \begin{vmatrix}
   3 & 0 \\ \\
   0 & 2 \\
  \end{vmatrix}
  = 6 .
 \end{align*}
We also need to find the limits of integration in the transformed integral. The region $R$ is bounded by the ellipse, $4x^2 + 9y^2 = 36$, which becomes the region bounded by the circle $u^2 + v^2 = 1$. Therefore
\begin{align*}
  \iint\limits_R x^2 \ dxdy &= \iint\limits_{u^2 + v^2 \le 1} (9u^2) 6dudv =  54 \iint\limits_{u^2 + v^2 \le 1} (u^2) dudv
\end{align*}
Switching to polar coordinates, 
\begin{align*}
  u = r\cos\theta, \quad v = r\sin\theta, \quad J = r
\end{align*}
our double integral becomes
\begin{align*}
  54 \iint\limits_{u^2 + v^2 \le 1} (u^2) dudv 
  & = 54 \mathop{\int_0^{2\pi} \!\! \int_0^1} r^3\cos^2\theta drd\theta  \\
  & = 54 \Bigg( \int_0^{2\pi} \cos^2\theta d\theta \Bigg)\Bigg( \int_0^1 r^3 dr \Bigg) \\
  & = 54 \Bigg( \int_0^{2\pi} \frac{1}{2} \big(1+\cos(2\theta) \big) d\theta \Bigg) \Bigg( \frac{1}{4} \Bigg) \\
  & =\frac{27}{4} \Bigg( \int_0^{2\pi}  \big(1+\cos(2\theta) \big) d\theta \Bigg) \\
  & =\frac{27}{4} \big(\theta+\frac{1}{2}\sin(2\theta) \big|_0^{2\pi}  \\
  & =\frac{27}{4} \big(2\pi\big)  \\
  & =\frac{27\pi}{2} 
\end{align*}
% ~~~~~~~~~~~~~~~~~~~~~~~~~~~~~~~~~~~~~~~~~~~~~~~~~~~~~~~~~~~~~~~~~~~~~~~~~~~~~~~~~
\item % SIMPLE CYLINDRICAL WITH CONE
In cylindrical coordinates, the region is described by 
\begin{align*}
  V = \{(r,\theta,z) \ | \ 0 \le r \le 1, 0 \le \theta \le 2\pi, 0 \le z \le 3r \}.
\end{align*}
Our integral becomes 
\begin{align*}
  \iiint\limits_V (r\sin\theta)^2 rdzdrd\theta 
  &=    \mathop{\int_0^{2\pi} \!\! \int_0^1  \!\! \int_0^{3r}} r^3\sin^2\theta dzdrd\theta  \\
  &=   3 \mathop{\int_0^{2\pi} \!\! \int_0^1} r^4\sin^2\theta drd\theta  \\
  &=   \frac{3}{5} \int_0^{2\pi} \sin^2\theta d\theta  \\
  &=   \frac{3}{10} \int_0^{2\pi} \big(1 - \cos(2\theta)\big) d\theta  \\
  &=   \frac{3}{10} \big(\theta - \frac{1}{2}\sin(2\theta)\big) \Big|_0^{2\pi} \\
  &=   \frac{3\pi}{5}
\end{align*}
% ~~~~~~~~~~~~~~~~~~~~~~~~~~~~~~~~~~~~~~~~~~~~~~~~~~~~~~~~~~~~~~~~~~~~~~~~~~~~~~~~~
\item % TRIPLE INTEGRAL IN CYLINDRICAL COORDINATES
In cylindrical coordinates,  $V$ is the region bounded by
\begin{align*}
  0 \le\ &r \le 2 \\
  0 \le\ &\theta \le \frac{\pi}{2}\\
  0 \le\ &z \le \sqrt{4 - r^2}
\end{align*}
The triple integral becomes
\begin{align*}
  \iiint\limits_V dxdydz
  &=    \mathop{\int_0^{\pi/2} \!\! \int_0^2  \!\! \int_0^{\sqrt{4 - r^2}}} r dzdrd\theta  \\
  &=    \mathop{\int_0^{\pi/2} \!\! \int_0^2 } r\sqrt{4 - r^2}drd\theta  \\
  &=   \frac{-1}{3} \int_0^{\pi/2} (4 - r^2)^{3/2} \Big|_0^2 d\theta  \\
  &=   \frac{-1}{3} \int_0^{\pi/2} (0 - 8) d\theta  \\
  &=   \frac{4\pi}{3}
\end{align*}
% ~~~~~~~~~~~~~~~~~~~~~~~~~~~~~~~~~~~~~~~~~~~~~~~~~~~~~~~~~~~~~~~~~~~~~~~~~~~~~~~~~
\item % TRIPLE INTEGRAL IN SPHERICAL COORDINATES
\Emph{Triple Integral In Spherical Coordinates} \\
The solid is a section of a sphere with radius 1, centered at the origin. The section is the part of the sphere that lies above the plane $z=0$, and between the planes $y=0$, and $y=x$. 
\begin{align*}
  \mathop{\int_0^{\pi/4} \!\! \int_{0}^{\pi/2} \!\! \int_0^1 } ( \rho^2 \sin\phi ) \ d\rho\  d\phi\  d\theta
  &=   \mathop{\int_0^{\pi/4} \!\! \int_{0}^{\pi/2} }  \frac{ \sin\phi}{3}  d\phi\  d\theta\\
  &=   \int_0^{\pi/4} \frac{-(0-1)}{3}   d\theta \\
  &=  \pi/12
\end{align*}
% ~~~~~~~~~~~~~~~~~~~~~~~~~~~~~~~~~~~~~~~~~~~~~~~~~~~~~~~~~~~~~~~~~~~~~~~~~~~~~~~~~
\item % SIMPLIFYING
\Emph{Simplifying a Double Integral as a Product of Two Single Integrals} 
\BEN
\item
Substituting $f(x,y) = g(x)h(y)$ into the double integral yields
\begin{align*}
   \iint\limits_{R} f(x,y) dA 
   =  \int_a^b  \int_c^d g(x) h(y) dydx 
   =  \int_a^b \Bigg( \int_c^d g(x) h(y) dy \Bigg) dx .
\end{align*}
The function $g(x)$ is a constant in the inner integral, so
\begin{align*}
   \iint\limits_{R} f(x,y) dA 
   &=  \int_a^b  \Bigg( g(x)\int_c^d h(y) dy \Bigg) dx \\
   &=  \int_a^b   g(x) \Bigg(\int_c^d h(y) dy \Bigg) dx .
\end{align*}
The term in the brackets is a constant, and so 
\begin{align*}
   \iint\limits_{R} f(x,y) dA 
   &=  \int_a^b   g(x) \Bigg(\int_c^d h(y) dy \Bigg) dx \\
   &=  \int_a^b   g(x) dx \int_c^d h(y) dy.
\end{align*}
This is the result we were asked to prove.
\item Let the left-hand-side be equal to $I$,
\begin{align*}
   I=\int_{0}^{\infty} e^{-x^2} dx.
\end{align*}
$I$ is also equal to
\begin{align*}
   I=\int_{0}^{\infty} e^{-y^2} dy.
\end{align*}
Therefore,
\begin{align*}
  I^2 &= \Bigg(\int_{0}^{\infty} e^{-x^2} dx\Bigg)\Bigg(\int_{0}^{\infty} e^{-y^2} dy\Bigg).
\end{align*}
Our result from part (a) allows us to write this as
\begin{align*}
  I^2 &= \int_{0}^{\infty} \int_{0}^{\infty} e^{-x^2-y^2} dxdy.
\end{align*}
Switching to polar coordinates,
\begin{align*}
  I^2 &= \int_{0}^{\infty} \int_{0}^{\pi/2} r e^{-r^2}  \ d\theta dr \\
  &= \frac{\pi}{2} \int_{0}^{\infty}  r e^{-r^2}  \ dr \\
  &= \frac{\pi}{2} \Big(-\frac{1}{2}e^{-r^2}\Big)\Big|_{0}^{\infty}  \\
  &= -\frac{\pi}{4} \Big( e^{-r^2}\Big)\Big|_{0}^{\infty}    \\
  &= -\frac{\pi}{4} \Big( 0 - e^0 \Big) \\
  &= \frac{\pi}{4} 
\end{align*}
Taking the square root of both sides of this yields the desired result,
\begin{align*}
  I &= \frac{\sqrt{\pi}}{2} .
\end{align*}
\EEN
% ~~~~~~~~~~~~~~~~~~~~~~~~~~~~~~~~~~~~~~~~~~~~~~~~~~~~~~~~~~~~~~~~~~~~~~~~~~~~~~~~~
\EEN % END OF SUBSECTION
%% s~s~s~s~s~s~s~s~s~s~s~s~s~s~s~s~s~s~s~s~s~s~s~s~s~s~s~s~s~s~s~s~s~s~s~s~s~s~s~s~s~s~s~s
%% s~s~s~s~s~s~s~s~s~s~s~s~s~s~s~s~s~s~s~s~s~s~s~s~s~s~s~s~s~s~s~s~s~s~s~s~s~s~s~s~s~s~s~s
%\subsection{Application: Center of Mass}
%\BEN
%% ~~~~~~~~~~~~~~~~~~~~~~~~~~~~~~~~~~~~~~~~~~~~~~~~~~~~~~~~~~~~~~~~~~~~~~~~~~~~~~~~~
%\item % CENTER OF MASS OF 2D TRIANGULAR PLATE
%\Emph{Center of Mass of a 2D Triangular Plate} \\
%The total mass of the plate is
%\begin{align*}
%  M 
%  &= \iint\limits_R \delta(x,y) dA \\
%  &=    \mathop{\int_0^1 \!\! \int_{2x}^{4x} } xy \ dydx \\
%  &=  \frac{1}{2}  \int_0^1 x(16x^2 - 4x^2) \ dx \\
%  &= 6  \int_0^1 x^3 \ dx \\
%  &=   \frac{3}{2}.
%\end{align*}
%
%The $x$-coordinate of the center of mass is
%\begin{align*}
%  \bar{x} = \frac{M_y}{M} 
%  &= \frac{2}{3} \iint\limits_R x \delta(x,y) dA \\
%  &=    \mathop{\int_0^1 \!\! \int_{2x}^{4x} } x^2y \ dydx \\
%  &=  \frac{1}{2}  \int_0^1 x^2(16x^2 - 4x^2) \ dx \\
%  &= 6  \int_0^1 x^4 \ dx \\
%  &=   \frac{6}{5}.
%\end{align*}
%
%The $y$-coordinate of the center of mass is
%\begin{align*}
%  \bar{y} = \frac{M_x}{M} 
%  &= \frac{2}{3} \iint\limits_R y \delta(x,y) dA \\
%  &=    \mathop{\int_0^1 \!\! \int_{2x}^{4x} } xy^2 \ dydx \\
%  &=  \frac{1}{3}  \int_0^1 x(64x^3 - 8x^3) \ dx \\
%  &=  \frac{56}{3}  \int_0^1 x^4 \ dx \\
%  &=   \frac{56}{15}.
%\end{align*}
%Thus, the coordinates of the center of mass are $(6/5,56/15)$.
%% ~~~~~~~~~~~~~~~~~~~~~~~~~~~~~~~~~~~~~~~~~~~~~~~~~~~~~~~~~~~~~~~~~~~~~~~~~~~~~~~~~
%\item % CENTER OF MASS OF 2D RADIAL DENSITY FUNCTION
%\Emph{Center of Mass of a 2D Plate, Radial Density Function} \\
%We could express the coordinates of the center of mass using either Cartesian or polar coordinates. Using Cartesian coordinates, the density, $\delta$, at point $(x,y)$, is given by
%\begin{align*}
%  \delta(x,y) = L(x,y) = a - \sqrt{x^2+y^2}.
%\end{align*}
%\BEN
%\item
%The $x$-coordinate of the center of mass is
%\begin{align*}
%  \bar{x} = \frac{M_y}{M} 
%  = \frac{1}{M} \iint\limits_R x \delta(x,y) dA 
%  = \frac{1}{M}  \mathop{\int_{-a}^{a} \!\! \int_{0}^{\sqrt{x^2+y^2}}} x\Big( a - \sqrt{x^2+y^2}\Big) \ dydx 
%\end{align*}
%\item
%The $y$-coordinate of the center of mass is
%\begin{align*}
%  \bar{y} = \frac{M_x}{M} 
%  = \frac{1}{M} \iint\limits_R y \delta(x,y) dA 
%  = \frac{1}{M}  \mathop{\int_{-a}^{a} \!\! \int_{0}^{\sqrt{x^2+y^2}}} y\Big( a - \sqrt{x^2+y^2}\Big) \ dydx 
%\end{align*}
%\item The center of mass is located on the $y$-axis. In other words, $\bar{x}=0$. This is because of symmetry: the integrand of $\bar{x}$ is odd in $x$, the integral with respect to $x$ is calculated about an interval that is symmetric about the $y$-axis, and so the integral with respect to $x$ is zero. 
%\EEN
%Note that we could just as easily set up the above integrals using polar coordinates. The 2D circular plate, in polar coordinates, is bounded by 
%\begin{align*}
%  0 \le r \le a, \quad 0 \le \theta \le \pi, \quad a > 0.
%\end{align*}
%The coordinates for the center of mass are
%\begin{align*}
%  \bar{x} &= \frac{M_y}{M} 
%  = \frac{1}{M} \iint\limits_R x \delta(x,y) dA 
%  = \frac{1}{M}  \mathop{\int_{0}^{\pi} \!\! \int_{0}^r} r\cos(\theta) ( a - r ) \ rdrd\theta \\
%  \bar{y} &= \frac{M_x}{M} 
%  = \frac{1}{M} \iint\limits_R y \delta(x,y) dA 
%  = \frac{1}{M}  \mathop{\int_{0}^{\pi} \!\! \int_{0}^r} r\sin(\theta) ( a -r ) \ rdrd\theta
%\end{align*}
% 
%% ~~~~~~~~~~~~~~~~~~~~~~~~~~~~~~~~~~~~~~~~~~~~~~~~~~~~~~~~~~~~~~~~~~~~~~~~~~~~~~~~~
%\item % EVEN AND ODD
%The integral can be written as
%\begin{align*}
%  \mathop{\int_0^{\pi} \!\! \int_{-1}^1} e^{x^2 + y^2}\sin(y) dydx 
%  &= \mathop{\int_0^{\pi} \!\! \int_{-1}^1} e^{x^2}e^{y^2}\sin(y) dydx  = \Bigg( \int_0^{\pi} e^{x^2} dx \Bigg)\Bigg( \int_{-1}^1 e^{y^2}\sin(y)dy \Bigg)  
%\end{align*}
%The second term is the integral of an odd function over a symmetrical interval, and so is equal to zero. Therefore,
%\begin{align*}
%  \mathop{\int_0^{\pi} \!\! \int_{-1}^1} e^{x^2 + y^2}\sin(y) dydx =0.
%\end{align*}
%
%% ~~~~~~~~~~~~~~~~~~~~~~~~~~~~~~~~~~~~~~~~~~~~~~~~~~~~~~~~~~~~~~~~~~~~~~~~~~~~~~~~~
%
%
%
%% ~~~~~~~~~~~~~~~~~~~~~~~~~~~~~~~~~~~~~~~~~~~~~~~~~~~~~~~~~~~~~~~~~~~~~~~~~~~~~~~~~
%\item % CHANGING ORDER OF INTEGRATION
%\rednote{This is a good midterm question}\\
%\textbf{Changing the Order of Integration}\\
%Noting that the limits of integration are not dependent on $y$, the $z$ and the $y$ integrals can be interchanged:
%\begin{align*}
%  V &= \mathop{\int_{0}^{1} \!\! \int_{-\sqrt{x}}^{\sqrt{x}} \! \int_0^{1-x} } dz\ dy\ dx \\
%   &= \mathop{\int_{0}^{1} \Bigg( \int_{-\sqrt{x}}^{\sqrt{x}} dy \Bigg) \Bigg( \int_0^{1-x} } dz\Bigg) dx \\
%   &= \mathop{\int_{0}^{1}  \Bigg( \int_0^{1-x} } dz\Bigg) \Bigg( \int_{-\sqrt{x}}^{\sqrt{x}} dy \Bigg)dx \\
%   &= \mathop{\int_{0}^{1} \!\! \int_0^{1-x} \!\! \int_{-\sqrt{x}}^{\sqrt{x}}} dy\ dz\ dx 
%\end{align*}
%We can also integrate $y$ last. We are given that the solid can be expressed as the region 
%\begin{align*}
%  S = \{(x,y,z)\ |\  0 \le x \le 1, -\sqrt{x} \le y \le \sqrt{x}, 0 \le z \le 1-x \}.
%\end{align*}
%We can re-write this as
%\begin{align*}
%  S = \{(x,y,z)\ |\  y^2 \le x \le 1, -1 \le y \le 1, 0 \le z \le 1-x \}.
%\end{align*}
%This gives us the integral
%\begin{align*}
%  V &= \mathop{\int_{-1}^{1} \! \int_{y^2}^{1} \! \int_0^{1-x} } dz\ dx\ dy
%\end{align*}
%We can also define the region in this way \rednote{wrong}
%\begin{align*}
%  S = \{(x,y,z)\ |\  0 \le x \le 1-z, -1 \le y \le 1, 0 \le z \le 1 \}.
%\end{align*}
%This gives us the integral
%\begin{align*}
%  V &= \mathop{\int_{-1}^{1} \! \int_{0}^{1} \! \int_0^{1-z} } dx\ dz\ dy
%\end{align*}
%
%\EEN % END OF SOLUTIONS
