\subsection{Surface Integrals}
\FromC{\From Sections 4.4 and 4.5 of \VCT.}

\BEN
%~~~~~~~~~~~~~~~~~~~~~~~~~~~~~~~~~~~~~~~~~~~~~~~~~~~~~~~~~~~~~~~~~~~~~~~~~~~~~~

\item
\textbf{Surface area} \\
Prove, using surface integration (in rectangular coordinates), that the surface
area of a cone with base radius $r$ and height $h$ is $\pi r \sqrt{h^2 + r^2}$.

\item
\textbf{Computing mass from density} \\
Compute, using surface integration, the mass of the top half of a hollow
spherical shell of radius $R$ whose density at the point $(x,y,z)$, assuming
the center of the shell lies at the origin, is given by the function $d(x,y,z)
= x + y + z^2$. You may assume that the shell is infinitesimally thin.

\item \label{flux-surface}
\textbf{Flux through a surface} \\
Consider the vector field
$\mathbf{f}(x,y,z) = x\mathbf{i} + y\mathbf{j} + z\mathbf{k}$
and the surface $S$ parameterized by
$\mathbf{r}(s,t) = (s^2 + t^2)\mathbf{i} + s\mathbf{j} + t\mathbf{k}$
for $-2 \leq s \leq 2$, $-3 \leq t \leq 3$.  Compute the flux of $\mathbf{f}$
through $S$.

\item
\textbf{Flux and the Divergence Theorem} \\
Consider the vector field $\mathbf{f}(x,y,z) = 3xz^2\mathbf{i} +
xe^{y}\mathbf{j} + \frac{1}{z}\mathbf{k}$.  Use the Divergence Theorem to
compute the flux of $f$ through the box $S$ bounded by $0 \leq x \leq 2$, $-2
\leq y \leq 4$, $-3 \leq z \leq -1$.

\item
\textbf{Application: Work done by wind} \\
An airplane's path from point $A$ to point $B$ can be parameterized by the
equation $\mathbf{r}(t) = t\mathbf{i} + \cos^2(t)\mathbf{j} +
\sin(t)\mathbf{k}$ for $0 \leq t \leq \pi$.  The wind exerts a force on the
airplance that can be described by the vector field $f(x,y,z) = (3y +
3x^2z)\mathbf{i} + (3x + z^2)\mathbf{j} + (2yz + x^3)\mathbf{k}$.
\BEN
\item Find a potential for $f$.
\item Compute the work done by the wind on the airplane during its flight.
\EEN

\item
\textbf{Stokes' Theorem} \\
Recall the surface $S$ from (\ref{flux-surface}) and let $\mathbf{f}(x,y,z) =
2xy\mathbf{i} + y^2\mathbf{j} + (x+z)\mathbf{k}$.  Use Stokes' Theorem to
compute the line integral $\int_{\partial S} \! \mathbf{f} \cdot d\mathbf{r}
\,$.


%~~~~~~~~~~~~~~~~~~~~~~~~~~~~~~~~~~~~~~~~~~~~~~~~~~~~~~~~~~~~~~~~~~~~~~~~~~~~~~
\EEN
