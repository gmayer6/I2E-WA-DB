\documentclass{article}
% ~ ~ ~ ~ ~ ~ ~ ~ ~ ~ ~ ~ ~ ~ ~ ~ ~ ~ ~ ~ ~ ~ ~ ~ ~ ~ ~ ~ ~ ~ ~ ~ ~ ~ ~ ~ ~ ~ ~ ~ ~ ~ ~ ~ ~ ~ ~ ~

\usepackage[utf8]{inputenc}
\usepackage{amsmath}
\usepackage{amssymb} % for \mathbb
\usepackage{graphicx} % for figures
\usepackage{color}
\usepackage[usenames,dvipsnames]{xcolor}
\usepackage{hyperref} % for hyperlinks
\usepackage{float} % for figures
% ~ ~ ~ ~ ~ ~ ~ ~ ~ ~ ~ ~ ~ ~ ~ ~ ~ ~ ~ ~ ~ ~ ~ ~ ~ ~ ~ ~ ~ ~ ~ ~ ~ ~ ~ ~ ~ ~ ~ ~ ~ ~ ~ ~ ~ ~ ~ ~
% CUSTOM COLOUR
\definecolor{DGrey}{rgb}{0.1,0.1,0.1} % Dark Grey
% ~ ~ ~ ~ ~ ~ ~ ~ ~ ~ ~ ~ ~ ~ ~ ~ ~ ~ ~ ~ ~ ~ ~ ~ ~ ~ ~ ~ ~ ~ ~ ~ ~ ~ ~ ~ ~ ~ ~ ~ ~ ~ ~ ~ ~ ~ ~ ~
\title{ Introduction to Engineering Written Assignment Questions}
\date{January 2013}
\author{Greg Mayer and Daniel Connelly}
% ~ ~ ~ ~ ~ ~ ~ ~ ~ ~ ~ ~ ~ ~ ~ ~ ~ ~ ~ ~ ~ ~ ~ ~ ~ ~ ~ ~ ~ ~ ~ ~ ~ ~ ~ ~ ~ ~ ~ ~ ~ ~ ~ ~ ~ ~ ~ ~
% HEADER/FOOTER
\usepackage{fancyheadings}
\pagestyle{myheadings} % set headings to be user defined
\fancyhead{} % To create custom header, clear default layout
\renewcommand{\subsectionmark}[1]{\markright{{\color{DGrey}\thesubsection} \ {\color{DGrey}#1}}}
\fancyhead[LE,LO]{\subsectionmark} % To create custom header, clear default layout

% ~ ~ ~ ~ ~ ~ ~ ~ ~ ~ ~ ~ ~ ~ ~ ~ ~ ~ ~ ~ ~ ~ ~ ~ ~ ~ ~ ~ ~ ~ ~ ~ ~ ~ ~ ~ ~ ~ ~ ~ ~ ~ ~ ~ ~ ~ ~ ~
% ENUMERATION

\usepackage{enumitem}   % so that question numbers can be formatted 
\setenumerate[1]{label=\thesubsection.\arabic*.} % enumerate environment: add section numbers to items
% ~ ~ ~ ~ ~ ~ ~ ~ ~ ~ ~ ~ ~ ~ ~ ~ ~ ~ ~ ~ ~ ~ ~ ~ ~ ~ ~ ~ ~ ~ ~ ~ ~ ~ ~ ~ ~ ~ ~ ~ ~ ~ ~ ~ ~ ~ ~ ~
% MARGINS
\usepackage{anysize}
\marginsize{2.5cm}{2.5cm}{1cm}{1cm}
% ~ ~ ~ ~ ~ ~ ~ ~ ~ ~ ~ ~ ~ ~ ~ ~ ~ ~ ~ ~ ~ ~ ~ ~ ~ ~ ~ ~ ~ ~ ~ ~ ~ ~ ~ ~ ~ ~ ~ ~ ~ ~ ~ ~ ~ ~ ~ ~
% PAGE NUMBERING
\pagenumbering{arabic}
% ~ ~ ~ ~ ~ ~ ~ ~ ~ ~ ~ ~ ~ ~ ~ ~ ~ ~ ~ ~ ~ ~ ~ ~ ~ ~ ~ ~ ~ ~ ~ ~ ~ ~ ~ ~ ~ ~ ~ ~ ~ ~ ~ ~ ~ ~ ~ ~
% Custom Commands
\newcommand{\Emph}[1]{\textbf{#1}} % Emphasize
\newcommand{\R}{\mathbb{R}} 
\newcommand{\BM}{\begin{bmatrix}} % Begin Matrix
\newcommand{\EM}{\end{bmatrix}} % End Matrix
\newcommand{\BEN}{\begin{enumerate}[leftmargin=1.1cm]}% Begin ENumerate
\newcommand{\EEN}{\end{enumerate}} % End ENumerate
\newcommand{\MB}{\mathbf} % Math Bold

\newcommand{\px}{\frac{\partial}{\partial x}} % Partial wrt x
\newcommand{\py}{\frac{\partial}{\partial y}} % Partial wrt y

\newcommand{\pfx}{\frac{\partial f}{\partial x}} % Partial of f wrt x
\newcommand{\pfy}{\frac{\partial f}{\partial y}} % Partial of f wrt y
\newcommand{\pfxy}{\frac{\partial^2 f}{\partial y \partial x}} % Partial of f wrt y
\newcommand{\pfyx}{\frac{\partial^2 f}{\partial x \partial y}} % Partial of f wrt y

\newcommand{\ux}{\frac{\partial u}{\partial x }} % Partial of u wrt x
\newcommand{\uk}{\frac{\partial u}{\partial k }} % Partial of u wrt k
\newcommand{\ut}{\frac{\partial u}{\partial t}} % Partial of u wrt t
\newcommand{\utt}{\frac{\partial^2u}{\partial t^2}} % Partial of u wrt t
\newcommand{\us}{\frac{\partial u}{\partial s}} % Partial of u wrt t
\newcommand{\uss}{\frac{\partial^2 u}{\partial s^2}} % Partial of u wrt t
\newcommand{\kx}{\frac{\partial k}{\partial x }} % Partial of k wrt x
\newcommand{\kt}{\frac{\partial k}{\partial t }} % Partial of k wrt t

\newcommand{\pxu}{\frac{\partial x}{\partial u}} % x wrt u
\newcommand{\pxv}{\frac{\partial x}{\partial v}} % x wrt v
\newcommand{\pxw}{\frac{\partial x}{\partial w}} % x wrt v
\newcommand{\pxt}{\frac{\partial x}{\partial t}} % x wrt t
\newcommand{\pyu}{\frac{\partial y}{\partial u}} % y wrt u
\newcommand{\pyv}{\frac{\partial y}{\partial v}} % y wrt v
\newcommand{\pyw}{\frac{\partial y}{\partial w}} % y wrt v
\newcommand{\pyt}{\frac{\partial y}{\partial t}} % y wrt t
\newcommand{\pzu}{\frac{\partial z}{\partial u}} % z wrt u
\newcommand{\pzv}{\frac{\partial z}{\partial v}} % z wrt v
\newcommand{\pzw}{\frac{\partial z}{\partial w}} % z wrt v
\newcommand{\pzt}{\frac{\partial z}{\partial t}} % z wrt t


\newcommand{\VCT}{\textit{Vector Calculus} by Michael Corral} % Vector Calculus Textbook
\newcommand{\CAT}{\textit{College Algebra} by Carl Stitz and Jeff Zeager} % College Algebra Textbook
\newcommand{\From}{The following questions are related to } % Questions ....
% ~ ~ ~ ~ ~ ~ ~ ~ ~ ~ ~ ~ ~ ~ ~ ~ ~ ~ ~ ~ ~ ~ ~ ~ ~ ~ ~ ~ ~ ~ ~ ~ ~ ~ ~ ~ ~ ~ ~ ~ ~ ~ ~ ~ ~ ~ ~ ~
% ONLY USED FOR EDITING
\newcommand{\rednote}[1]{{\color{red}\textit{\textbf{#1}}}} % Shortcut for formatting notes for developers
\newcommand{\FromC}[1]{{\color{DGrey}\textit{#1}}} % Shortcut for coloring the "from" text
% ~ ~ ~ ~ ~ ~ ~ ~ ~ ~ ~ ~ ~ ~ ~ ~ ~ ~ ~ ~ ~ ~ ~ ~ ~ ~ ~ ~ ~ ~ ~ ~ ~ ~ ~ ~ ~ ~ ~ ~ ~ ~ ~ ~ ~ ~ ~ ~
% AUGMENTED MATRIX MACRO
% thanks to http://tex.stackexchange.com/questions/2233/whats-the-best-way-make-an-augmented-coefficient-matrix
\newenvironment{amatrix}[1]{%
  \left[\begin{array}{@{}*{#1}{c}|c@{}}
}{%
  \end{array}\right]
}
% ~ ~ ~ ~ ~ ~ ~ ~ ~ ~ ~ ~ ~ ~ ~ ~ ~ ~ ~ ~ ~ ~ ~ ~ ~ ~ ~ ~ ~ ~ ~ ~ ~ ~ ~ ~ ~ ~ ~ ~ ~ ~ ~ ~ ~ ~ ~ ~
% PAGE LAYOUT
\addtolength{\topmargin}{10pt}
\addtolength{\headsep}{10pt}
\addtolength{\textheight}{-20pt}


\usepackage{multienum}
\usepackage{wrapfig}

\date{}
% ~ ~ ~ ~ ~ ~ ~ ~ ~ ~ ~ ~ ~ ~ ~ ~ ~ ~ ~ ~ ~ ~ ~ ~ ~ ~ ~ ~ ~ ~ ~ ~ ~ ~ ~ ~ ~ ~ ~ ~ ~ ~ ~ ~ ~ ~ ~ ~
\begin{document}
\begin{center}
\textsc{\LARGE Written Assignment 4}\\[0.5cm]
\end{center}
\From sections 2.2 through 2.4 of \VCT.

\section*{Questions}

\begin{enumerate}
% ~~~~~~~~~~~~~~~~~~~~~~~~~~~~~~~~~~~~~~~~~~~~~~~~~~~~~~~~~
\item

Compute all first and second partial derivatives and the gradient 
for each of the following functions.
\begin{enumerate}
 \item $f(x,y) = xy\sin(\frac{x}{y})$
 \item $g(x,y) = \dfrac{x+y}{e^{xy}}$
\end{enumerate}

% ~~~~~~~~~~~~~~~~~~~~~~~~~~~~~~~~~~~~~~~~~~~~~~~~~~~~~~~~~
\item

Compute the gradient of the function $z = h(x, y)$ defined implicitly by the
equation $xy = z^{x+y}$.

% ~~~~~~~~~~~~~~~~~~~~~~~~~~~~~~~~~~~~~~~~~~~~~~~~~~~~~~~~~
\item

Find the equations of the planes tangent to the following surfaces
at the specified points.
\begin{enumerate}
 \item $f(x,y) = \frac{x^2}{9} - y^2$ at the point $(2, \frac{1}{3})$.
 \item $x^2 + x + y - z^2 = 7$ at the point $(2, 5, 2)$.
\end{enumerate}

% ~~~~~~~~~~~~~~~~~~~~~~~~~~~~~~~~~~~~~~~~~~~~~~~~~~~~~~~~~
\item

Let $f(x,y) = e^{-(x^2+x+1+y^2-2y)}$.
\begin{enumerate}
 \item In which direction does $f$ increase fastest from the point
  (1,1)?
 \item Compute the rate of change of $f$ in the direction of the
  point $(5,4)$ from the point $(2,1)$.
\end{enumerate}

% ~~~~~~~~~~~~~~~~~~~~~~~~~~~~~~~~~~~~~~~~~~~~~~~~~~~~~~~~~
\item

Find the Jacobian matrix for the function
$f(x,y) = \begin{bmatrix} \sin(x+y) \\ \ln(xy) \end{bmatrix}$.

% ~~~~~~~~~~~~~~~~~~~~~~~~~~~~~~~~~~~~~~~~~~~~~~~~~~~~~~~~~
\item
\Emph{Application: The Heat Equation}

A \emph{partial differential equation} is an equation that involves
the partial derivatives of a function.  For example, 
$(\frac{\partial f}{\partial x})^2 - \frac{\partial^2 f}{\partial x 
\partial y} = 0$ is a partial differential equation that relates
the first- and second-order partial derivatives of an unknown function 
$f$.

A function $f$ is said to \emph{satisfy} a partial differential
equation when the equation holds upon substitution of the function's
partial derivatives.  For example, the function $f(x, y) = y$ 
satisfies the above partial differential equation since
$(\frac{\partial}{\partial x}(y))^2 = 0$ and $\frac{\partial^2}{\partial x \partial y}(y) = \frac{\partial}{\partial x}(1) = 0$.

The one-dimensional heat equation is a partial differential equation
which relates physical properties of a material to the function that
specifies the exact temperature at any point along a fixed-length rod
of that material.  Under a few assumptions, it can be stated in the
following form:
\begin{equation}
\frac{\partial u}{\partial t} = \frac{K_0}{c\rho} \frac{\partial^2 u}{\partial x^2} \label{eqn:heat}
\end{equation}
where $u = u(x,t)$, $x$ is the position along the rod, $t$ is the point in time, 
$K_0$ is the thermal conductivity of the material, $c$ is its specific heat, 
and $\rho$ is its mass density.

Show that $u(x,t) = e^{-tK_0/(c\rho)}\sin(x)$ is a solution to (\ref{eqn:heat}).

\end{enumerate}

%%%%%%%%%%%%%%%%%%%%%%%%%%%%%%%%%%%%%%
%%%%%%%%%%%%%%%%%%%%%%%%%%%%%%%%%%%%%%
%%%%%%%%%%%%%%%%%%%%%%%%%%%%%%%%%%%%%%

% SOLUTIONS
\newpage
\section*{Solutions}
\begin{enumerate}

% ~~~~~~~~~~~~~~~~~~~~~~~~~~~~~~~~~~~~~~~~~~~~~~~~~~~~~~~~~
\item
\begin{enumerate}

 \item
  \begin{align*}
   \frac{\partial f}{\partial x}
   &= \frac{\partial}{\partial x}(xy \sin(\frac{x}{y})) \\
   &= y\sin(\frac{x}{y}) + xy\cos(\frac{x}{y})\frac{1}{y} \\
   &= y\sin(\frac{x}{y}) + x\cos(\frac{x}{y}) \\
   \frac{\partial^2 f}{\partial x^2}
   &= \frac{\partial}{\partial x}\frac{\partial f}{\partial x} \\
   &= \frac{\partial}{\partial x}(y\sin(\frac{x}{y}) + x\cos(\frac{x}{y})) \\
   &= y\cos(\frac{x}{y})\frac{1}{y} + \cos(\frac{x}{y}) - x\sin(\frac{x}{y})\frac{1}{y} \\
   &= 2\cos(\frac{x}{y}) - \frac{x}{y}\sin(\frac{x}{y}) \\
  \frac{\partial^2 f}{\partial y \partial x}
   &= \frac{\partial}{\partial y}\frac{\partial f}{\partial x} \\
   &= \frac{\partial}{\partial y}(y\sin(\frac{x}{y}) + x\cos(\frac{x}{y})) \\
   &= \sin(\frac{x}{y}) + y\cos(\frac{x}{y})(\frac{-x}{y^2}) - x\sin(\frac{x}{y})(\frac{-x}{y^2}) \\
   &= \sin(\frac{x}{y}) - \frac{x}{y}\cos(\frac{x}{y}) + \frac{x^2}{y^2}\sin(\frac{x}{y}) \\
   &= \frac{x^2+y^2}{y^2}\sin(\frac{x}{y}) - \frac{x}{y}\cos(\frac{x}{y}) \\
   \frac{\partial f}{\partial y}
   &= \frac{\partial}{\partial y}(xy \sin(\frac{x}{y})) \\
   &= x\sin(\frac{x}{y}) + xy\cos(\frac{x}{y})(\frac{-x}{y^2}) \\
   &= x\sin(\frac{x}{y}) - \frac{x^2}{y}\cos(\frac{x}{y}) \\
   \frac{\partial^2 f}{\partial x \partial y}
   &= \frac{\partial}{\partial x}\frac{\partial f}{\partial y} \\
   &= \frac{\partial}{\partial x}(x\sin(\frac{x}{y}) - \frac{x^2}{y}\cos(\frac{x}{y})) \\
   &= \sin(\frac{x}{y}) + x\cos(\frac{x}{y})(\frac{1}{y}) - \frac{2x}{y}\cos(\frac{x}{y}) + \frac{x^2}{y}\sin(\frac{x}{y})(\frac{1}{y}) \\
   &= \frac{x^2+y^2}{y^2}\sin(\frac{x}{y}) - \frac{x}{y}\cos(\frac{x}{y}) \\
   \nabla f
   &= \begin{bmatrix}
       \frac{\partial f}{\partial x} \\
       \frac{\partial f}{\partial y}
      \end{bmatrix} \\
   &= \begin{bmatrix}
       y\sin(\frac{x}{y}) + x\cos(\frac{x}{y}) \\
       x\sin(\frac{x}{y}) - \frac{x^2}{y}\cos(\frac{x}{y})
      \end{bmatrix}
  \end{align*}

 \item
  \begin{align*}
   \frac{\partial f}{\partial x}
   &= \frac{\partial}{\partial x}(\frac{x+y}{e^{xy}}) \\
   &= \frac{e^{xy} - (x+y)ye^{xy}}{e^{2xy}} \\
   &= \frac{1 - xy - y^2}{e^{xy}} \\
   \frac{\partial^2 f}{\partial x^2}
   &= \frac{\partial}{\partial x}\frac{\partial f}{\partial x} \\
   &= \frac{\partial}{\partial x}(\frac{1 - xy - y^2}{e^{xy}}) \\
   &= \frac{e^{xy}(-y) - (1 - xy - y^2)ye^{xy}}{e^{2xy}} \\
   &= \frac{-2y + xy^2 + y^3}{e^{xy}} \\
   \frac{\partial^2 f}{\partial y \partial x}
   &= \frac{\partial}{\partial y}\frac{\partial f}{\partial x} \\
   &= \frac{\partial}{\partial y}(\frac{1 - xy - y^2}{e^{xy}}) \\
   &= \frac{e^{xy}(-x - 2y) - (1 - xy - y^2)xe^{xy}}{e^{2xy}} \\
   &= \frac{-2x - 2y + x^2y + xy^2}{e^{xy}} \\
   \frac{\partial f}{\partial y}
   &= \frac{\partial}{\partial y}(\frac{x+y}{e^{xy}}) \\
   &= \frac{e^{xy} - (x+y)xe^{xy}}{e^{2xy}} \\
   &= \frac{1 - x^2 - xy}{e^{xy}} \\
   \frac{\partial^2 f}{\partial y^2}
   &= \frac{\partial}{\partial y}\frac{\partial f}{\partial y} \\
   &= \frac{\partial}{\partial y}(\frac{1 - x^2 - xy}{e^{xy}}) \\
   &= \frac{e^{xy}(-x) - (1 - x^2 - xy)xe^{xy}}{e^{2xy}} \\
   &= \frac{-2x + x^3 + x^2y}{e^{xy}} \\
   \frac{\partial^2 f}{\partial x \partial y}
   &= \frac{\partial}{\partial x}\frac{\partial f}{\partial y} \\
   &= \frac{\partial}{\partial x}(\frac{1 - x^2 - xy}{e^{xy}}) \\
   &= \frac{e^{xy}(-2x - y) - (1 - x^2 - xy)ye^{xy}}{e^{2xy}} \\
   &= \frac{-2x - 2y + x^2y + xy^2}{e^{xy}} \\
   \nabla f
   &= \begin{bmatrix}
       \frac{\partial f}{\partial x} \\
       \frac{\partial f}{\partial y}
      \end{bmatrix} \\
   &= \begin{bmatrix}
       \dfrac{1 - xy - y^2}{e^{xy}} \\
       \dfrac{1 - x^2 - xy}{e^{xy}}
      \end{bmatrix}
  \end{align*}

\end{enumerate}

% ~~~~~~~~~~~~~~~~~~~~~~~~~~~~~~~~~~~~~~~~~~~~~~~~~~~~~~~~~
\item

We must first compute the partial derivatives $\frac{\partial z}{\partial x}$ and $\frac{\partial z}{\partial y}$:

 \begin{align*}
  \frac{\partial}{\partial x}(xy) &= \frac{\partial}{\partial x}(z^{x+y}) \\
  y &= \frac{\partial}{\partial x}(e^{\ln(z)(x+y)}) \\
  y &= e^{\ln(z)(x+y)}\frac{\partial}{\partial x}(\ln(z)(x+y)) \\
  y &= e^{\ln(z)(x+y)}(\frac{1}{z}\frac{\partial z}{\partial x}(x+y) + \ln(z)) \\
  y &= z^{x+y}(\frac{x+y}{z}\frac{\partial z}{\partial x} + \ln(z)) \\
  y &= \frac{\partial z}{\partial x}(x+y)z^{x+y-1} + z^{x+y}\ln(z) \\
  \frac{\partial z}{\partial x} &= \frac{y - z^{x+y}\ln(z)}{(x+y)z^{x+y-1}}
 \end{align*}

 \begin{align*}
  \frac{\partial}{\partial y}(xy) &= \frac{\partial}{\partial y}(z^{x+y}) \\
  x &= \frac{\partial}{\partial y}(e^{\ln(z)(x+y)}) \\
  x &= e^{\ln(z)(x+y)}\frac{\partial}{\partial y}(\ln(z)(x+y)) \\
  x &= e^{\ln(z)(x+y)}(\frac{1}{z}\frac{\partial z}{\partial y}(x+y) + \ln(z)) \\
  x &= z^{x+y}(\frac{x+y}{z}\frac{\partial z}{\partial y} + \ln(z)) \\
  x &= \frac{\partial z}{\partial y}(x+y)z^{x+y-1} + z^{x+y}\ln(z) \\
  \frac{\partial z}{\partial y} &= \frac{x - z^{x+y}\ln(z)}{(x+y)z^{x+y-1}}
 \end{align*}

So $\nabla h(x,y) = \begin{bmatrix}
                \dfrac{y - z^{x+y}\ln(z)}{(x+y)z^{x+y-1}} \\
                \dfrac{x - z^{x+y}\ln(z)}{(x+y)z^{x+y-1}}
               \end{bmatrix}$.

% ~~~~~~~~~~~~~~~~~~~~~~~~~~~~~~~~~~~~~~~~~~~~~~~~~~~~~~~~~
\item

 Recall that the equation of the plane tangent to the surface defined by 
 $g(x,y,z) = 0$ at the point $(x_0, y_0, z_0)$ is $\nabla g(x_0,y_0,z_0)
 \cdot \begin{bmatrix} x - x_0 \\ y - y_0 \\ z - z_0 \end{bmatrix} = 0$.

 \begin{enumerate}
  \item Define $g(x,y,z) = f(x,y) - z = \frac{x^2}{9} - y^2 - z$, so that the 
   surface  $z = f(x,y)$ is equivalently defined by the equation
   $g(x,y,z) = 0$.
   We can then find the equation of the tangent plane at the point 
   $(2, \frac{1}{3}, \frac{1}{3})$ by computing $\nabla g$ and applying the 
   above formula:

   \begin{align*}
    \nabla g(x,y,z) &=
     \begin{bmatrix} \frac{2x}{9} \\ -2y \\ -1 \end{bmatrix} \\
    \nabla g(2,\frac{1}{3},\frac{1}{3}) &=
     \begin{bmatrix} \frac{4}{9} \\ -\frac{2}{3} \\ -1 \end{bmatrix} \\
   \end{align*}

   This yields the equation
   \begin{align*}
    \frac{4}{9}(x - 2) - \frac{2}{3}(y - \frac{1}{3}) - (z - \frac{1}{3}) &= 0 \\
    \frac{4}{9}x - \frac{2}{3}y - z &= \frac{1}{3} \\
    4x - 6y - 9z &= 3.
   \end{align*}

  \item Define $g(x,y,z) = x^2 + x + y - z^2 - 7$.  Then compute
   \begin{align*}
    \nabla g(x,y,z) &= \begin{bmatrix} 2x + 1 \\ 1 \\ -2z \end{bmatrix} \\
    \nabla g(2,5,2) &= \begin{bmatrix} 5 \\ 1 \\ -4 \end{bmatrix}.
   \end{align*}

   Thus the equation of the tangent plane is
   \begin{align*}
    5(x - 2) + (y - 5) - 4(z - 2) &= 0 \\
    5x + y - 4z &= 7.
   \end{align*}
 \end{enumerate}

% ~~~~~~~~~~~~~~~~~~~~~~~~~~~~~~~~~~~~~~~~~~~~~~~~~~~~~~~~~
\item

 \begin{enumerate}
  \item $f$ will increase fastest in the direction of the vector
   $\nabla f(x_0,y_0)$ from the point $(x_0, y_0)$.  Therefore we compute and
   evaluate the gradient $\nabla f$ at the point $(1, 1)$:

   \begin{align*}
    \nabla f &= \begin{bmatrix}
                 \dfrac{\partial f}{\partial x} \\
                 \dfrac{\partial f}{\partial y}
                \end{bmatrix} \\
             &= \begin{bmatrix}
                 \dfrac{\partial}{\partial x}(e^{-(x^2+x+1+y^2-2y)}) \\
                 \dfrac{\partial}{\partial y}(e^{-(x^2+x+1+y^2-2y)})
                \end{bmatrix} \tag{1} \label{fastest_incr:grad}  \\
             &= \begin{bmatrix}
                 (-2x - 1)e^{-(x^2+x+1+y^2-2y)} \\
                 (2y - 2)e^{-(x^2+x+1+y^2-2y)}
                \end{bmatrix} \\
    \nabla f(1, 1) &= \begin{bmatrix} -3e^{-2} \\ 0 \end{bmatrix}
   \end{align*}

 \item The rate of change of $f$ in the direction of the vector $\mathbf{v}$
  from the point $(x, y)$ is given by $\nabla f(x,y) \cdot \mathbf{v}$.  Using
  (\ref{fastest_incr:grad}) we find that $\nabla f(2, 1) =
  \begin{bmatrix} -5e^{-6} \\ 0 \end{bmatrix}$.  The direction from $(2, 1)$
  to $(5, 4)$ is given by the vector $\begin{bmatrix} 3 \\ 3 \end{bmatrix}$;
  thus the rate of change of $f$ from $(2, 1)$ towards $(5, 4)$ is
  $\begin{bmatrix} -5e^{-6} \\ 0 \end{bmatrix} \cdot
   \begin{bmatrix} 3 \\ 3 \end{bmatrix} = -15e^{-6}$.

 \end{enumerate}

% ~~~~~~~~~~~~~~~~~~~~~~~~~~~~~~~~~~~~~~~~~~~~~~~~~~~~~~~~~
\item

First we find the partial derivatives of each component with respect to each
variable:

 \begin{align*}
  \frac{\partial f_1}{\partial x} &= cos(x + y) \\
  \frac{\partial f_1}{\partial y} &= cox(x + y) \\
  \frac{\partial f_2}{\partial x} &= \frac{1}{x} \\
  \frac{\partial f_2}{\partial y} &= \frac{1}{y}
 \end{align*}

The Jacobian matrix, then, is
$\begin{bmatrix}
 cos(x + y) & cos(x + y) \\ \frac{1}{x} & \frac{1}{y}
\end{bmatrix}$.

% ~~~~~~~~~~~~~~~~~~~~~~~~~~~~~~~~~~~~~~~~~~~~~~~~~~~~~~~~~
\item

We only need to verify that (\ref{eqn:heat}) holds for the given function $u$:

\begin{align*}
 \frac{\partial u}{\partial t} &= \frac{K_0}{c\rho} \frac{\partial^2 u}{\partial x^2} \\
 -\frac{K_0}{c\rho}e^{-tK_0/(c\rho)}\sin(x) &=
 \frac{K_0}{c\rho}\frac{\partial}{\partial x}(e^{-tK_0/(c\rho)}\cos(x)) \\
 -\frac{K_0}{c\rho}e^{-tK_0/(c\rho)}\sin(x) &=
 -\frac{K_0}{c\rho}e^{-tK_0/(c\rho)}\sin(x)
\end{align*}

which is true.  So $u(x,t) = e^{-tK_0/(c\rho)}\sin(x)$ is indeed a solution
to the heat equation (\ref{eqn:heat}).

\end{enumerate}

\end{document}
