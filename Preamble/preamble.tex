% ~ ~ ~ ~ ~ ~ ~ ~ ~ ~ ~ ~ ~ ~ ~ ~ ~ ~ ~ ~ ~ ~ ~ ~ ~ ~ ~ ~ ~ ~ ~ ~ ~ ~ ~ ~ ~ ~ ~ ~ ~ ~ ~ ~ ~ ~ ~ ~

\usepackage[utf8]{inputenc}
\usepackage{amsmath}
\usepackage{amssymb} % for \mathbb
\usepackage{graphicx} % for figures
\usepackage{color}
\usepackage[usenames,dvipsnames]{xcolor}
\usepackage{hyperref} % for hyperlinks
\usepackage{float} % for figures
% ~ ~ ~ ~ ~ ~ ~ ~ ~ ~ ~ ~ ~ ~ ~ ~ ~ ~ ~ ~ ~ ~ ~ ~ ~ ~ ~ ~ ~ ~ ~ ~ ~ ~ ~ ~ ~ ~ ~ ~ ~ ~ ~ ~ ~ ~ ~ ~
% CUSTOM COLOUR
\definecolor{DGrey}{rgb}{0.1,0.1,0.1} % Dark Grey
% ~ ~ ~ ~ ~ ~ ~ ~ ~ ~ ~ ~ ~ ~ ~ ~ ~ ~ ~ ~ ~ ~ ~ ~ ~ ~ ~ ~ ~ ~ ~ ~ ~ ~ ~ ~ ~ ~ ~ ~ ~ ~ ~ ~ ~ ~ ~ ~
\newcommand{\DocumentTitle}{Introduction to Engineering Written Assignment Questions  }
\title{\DocumentTitle}
\date{}
\author{Greg Mayer and Daniel Connelly}
% ~ ~ ~ ~ ~ ~ ~ ~ ~ ~ ~ ~ ~ ~ ~ ~ ~ ~ ~ ~ ~ ~ ~ ~ ~ ~ ~ ~ ~ ~ ~ ~ ~ ~ ~ ~ ~ ~ ~ ~ ~ ~ ~ ~ ~ ~ ~ ~
% HEADER/FOOTER
\usepackage{fancyheadings}
\pagestyle{myheadings} % set headings to be user defined
\fancyhead{} % To create custom header, clear default layout
\renewcommand{\subsectionmark}[1]{\markright{{\color{DGrey}\thesubsection} \ {\color{DGrey}#1}}}
\fancyhead[LE,LO]{\subsectionmark} % To create custom header, clear default layout

% ~ ~ ~ ~ ~ ~ ~ ~ ~ ~ ~ ~ ~ ~ ~ ~ ~ ~ ~ ~ ~ ~ ~ ~ ~ ~ ~ ~ ~ ~ ~ ~ ~ ~ ~ ~ ~ ~ ~ ~ ~ ~ ~ ~ ~ ~ ~ ~
% ENUMERATION

\usepackage{enumitem}   % so that question numbers can be formatted 
\setenumerate[1]{label=\thesubsection.\arabic*.} % enumerate environment: add section numbers to items
% ~ ~ ~ ~ ~ ~ ~ ~ ~ ~ ~ ~ ~ ~ ~ ~ ~ ~ ~ ~ ~ ~ ~ ~ ~ ~ ~ ~ ~ ~ ~ ~ ~ ~ ~ ~ ~ ~ ~ ~ ~ ~ ~ ~ ~ ~ ~ ~
% MARGINS
\usepackage{anysize}
\marginsize{2.5cm}{2.5cm}{1cm}{1cm}
% ~ ~ ~ ~ ~ ~ ~ ~ ~ ~ ~ ~ ~ ~ ~ ~ ~ ~ ~ ~ ~ ~ ~ ~ ~ ~ ~ ~ ~ ~ ~ ~ ~ ~ ~ ~ ~ ~ ~ ~ ~ ~ ~ ~ ~ ~ ~ ~
% PAGE NUMBERING
\pagenumbering{arabic}
% ~ ~ ~ ~ ~ ~ ~ ~ ~ ~ ~ ~ ~ ~ ~ ~ ~ ~ ~ ~ ~ ~ ~ ~ ~ ~ ~ ~ ~ ~ ~ ~ ~ ~ ~ ~ ~ ~ ~ ~ ~ ~ ~ ~ ~ ~ ~ ~
% Custom Commands
\newcommand{\Emph}[1]{\textbf{#1}} % Emphasize
\newcommand{\R}{\mathbb{R}} 
\newcommand{\BM}{\begin{bmatrix}} % Begin Matrix
\newcommand{\EM}{\end{bmatrix}} % End Matrix
\newcommand{\BEN}{\begin{enumerate}[leftmargin=1.1cm]}% Begin ENumerate
\newcommand{\EEN}{\end{enumerate}} % End ENumerate
\newcommand{\MB}{\mathbf} % Math Bold

\newcommand{\px}{\frac{\partial}{\partial x}} % Partial wrt x
\newcommand{\py}{\frac{\partial}{\partial y}} % Partial wrt y

\newcommand{\pfx}{\frac{\partial f}{\partial x}} % Partial of f wrt x
\newcommand{\pfy}{\frac{\partial f}{\partial y}} % Partial of f wrt y
\newcommand{\pfxy}{\frac{\partial^2 f}{\partial y \partial x}} % 
\newcommand{\pfyx}{\frac{\partial^2 f}{\partial x \partial y}} % 
\newcommand{\pfxx}{\frac{\partial^2 f}{ \partial x^2}} % 
\newcommand{\pfyy}{\frac{\partial^2 f}{ \partial y^2}} % 

\newcommand{\ux}{\frac{\partial u}{\partial x }} % Partial of u wrt x
\newcommand{\uk}{\frac{\partial u}{\partial k }} % Partial of u wrt k
\newcommand{\ut}{\frac{\partial u}{\partial t}} % Partial of u wrt t
\newcommand{\utt}{\frac{\partial^2u}{\partial t^2}} % Partial of u wrt t
\newcommand{\us}{\frac{\partial u}{\partial s}} % Partial of u wrt t
\newcommand{\uss}{\frac{\partial^2 u}{\partial s^2}} % Partial of u wrt t
\newcommand{\kx}{\frac{\partial k}{\partial x }} % Partial of k wrt x
\newcommand{\kt}{\frac{\partial k}{\partial t }} % Partial of k wrt t

\newcommand{\pxu}{\frac{\partial x}{\partial u}} % x wrt u
\newcommand{\pxv}{\frac{\partial x}{\partial v}} % x wrt v
\newcommand{\pxw}{\frac{\partial x}{\partial w}} % x wrt v
\newcommand{\pxt}{\frac{\partial x}{\partial t}} % x wrt t
\newcommand{\pyu}{\frac{\partial y}{\partial u}} % y wrt u
\newcommand{\pyv}{\frac{\partial y}{\partial v}} % y wrt v
\newcommand{\pyw}{\frac{\partial y}{\partial w}} % y wrt v
\newcommand{\pyt}{\frac{\partial y}{\partial t}} % y wrt t
\newcommand{\pzu}{\frac{\partial z}{\partial u}} % z wrt u
\newcommand{\pzv}{\frac{\partial z}{\partial v}} % z wrt v
\newcommand{\pzw}{\frac{\partial z}{\partial w}} % z wrt v
\newcommand{\pzt}{\frac{\partial z}{\partial t}} % z wrt t


\newcommand{\VCT}{\textit{Vector Calculus} by Michael Corral} % Vector Calculus Textbook
\newcommand{\CAT}{\textit{College Algebra} by Carl Stitz and Jeff Zeager} % College Algebra Textbook
\newcommand{\From}{The following questions are related to } % Questions ....
% ~ ~ ~ ~ ~ ~ ~ ~ ~ ~ ~ ~ ~ ~ ~ ~ ~ ~ ~ ~ ~ ~ ~ ~ ~ ~ ~ ~ ~ ~ ~ ~ ~ ~ ~ ~ ~ ~ ~ ~ ~ ~ ~ ~ ~ ~ ~ ~
% ONLY USED FOR EDITING
\newcommand{\rednote}[1]{{\color{red}\textit{\textbf{#1}}}} % Shortcut for formatting notes for developers
\newcommand{\FromC}[1]{{\color{DGrey}\textit{#1}}} % Shortcut for coloring the "from" text
% ~ ~ ~ ~ ~ ~ ~ ~ ~ ~ ~ ~ ~ ~ ~ ~ ~ ~ ~ ~ ~ ~ ~ ~ ~ ~ ~ ~ ~ ~ ~ ~ ~ ~ ~ ~ ~ ~ ~ ~ ~ ~ ~ ~ ~ ~ ~ ~
% AUGMENTED MATRIX MACRO
% thanks to http://tex.stackexchange.com/questions/2233/whats-the-best-way-make-an-augmented-coefficient-matrix
\newenvironment{amatrix}[1]{%
  \left[\begin{array}{@{}*{#1}{c}|c@{}}
}{%
  \end{array}\right]
}
% ~ ~ ~ ~ ~ ~ ~ ~ ~ ~ ~ ~ ~ ~ ~ ~ ~ ~ ~ ~ ~ ~ ~ ~ ~ ~ ~ ~ ~ ~ ~ ~ ~ ~ ~ ~ ~ ~ ~ ~ ~ ~ ~ ~ ~ ~ ~ ~
\hypersetup{
%    bookmarks=true,         % show bookmarks bar?
%    unicode=false,          % non-Latin characters in Acrobat�s bookmarks
%    pdftoolbar=true,        % show Acrobat�s toolbar?
%    pdfmenubar=true,        % show Acrobat�s menu?
%    pdffitwindow=false,     % window fit to page when opened
%    pdfstartview={FitH},    % fits the width of the page to the window
%    pdftitle={My title},    % title
%    pdfauthor={Author},     % author
%    pdfsubject={Subject},   % subject of the document
%    pdfcreator={Creator},   % creator of the document
%    pdfproducer={Producer}, % producer of the document
%    pdfkeywords={keyword1} {key2} {key3}, % list of keywords
    pdfnewwindow=true,      % links in new window
    colorlinks=true,       % false: boxed links; true: colored links
    linkcolor=black,          % color of internal links (change box color with linkbordercolor)
    citecolor=green,        % color of links to bibliography
    filecolor=magenta,      % color of file links
    urlcolor=blue           % color of external links
}

% PAGE LAYOUT
\addtolength{\topmargin}{10pt}
\addtolength{\headsep}{10pt}
\addtolength{\textheight}{-20pt}

