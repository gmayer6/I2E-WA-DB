\documentclass{article}
% ~ ~ ~ ~ ~ ~ ~ ~ ~ ~ ~ ~ ~ ~ ~ ~ ~ ~ ~ ~ ~ ~ ~ ~ ~ ~ ~ ~ ~ ~ ~ ~ ~ ~ ~ ~ ~ ~ ~ ~ ~ ~ ~ ~ ~ ~ ~ ~

\usepackage[utf8]{inputenc}
\usepackage{amsmath}
\usepackage{amssymb} % for \mathbb
\usepackage{graphicx} % for figures
\usepackage{color}
\usepackage[usenames,dvipsnames]{xcolor}
\usepackage{hyperref} % for hyperlinks
\usepackage{float} % for figures
% ~ ~ ~ ~ ~ ~ ~ ~ ~ ~ ~ ~ ~ ~ ~ ~ ~ ~ ~ ~ ~ ~ ~ ~ ~ ~ ~ ~ ~ ~ ~ ~ ~ ~ ~ ~ ~ ~ ~ ~ ~ ~ ~ ~ ~ ~ ~ ~
% CUSTOM COLOUR
\definecolor{DGrey}{rgb}{0.1,0.1,0.1} % Dark Grey
% ~ ~ ~ ~ ~ ~ ~ ~ ~ ~ ~ ~ ~ ~ ~ ~ ~ ~ ~ ~ ~ ~ ~ ~ ~ ~ ~ ~ ~ ~ ~ ~ ~ ~ ~ ~ ~ ~ ~ ~ ~ ~ ~ ~ ~ ~ ~ ~
\title{ Introduction to Engineering Written Assignment Questions}
\date{January 2013}
\author{Greg Mayer and Daniel Connelly}
% ~ ~ ~ ~ ~ ~ ~ ~ ~ ~ ~ ~ ~ ~ ~ ~ ~ ~ ~ ~ ~ ~ ~ ~ ~ ~ ~ ~ ~ ~ ~ ~ ~ ~ ~ ~ ~ ~ ~ ~ ~ ~ ~ ~ ~ ~ ~ ~
% HEADER/FOOTER
\usepackage{fancyheadings}
\pagestyle{myheadings} % set headings to be user defined
\fancyhead{} % To create custom header, clear default layout
\renewcommand{\subsectionmark}[1]{\markright{{\color{DGrey}\thesubsection} \ {\color{DGrey}#1}}}
\fancyhead[LE,LO]{\subsectionmark} % To create custom header, clear default layout

% ~ ~ ~ ~ ~ ~ ~ ~ ~ ~ ~ ~ ~ ~ ~ ~ ~ ~ ~ ~ ~ ~ ~ ~ ~ ~ ~ ~ ~ ~ ~ ~ ~ ~ ~ ~ ~ ~ ~ ~ ~ ~ ~ ~ ~ ~ ~ ~
% ENUMERATION

\usepackage{enumitem}   % so that question numbers can be formatted 
\setenumerate[1]{label=\thesubsection.\arabic*.} % enumerate environment: add section numbers to items
% ~ ~ ~ ~ ~ ~ ~ ~ ~ ~ ~ ~ ~ ~ ~ ~ ~ ~ ~ ~ ~ ~ ~ ~ ~ ~ ~ ~ ~ ~ ~ ~ ~ ~ ~ ~ ~ ~ ~ ~ ~ ~ ~ ~ ~ ~ ~ ~
% MARGINS
\usepackage{anysize}
\marginsize{2.5cm}{2.5cm}{1cm}{1cm}
% ~ ~ ~ ~ ~ ~ ~ ~ ~ ~ ~ ~ ~ ~ ~ ~ ~ ~ ~ ~ ~ ~ ~ ~ ~ ~ ~ ~ ~ ~ ~ ~ ~ ~ ~ ~ ~ ~ ~ ~ ~ ~ ~ ~ ~ ~ ~ ~
% PAGE NUMBERING
\pagenumbering{arabic}
% ~ ~ ~ ~ ~ ~ ~ ~ ~ ~ ~ ~ ~ ~ ~ ~ ~ ~ ~ ~ ~ ~ ~ ~ ~ ~ ~ ~ ~ ~ ~ ~ ~ ~ ~ ~ ~ ~ ~ ~ ~ ~ ~ ~ ~ ~ ~ ~
% Custom Commands
\newcommand{\Emph}[1]{\textbf{#1}} % Emphasize
\newcommand{\R}{\mathbb{R}} 
\newcommand{\BM}{\begin{bmatrix}} % Begin Matrix
\newcommand{\EM}{\end{bmatrix}} % End Matrix
\newcommand{\BEN}{\begin{enumerate}[leftmargin=1.1cm]}% Begin ENumerate
\newcommand{\EEN}{\end{enumerate}} % End ENumerate
\newcommand{\MB}{\mathbf} % Math Bold

\newcommand{\px}{\frac{\partial}{\partial x}} % Partial wrt x
\newcommand{\py}{\frac{\partial}{\partial y}} % Partial wrt y

\newcommand{\pfx}{\frac{\partial f}{\partial x}} % Partial of f wrt x
\newcommand{\pfy}{\frac{\partial f}{\partial y}} % Partial of f wrt y
\newcommand{\pfxy}{\frac{\partial^2 f}{\partial y \partial x}} % Partial of f wrt y
\newcommand{\pfyx}{\frac{\partial^2 f}{\partial x \partial y}} % Partial of f wrt y

\newcommand{\ux}{\frac{\partial u}{\partial x }} % Partial of u wrt x
\newcommand{\uk}{\frac{\partial u}{\partial k }} % Partial of u wrt k
\newcommand{\ut}{\frac{\partial u}{\partial t}} % Partial of u wrt t
\newcommand{\utt}{\frac{\partial^2u}{\partial t^2}} % Partial of u wrt t
\newcommand{\us}{\frac{\partial u}{\partial s}} % Partial of u wrt t
\newcommand{\uss}{\frac{\partial^2 u}{\partial s^2}} % Partial of u wrt t
\newcommand{\kx}{\frac{\partial k}{\partial x }} % Partial of k wrt x
\newcommand{\kt}{\frac{\partial k}{\partial t }} % Partial of k wrt t

\newcommand{\pxu}{\frac{\partial x}{\partial u}} % x wrt u
\newcommand{\pxv}{\frac{\partial x}{\partial v}} % x wrt v
\newcommand{\pxw}{\frac{\partial x}{\partial w}} % x wrt v
\newcommand{\pxt}{\frac{\partial x}{\partial t}} % x wrt t
\newcommand{\pyu}{\frac{\partial y}{\partial u}} % y wrt u
\newcommand{\pyv}{\frac{\partial y}{\partial v}} % y wrt v
\newcommand{\pyw}{\frac{\partial y}{\partial w}} % y wrt v
\newcommand{\pyt}{\frac{\partial y}{\partial t}} % y wrt t
\newcommand{\pzu}{\frac{\partial z}{\partial u}} % z wrt u
\newcommand{\pzv}{\frac{\partial z}{\partial v}} % z wrt v
\newcommand{\pzw}{\frac{\partial z}{\partial w}} % z wrt v
\newcommand{\pzt}{\frac{\partial z}{\partial t}} % z wrt t


\newcommand{\VCT}{\textit{Vector Calculus} by Michael Corral} % Vector Calculus Textbook
\newcommand{\CAT}{\textit{College Algebra} by Carl Stitz and Jeff Zeager} % College Algebra Textbook
\newcommand{\From}{The following questions are related to } % Questions ....
% ~ ~ ~ ~ ~ ~ ~ ~ ~ ~ ~ ~ ~ ~ ~ ~ ~ ~ ~ ~ ~ ~ ~ ~ ~ ~ ~ ~ ~ ~ ~ ~ ~ ~ ~ ~ ~ ~ ~ ~ ~ ~ ~ ~ ~ ~ ~ ~
% ONLY USED FOR EDITING
\newcommand{\rednote}[1]{{\color{red}\textit{\textbf{#1}}}} % Shortcut for formatting notes for developers
\newcommand{\FromC}[1]{{\color{DGrey}\textit{#1}}} % Shortcut for coloring the "from" text
% ~ ~ ~ ~ ~ ~ ~ ~ ~ ~ ~ ~ ~ ~ ~ ~ ~ ~ ~ ~ ~ ~ ~ ~ ~ ~ ~ ~ ~ ~ ~ ~ ~ ~ ~ ~ ~ ~ ~ ~ ~ ~ ~ ~ ~ ~ ~ ~
% AUGMENTED MATRIX MACRO
% thanks to http://tex.stackexchange.com/questions/2233/whats-the-best-way-make-an-augmented-coefficient-matrix
\newenvironment{amatrix}[1]{%
  \left[\begin{array}{@{}*{#1}{c}|c@{}}
}{%
  \end{array}\right]
}
% ~ ~ ~ ~ ~ ~ ~ ~ ~ ~ ~ ~ ~ ~ ~ ~ ~ ~ ~ ~ ~ ~ ~ ~ ~ ~ ~ ~ ~ ~ ~ ~ ~ ~ ~ ~ ~ ~ ~ ~ ~ ~ ~ ~ ~ ~ ~ ~
% PAGE LAYOUT
\addtolength{\topmargin}{10pt}
\addtolength{\headsep}{10pt}
\addtolength{\textheight}{-20pt}


\usepackage{multienum}
\usepackage{wrapfig}

\date{}
% ~ ~ ~ ~ ~ ~ ~ ~ ~ ~ ~ ~ ~ ~ ~ ~ ~ ~ ~ ~ ~ ~ ~ ~ ~ ~ ~ ~ ~ ~ ~ ~ ~ ~ ~ ~ ~ ~ ~ ~ ~ ~ ~ ~ ~ ~ ~ ~
\begin{document}
\begin{center}
\textsc{\LARGE Midterm 1}\\[0.5cm]
\end{center}
\section*{Questions}

\begin{enumerate}
% ~~~~~~~~~~~~~~~~~~~~~~~~~~~~~~~~~~~~~~~~~~~~~~~~~~~~~~~~~~~~~~~~~~~~~~~~~~~~~~~
\item 
\Emph{Finding the Equation of a Sphere from Four Points}\\
Solve Question 10 from Section 1.6 of \VCT, which is: 
\begin{quotation}
\noindent
It can be shown that any four non-coplanar points (i.e. points that do not lie in the same plane) determine a sphere. Find the equation of the sphere that passes through the points $(0, 0, 0), (0, 0, 2), (1, -4, 3)$ and $(0, -1, 3)$. (Hint: Equation (1.31))
\end{quotation}
\rednote{This question has not been reviewed.}
% ~~~~~~~~~~~~~~~~~~~~~~~~~~~~~~~~~~~~~~~~~~~~~~~~~~~~~~~~~~~~~~~~~~~~~~~~~~~~~~~
\item 
% QUESTION
Provide an example of a function, $f(x,y)$, that has the following two properties.
\begin{itemize}
\item the limit of $f$, along any straight line $L$, as $(x,y) \rightarrow (0,0)$ is equal to zero
\item the limit of $f$ as $(x,y) \rightarrow (0,0)$ does not exist
\end{itemize}
If it is not possible to find a function with these properties, explain why. If it is possible, then show that the limit does not exist.    \textit{Hint: consider straight lines of the form $y=mx$, and parabolas of the form $y=cx^2$.}
% ~~~~~~~~~~~~~~~~~~~~~~~~~~~~~~~~~~~~~~~~~~~~~~~~~~~~~~~~~~~~~~~~~~
\item 
% EQUATION PLANES
Find an equation of the plane that contains the line of intersection of the planes $x+y-z=6$ and $2x+3y+z=10$, and is 
\begin{enumerate}
\item perpendicular to the $xy$ plane.
\item perpendicular to the $xz$ plane.
\item perpendicular to the $yz$ plane.
\end{enumerate}
% ~~~~~~~~~~~~~~~~~~~~~~~~~~~~~~~~~~~~~~~~~~~~~~~~~~~~~~~~~~~~~~~~~~
\item 
% RHOMBUS 
A rhombus is a parallelogram that has sides with equal length. Show that the diagonal lines connecting the opposite corners of the rhombus are perpendicular to each other.
% ~~~~~~~~~~~~~~~~~~~~~~~~~~~~~~~~~~~~~~~~~~~~~~~~~~~~~~~~~~~~~~~~~~
\item 
% POINT IN TWO PLANES
Find a point $(x, y, z)$ that lies on both planes $A$ and $B$,
where $A$ is the plane containing the points
$(2, 4, 3)$, $(-1, 2, -1)$, and $(3, 1, 4)$, and $B$ is the plane
perpendicular to the vector \(\begin{bmatrix} 2 \\ 1 \\ -2 
\end{bmatrix}\) that contains the point $(2, 1, 2)$.\\\\
% ~~~~~~~~~~~~~~~~~~~~~~~~~~~~~~~~~~~~~~~~~~~~~~~~~~~~~~~~~~~~~~~~~~
\item % DETERMINANT LIMIT QUESTION
Define \(f(x,y) = \begin{vmatrix}
                   x^2 & \dfrac{y}{x-y} & 0 \\
                   0 & \dfrac{1}{x-y} & -1 \\
                   \dfrac{y}{y+x} & 0 & \dfrac{1}{y+x}
                  \end{vmatrix}
\).  Does $\lim\limits_{(x,y) \to (0,0)}f(x,y)$ exist?
% ~~~~~~~~~~~~~~~~~~~~~~~~~~~~~~~~~~~~~~~~~~~~~~~~~~~~~~~~~~~~~~~~~~
\item % BEZIER CURVE 
\textbf{Application to CAD}
\BEN
\item
Complete Exercise 14(a) from Section 1.8 of \textit{Vector Calculus} by M. Corral. 
\item
Complete Exercise 14(b) from Section 1.8 of \textit{Vector Calculus} by M. Corral. 
\item In words, describe what $\MB{b}_0^3(t)$ is. Your description should include what points $\MB{b}_0^3(t)$ intersects and for what values of $t$ it intersects those points.
\EEN
% ~~~~~~~~~~~~~~~~~~~~~~~~~~~~~~~~~~~~~~~~~~~~~~~~~~~~~~~~~~~~~~~~~~
% PARTICLE MOTION IN A CIRCLE
\item 
\textbf{Application to Particle Motion}
\BEN 
\item Complete Exercise 16 from Section 1.8 of \textit{Vector Calculus} by M. Corral. \textit{Hint: start by finding an expression for the position of the particle at time $t$}.
\item Find an equation that expresses $c$ in terms of $a$. 
\EEN
% ~~~~~~~~~~~~~~~~~~~~~~~~~~~~~~~~~~~~~~~~~~~~~~~~~~~~~~~~~~~~~~~~~~
\item Choose one device that you use in your everyday life. Describe the roots of its invention, its historical importance, and how it helps in every day life. Also, describe why the invention is important to the engineering field(s) today. Please be clear and use historical examples.
% ~~~~~~~~~~~~~~~~~~~~~~~~~~~~~~~~~~~~~~~~~~~~~~~~~~~~~~~~~~~~~~~~~~
\item From your unit readings in this course and your own knowledge base, elaborate on the place of engineering in society today.
% ~~~~~~~~~~~~~~~~~~~~~~~~~~~~~~~~~~~~~~~~~~~~~~~~~~~~~~~~~~~~~~~~~~
\item In a paragraph, please describe the importance of interdisciplinary engineering today. Please use examples of interdisciplinary fields and works. 

% ~~~~~~~~~~~~~~~~~~~~~~~~~~~~~~~~~~~~~~~~~~~~~~~~~~~~~~~~~~~~~~~~~~
\end{enumerate} % END OF QUESTIONS
%%%%%%%%%%%%%%%%%%%%%%%%%%%%%%%%%%%%%%
%%%%%%%%%%%%%%%%%%%%%%%%%%%%%%%%%%%%%%
%%%%%%%%%%%%%%%%%%%%%%%%%%%%%%%%%%%%%%

% SOLUTIONS
\newpage
\section*{Solutions}

\begin{enumerate}
% ~~~~~~~~~~~~~~~~~~~~~~~~~~~~~~~~~~~~~~~~~~~~~~~~~~~~~~~~~~~~~~~~~~~~~~~~~~~~~~~
\item 
\Emph{Finding the Equation of a Sphere from Four Points}\\
Equation 1.31 from \VCT \ is 
\begin{align*}
   x^2 + y^2 + z^2 + ax + by + cz + d = 0,
\end{align*}
where $a$, $b$, $c$, and $d$ are constants to be determined. Substituting the four points into this equation gives us four equations
\begin{align*}
   0 &= (0)^2 + (0)^2 + (0)^2 + a(0) + b(0) + c(0) + d \\
   0 &= (0)^2 + (0)^2 + (2)^2 + a(0) + b(0) + c(2) + d \\
   0 &= (1)^2 + (-4)^2 + (3)^2 + a(1) + b(-4) + c(3) + d \\
   0 &= (0)^2 + (-1)^2 + (3)^2 + a(0) + b(-1) + c(3) + d 
\end{align*} 
The first equation, which corresponds to the point (0,0,0), simplifies to $d=0$. The second equation reduces to
\begin{align*}
   0 &= (0)^2 + (0)^2 + (2)^2 + a(0) + b(0) + c(2) + d \\
    &= 4 + 2c
\end{align*}
Thus, $c=-2$.  The last equation can be used to find $b$:
\begin{align*}
   0 &= (0)^2 + (-1)^2 + (3)^2 + a(0) + b(-1) + c(3) + d \\
      &= 1 + 9 -b + (-2)(3) \\
      b&= 10 - 6 = 4
\end{align*} 
Thus, $b=4$. The equation that corresponds to the point $(1,-4,3)$, is
\begin{align*}
   0 &= (1)^2 + (-4)^2 + (3)^2 + a(1) + b(-4) + c(3) + d \\
   0 &= 1 + 16 + 9 + a + -4b + 3c + d \\
   0 &= 26 + a  -4(4) + 3(-2) + (0) \\
   0 &= 26 + a  -16 - 6  \\
   0 &= 4+ a 
\end{align*} 
Thus, $a= -4$. The equation of the sphere is
\begin{align*}
   x^2 + y^2 + z^2 - 4x + 4y - 2z = 0.
\end{align*}

% ~~~~~~~~~~~~~~~~~~~~~~~~~~~~~~~~~~~~~~~~~~~~~~~~~~~~~~~~~
% LIMIT
\item 
Take for example the function
\begin{align*}
f(x,y) = \frac{x^2}{x^2+y}
\end{align*}
Along any line $y = mx$, the function becomes
\begin{align*}
\frac{x^2}{x^2+mx} = \frac{x}{x+m}
\end{align*}
As $(x,y) \rightarrow (0,0)$ along any straight line, the function $f(x,y) \rightarrow 0$. However, along the parabolas $y=cx^2$, the function becomes
\begin{align*}
\frac{x^2}{x^2+cx^2} = \frac{1}{1 + c}
\end{align*}
Therefore, the limit as $(x,y) \rightarrow (0,0)$ along any parabola depends on $c$ and is not zero. Therefore the limit 
\begin{align*}
\lim_{(x,y) \rightarrow (0,0)} f(x,y)
\end{align*}
does not exist. 

% ~ ~ ~ ~ ~ ~ ~ ~ ~ ~ ~ ~ ~ ~ ~ ~ ~ ~ ~ ~ ~ ~ ~ ~ ~ ~ ~ ~ ~ ~ ~ ~ ~ ~ ~ ~ ~ ~ ~ ~ 
\item
% PART A
\begin{enumerate}

\item
Since the plane is to be perpendicular to the $xy$-plane, the basis vector \textbf{k} must also lie in the plane. A normal vector to the desired plane is
\begin{align*}
(5,-3,1) \times (0,0,1) &= (-3,-5,0).
\end{align*}
The desired equation for the plane is
\begin{align*}
0&=-3(x-8)+(-5)(y+2)+(0)(z-0)\\
&=-3x -5y +14.
\end{align*}
% ~ ~ ~ ~ ~ ~ ~ ~ ~ ~ ~ ~ ~ ~ ~ ~ ~ ~ ~ ~ ~ ~ ~ ~ ~ ~ ~ ~ ~ ~ ~ ~ ~ ~ ~ ~ ~ ~ ~ ~ 
% PART B
\item
Since the plane is to be perpendicular to the $yz$-plane, the basis vector \textbf{i} must also lie in the plane. A normal vector to the desired plane is
\begin{align*}
(5,-3,1) \times (1,0,0) &= (0-0)\mathbf{i} - (0-1)\mathbf{j}+(0+3)\mathbf{k} \\
&=(0,1,3)
\end{align*}
The desired equation is
\begin{align*}
0&=(0)(x-8)+(1)(y+2)+(3)(z-0)\\
&=y+3z+2.
\end{align*}
% ~ ~ ~ ~ ~ ~ ~ ~ ~ ~ ~ ~ ~ ~ ~ ~ ~ ~ ~ ~ ~ ~ ~ ~ ~ ~ ~ ~ ~ ~ ~ ~ ~ ~ ~ ~ ~ ~ ~ ~ 
% PART C
\item
Since the plane is to be perpendicular to the $xz$-plane, the basis vector \textbf{j} must also lie in the plane. A normal vector to the desired plane is
\begin{align*}
(5,-3,1) \times (0,1,0) &= (0-1)\mathbf{i} - (0-0)\mathbf{j}+(5-0)\mathbf{k} \\
&=(-1,0,5)
\end{align*}
The desired equation is
\begin{align*}
0&=(-1)(x-8)+(0)(y+2)+(5)(z-0)\\
&=-x+5z+8.
\end{align*}
\end{enumerate}
% ~~~~~~~~~~~~~~~~~~~~~~~~~~~~~~~~~~~~~~~~~~~~~~~~~~~~~~~~~
\item % RHOMBUS
Let the sides of the rhombus be $\MB{a}$ and $\MB{b}$, so that the diagonals of the rhombus, $\MB{d}_1$ and $\MB{d}_2$, are
\begin{align*}
  \MB{d}_1 &= \MB{a} + \MB{b}, \quad   \MB{d}_2 = \MB{a} - \MB{b} .
\end{align*}
We want to show that $\MB{d}_1 \perp \MB{d}_2$, which is equivalent to $\MB{d}_1 \cdot \MB{d}_2=0$. We are given that the sides of the rhombus are equal, so 
\begin{align*}
 || \MB{a} || = || \MB{b} ||
\end{align*}
Squaring both sides and additional algebraic manipulation yields
\begin{align*}
 || \MB{a} ||^2 & = || \MB{b} ||^2 \\
 0& =  || \MB{a} ||^2  - || \MB{b} ||^2 \\
 &= (\MB{a}\cdot\MB{a}) - (\MB{b}\cdot\MB{b}) \\
  &= (\MB{a}+\MB{b}) \cdot (\MB{a}-\MB{b}) \\
  &= (\MB{d}_1) \cdot (\MB{d}_2)  
\end{align*}
The dot product of the diagonals is zero, and therefore must be perpendicular.
% ~~~~~~~~~~~~~~~~~~~~~~~~~~~~~~~~~~~~~~~~~~~~~~~~~~~~~~~~~
\item % POINT IN TWO PLANES

We first find the equation of the plane $A$.  Define the vectors
\(\mathbf{a} = \begin{bmatrix} 2 \\ 4 \\ 3 \end{bmatrix},
  \mathbf{b} = \begin{bmatrix} -1 \\ 2 \\ -1 \end{bmatrix},
  \mathbf{c} = \begin{bmatrix} 3 \\ 1 \\ 4 \end{bmatrix}
\), so that the equation of A is given by
\begin{align}
 [(\mathbf{b} - \mathbf{a}) \times (\mathbf{c} - \mathbf{a})] \cdot 
 (\begin{bmatrix}
   \mathbf{x} \\ \mathbf{y} \\ \mathbf{z}
  \end{bmatrix} - \mathbf{a}) &= 0 \notag \\
\Bigg(\begin{bmatrix} -3 \\ -2 \\ -4 \end{bmatrix} \times
  \begin{bmatrix} 1 \\ -3 \\ 1 \end{bmatrix}\Bigg) \cdot
 \begin{bmatrix} x - 2 \\ y - 4 \\ z - 3 \end{bmatrix} &= 0 \notag \\
 \begin{bmatrix} -14 \\ -1 \\ 11 \end{bmatrix} \cdot
 \begin{bmatrix} x - 2 \\ y - 4 \\ z - 3 \end{bmatrix} &= 0 \notag \\
 -14(x - 2) - (y - 4) + 11(z - 3) &= 0 \notag \\
 -14x + 28 - y + 4 + 11z - 33 &= 0 \notag \\
 -14x - y + 11z &= -1 \label{eqn:a}
\end{align}

To find the equation of the plane $B$, define the vectors \(
 \mathbf{d} = \begin{bmatrix} 2 \\ 1 \\ -2 \end{bmatrix},
 \mathbf{e} = \begin{bmatrix} 2 \\ 1 \\ 2 \end{bmatrix}
\).  The equation of $B$, then, is
\begin{align}
 \mathbf{d} \cdot
 (\begin{bmatrix}
   x \\ y \\ z
  \end{bmatrix} - \mathbf{e}) &= 0 \notag \\
 \begin{bmatrix} 2 \\ 1 \\ -2 \end{bmatrix} \cdot
 \begin{bmatrix} x - 2 \\ y - 1 \\ z - 2 \end{bmatrix} &= 0 \notag \\
 2(x - 2) + (y - 1) - 2(z - 2) &= 0 \notag \\
 2x - 4 + y - 1 - 2z + 4 &= 0 \notag \\
 2x + y - 2z &= 1 \label{eqn:b}
\end{align}

To find a point that falls on both $A$ and $B$, we want to find
$x, y, z$ that simultaneously satisfy both (\ref{eqn:a}) and 
(\ref{eqn:b}).  To do so, we form the augmented matrix and apply
the row-reduction algorithm:
\begin{align}
 \begin{amatrix}{3}
  -14 & -1 & 11 & -1 \\
  2 & 1 & -2 & 1
 \end{amatrix}&
  R_1 \leftarrow R_1 + 7R_2
 \nonumber \\
 \begin{amatrix}{3}
  0 & 6 & -3 & 6 \\
  2 & 1 & -2 & 1
 \end{amatrix}&
 \nonumber \\
 \begin{amatrix}{3}
  0 & 1 & -1/2 & 1 \\
  2 & 0 & -3/2 &-1/3
 \end{amatrix}&
 \nonumber
\end{align}

The set of all points that fall on both $A$ and $B$, then, is
given by the parameterized set
\begin{align*}
 x &= -\frac{1}{6} + \frac{7}{10}t \\
 y &= 1 + \frac{1}{4}t \\
 z &= t
\end{align*}
where $t \in \mathbb{R}$.  To find a single point, let $t = 0$:
$(\frac{-1}{6}, \frac{4}{3}, 0)$.

% ~~~~~~~~~~~~~~~~~~~~~~~~~~~~~~~~~~~~~~~~~~~~~~~~~~~~~~~~~
\item % DETERMINANT LIMIT QUESTION
First we expand the $3\times3$ determinant:

\begin{align*}
 f(x,y) &= x^2 \Big(\frac{1}{x-y} \frac{1}{y+x} - 0\Big) - \frac{y}{x-y} \Big(0 - (1) \frac{y}{x+y} \Big) + 0 \\
 &= \frac{x^2 + y^2}{x^2 - y^2}.
\end{align*}

We wish to evaluate 
\begin{align*} 
  \lim\limits_{(x,y) \to (0,0)} \frac{x^2 + y^2}{x^2 - y^2}.
\end{align*}

Let $x = 0$; then 
\begin{align*} 
  \lim\limits_{(x,y) \to (0,0)} \frac{x^2 + y^2}{x^2 - y^2} = \lim\limits_{(x,y) \to (0,0)} \frac{y^2}{-y^2} = -1
\end{align*}
Now let $y = 0$; then 
\begin{align*}
   \lim\limits_{(x,y) \to (0,0)} \frac{x^2 + y^2}{x^2 - y^2} = \lim\limits_{(x,y) \to (0,0)} \frac{x^2}{x^2} = 1.
\end{align*}
Since these two limits aren't equal, the limit does not exist.

% ~~~~~~~~~~~~~~~~~~~~~~~~~~~~~~~~~~~~~~~~~~~~~~~~~~~~~~~~~
\item % QUESTION FROM VECTOR CALCULUS, BEZIER CURVE
\BEN
\item We are given, in the four non collinear point algorithm, that
\begin{align*}
  \MB{b}_0^3 = (1- t)\MB{b}_0^2 + t\MB{b}_1^2(t).
\end{align*}
Substituting the other given expressions into this equation yields
\begin{align*}
  \MB{b}_0^3 &= (1- t)\Big( (1-t)\MB{b}_0^1 + t\MB{b}_1^1 \Big) + t\Big( (1-t)\MB{b}_1^1 + t\MB{b}_2^1 \Big) \\
  &= (1-t)^2\MB{b}_0^1 + t(1-t)\MB{b}_1^1 + t (1-t)\MB{b}_1^1 + t^2\MB{b}_2^1 \\
  &= (1-t)^2\MB{b}_0^1 +2 t(1-t)\MB{b}_1^1 + t^2\MB{b}_2^1 \\  
  &= (1-t)^2\Big( (1-t)\MB{b}_0 +t\MB{b}_1\Big) +2 t(1-t)\Big( (1-t)\MB{b}_1 +t\MB{b}_2\Big) + t^2\Big( (1-t)\MB{b}_2 +t\MB{b}_3\Big) \\    
  &= (1-t)^3 \MB{b}_0 +t(1-t)^2\MB{b}_1+2 t(1-t)^2 \MB{b}_1 +2t^2(1-t)\MB{b}_2 + t^2(1-t)\MB{b}_2 +t^3\MB{b}_3\\
  &= (1-t)^3 \MB{b}_0 + 3 t(1-t)^2 \MB{b}_1 +3 t^2(1-t)\MB{b}_2 +t^3\MB{b}_3
\end{align*}
% ~  ~  ~  ~  ~  ~  ~  ~  ~  ~  ~  ~  ~  ~  ~  ~  ~  ~  ~  ~  ~  ~  ~  ~  ~  ~  ~  ~  ~  ~  ~  ~  
\item Using the given vectors $\MB{b}_0$, $\MB{b}_1$, $\MB{b}_2$, and $\MB{b}_3$, the vector $\MB{b}_0^3$ becomes
\begin{align*}
  \MB{b}_0^3 &= (1-t)^3 \MB{b}_0 + 3 t(1-t)^2 \MB{b}_1 +3 t^2(1-t)\MB{b}_2 +t^3\MB{b}_3\\   
  &= (1-t)^3 \BM 0\\0\\0\EM + 3 t(1-t)^2 \BM 0\\1\\1\EM +3 t^2(1-t)\BM 2\\3\\0\EM +t^3 \BM 4\\5\\2\EM \\   
  &=  \BM 0\\3 t(1-t)^2\\3 t(1-t)^2\EM + \BM 6 t^2(1-t)\\ 9 t^2(1-t)\\0\EM + \BM 4t^3\\5 t^3\\2 t^3\EM \\   
  &=  \BM 6 t^2(1-t) + 4t^3\\3 t(1-t)^2 +  9 t^2(1-t) + 5 t^3\\3 t(1-t)^2 +2t^3\EM  
\end{align*}
\item $\MB{b}_0^3$ is a continuous vector-valued function that passes through $ \MB{b}_0$ when $t=0$, and passes through $ \MB{b}_3$ when $t=1$. 
\EEN
% ~~~~~~~~~~~~~~~~~~~~~~~~~~~~~~~~~~~~~~~~~~~~~~~~~~~~~~~~~
\item % QUESTION FROM VECTOR CALCULUS, PARTICLE MOTION, CIRCLE
\BEN
\item If the particle moves in a circle of radius $a$ in the $xy$-plane, then its position is can be described by the vector-valued function
\begin{align*}
  \MB{r}(t) = a \cos( t)\MB{i} + a \sin( t)\MB{j}
\end{align*}
We are not told where the particle starts and whether it moves in a clockwise or counter-clockwise direction, so there are other vector functions that we could choose. The velocity of the particle is 
\begin{align*}
  \MB{r}'(t) = a \sin( t)\MB{i} + a \cos(t)\MB{j}.
\end{align*}
The acceleration of the particle is 
\begin{align*}
  \MB{r}''(t) = - a \cos( t)\MB{i} - a \sin(t)\MB{j}.
\end{align*}
By comparison, $\MB{r}(t) =  - \MB{r}''(t)$, and therefore, the two vectors are anti-parallel (in other words, they point in the opposite directions  for all $t$). 

% ~  ~  ~  ~  ~  ~  ~  ~  ~  ~  ~  ~  ~  ~  ~  ~  ~  ~  ~  ~  ~  ~  ~  ~  ~  ~  ~  ~  ~  ~  ~  ~  
\item
The speed of the particle is a scalar function, equal to the magnitude of the velocity at time $t$:
 \begin{align*}
  \MB{r}'(t) &= a \sin( t)\MB{i} + a \cos(t)\MB{j} \\
  |\MB{r}'(t)| &= \sqrt{ \big(a \sin( t)\big)^2 + \big(a \cos(t)\big)^2} \\  
  &= |a|\\
  &= a
\end{align*}
But, the speed is $c$. Therefore, $a= c$.
\EEN

\end{enumerate}

\end{document}


% SIMILAR MATRICES QUESTION
%Let $R$ be an $n \times n$ matrix such that $||R\mathbf{v}|| =
%||\mathbf{v}||$ for every $\mathbf{v} \in \mathbb{R}^n$.  What are the
%possible values of the determinant of $R$?  Must $R$ be invertible?
%
%
%Rewriting the vector norms using dot products, we get
%\begin{align}
% (R \mathbf{v}) \cdot (R \mathbf{v})
%  &= \mathbf{v} \cdot \mathbf{v} \notag \\
% (R \mathbf{v})^T (R \mathbf{v}) &= \mathbf{v}^T \mathbf{v} \notag \\
% \mathbf{v}^TR^TR\mathbf{v} &= \mathbf{v}^T\mathbf{v} \label{eqn:det}
%\end{align}
%so that $R^TR = I$; since $R$ is square, this shows that $R$ is
%invertible, and $R^{-1} = R^T$.  (This implies that $R$ is
%orthogonal.)  Now take the determinant of both sides of 
%(\ref{eqn:det}):
%\begin{align*}
% \det(R^TR) &= \det(I) \\
% \det(R^T)\det(R) &= \det(I) \\
% \det(R)^2 &= 1
%\end{align*}
%so that either $\det(R) = 1$ or $\det(R) = -1$.
